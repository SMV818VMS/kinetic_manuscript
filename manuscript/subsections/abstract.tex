In the initial stage of transcription, promoter-bound RNA polymerase (RNAP)
goes through repeated cycles of abortive RNA synthesis before reaching
promoter escape. In contrast to transcription elongation, where RNAP moves
along the DNA template, initial transcription requires DNA to be pulled into
the enzyme via the mechanism of scrunching. This is thought to result in
strain in the initial transcribing complex, which can result in abortive RNA
release or promoter escape. Here, we investigate the kinetics of initial
transcription on the N25 promoter using a novel method that uses abortive
probabilities from bulk experiments to calculate the rate constants of
backtracking. Applying this method within a parameter estimation scheme, where
we fit the model to single molecule measurements of the time RNAP spends in
abortive cycles, we identify the rate constants for the nucleotide addition
cycle and the steps of unscrunching and abortive RNA release. Our results show
that promoter-bound transcription proceeds at the same speed as for
transcription elongation. The rate limiting step for initial transcription is
found to be unscrunching and abortive RNA release, in agreement with previous
experimental evidence. We also find that the best fit by the model to
experimental data shows that the time RNAP spends in abortive cycling does not
follow an exponential distribution for abortive cycles shorter than 3.5
seconds, as had been previously suggested. Rather, a time-delay for
transcribing the first 11 basepairs must be taken into account.  Our findings
represent a detailed view of the kinetics of initial transcription, where
there is no evidence that the strain caused by scrunching affects the speed of
RNA synthesis.
