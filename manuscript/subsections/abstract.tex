\parttitle{Background}
During initial transcription, promoter-bound RNA polymerase (RNAP) pulls the
DNA template through the enzyme's active site via the mechanism of scrunching.
This mechanism induces strain on the initial transcribing complex, which may
provoke either an abortive RNA release or promoter escape, two competing
pathways that determine the rate of initiating RNAP that may proceed to
transcription elongation. Despite detailled experimental investigation in the
recent decade, the rate kinetics of these pathways are not entirely clear.

\parttitle{Results} 
We developed a simple model that describes forward transcription and
backtracking of promoter-bound RNAPs. We combined abortive probability data
acquired from bulk experiments with data for the time RNAP spends in abortive
cycles taken from single molecule measurements on the N25 promoter and
identified the rate constants in the model. Specifically, we found the kinetic
rates of backtracking, the nucleotide addition cycle, and unscrunching and
abortive RNA release. Our results show that the average speed of
promoter-bound transcription is similar to the speed attained after promoter
escape. Additionally, the results suggest that the rate limiting step of
initial transcription is unscrunching and abortive RNA release, consistent
with previous evidence. Moreover, our model predicts that only 4\% of initial
transcription events have a duration less than 1 seconds, in contrast with
previous suggestions (20\%).  

\parttitle{Conclusions}
Taken together, we propose that the strain caused by scrunching does not have
a strong impact on the average speed of transcription by promoter-bound RNAP.
