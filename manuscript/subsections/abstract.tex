In the initial stage of transcription, promoter-bound RNA polymerase (RNAP)
may go through repeated cycles of abortive RNA synthesis before reaching
promoter escape. In contrast to transcription elongation, where RNAP moves
along the DNA template, initial transcription requires DNA to be pulled into
the enzyme via the mechanism of scrunching. This is thought to result in
strain in the initial transcribing complex, which when released can result in
abortive RNA release or promoter escape. Here, we investigate the kinetics of
initial transcription on the N25 promoter using a novel method that uses
abortive probabilities from bulk experiments to calculate the rate constants
of backtracking. Applying this method within a parameter estimation scheme,
where we fit the model to single molecule measurements of the time RNAP spends
in abortive cycles, we identify rate constants for the nucleotide addition
cycle and unscrunching and abortive RNA release. Our results show that
promoter-bound transcription proceeds close to the same average speed as for
transcription elongation. We identified the rate limiting step for initial
transcription to be unscrunching and abortive RNA release, in agreement with
previous experimental evidence. We also found that the best fit by the model
to experimental data predicts that the time RNAP spends in abortive cycling
does not follow an exponential distribution for abortive cycles or scrunches
shorter than 2.5 seconds, as had been previously suggested. Rather, a
time-delay for transcribing the first 11 basepairs must be taken into account.
Our findings suggest that the strain caused by scrunching does not have a
strong impact on the average speed of transcription by promoter-bound RNAP.
