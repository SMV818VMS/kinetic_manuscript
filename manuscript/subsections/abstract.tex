In the initial stage of gene expression, promoter-bound RNA polymerase (RNAP)
goes through repeated cycles of abortive RNA synthesis before reaching
promoter escape. In contrast to transcription elongation, where RNAP moves
along the DNA template, initial transcription requires DNA to be pulled into
the enzyme via the mechanism of scrunching. This is thought to result in
strain in the initial transcribing complex, which may affect abortive RNA
release and promoter escape. Here, we investigate the kinetics of initial
transcription on the N25 promoter using a novel method that uses abortive
probabilities found from bulk experiments to calculate the rate constants of
backtracking. Applying this method within a parameter estimation scheme, where
we fit the model to the time RNAP spends in abortive cycles as obtained from
single molecule experiments, we identify the rate constants for the nucleotide
addition cycle and the steps of unscrunching and abortive RNA release. Our
results suggest that promoter-bound transcription proceeds at the same speed
as for transcription elongation. This implies that the strain caused by
scrunching does not affect the speed of initial transcription. The best fit by
the model to experimental data shows that the time RNAP spends in abortive
cycling does not follow an exponential distribution for abortive cycles
shorter than 3.5 seconds, as had been previously suggested. Rather, a
time-delay for transcribing the first 11 basepairs must be considered. We
conclude that the rate limiting step for initial transcription is unscrunching
and abortive RNA release, in agreement with experimental evidence. 
