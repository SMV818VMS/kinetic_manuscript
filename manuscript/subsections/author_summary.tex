The enzyme RNA polymerase (RNAP) is responsible for reading the genetic code.
It travels along DNA, producing RNA molecules that may in turn be used to make
protein. To begin reading DNA, RNAP must first bind tightly to a promoter
region of DNA. Since it is bound to the promoter, RNAP cannot immediately
travel along DNA when it begins to produce RNA. Instead, when the first part
of RNA is made, DNA must be pulled into RNAP. As more and more DNA is pulled
in, strain builds up. One way this strain can be released is if DNA is
released again from within the enzyme. When that happens, RNA becomes released
from RNAP and RNA production must start over. This all happens very fast,
which has made it difficult to measure the speed of the process. In this
paper, we have made a kinetic model for the repeated production and release of
short RNAs from promoter bound RNAPs. By comparing with experimental data, we
find that promoter bound RNAP makes RNA at the same speed as promoter-free
RNAP. This implies that the strain of pulling in DNA does not affect the
reaction steps involved in RNA synthesis.
