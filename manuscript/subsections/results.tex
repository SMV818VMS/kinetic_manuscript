%\addbibresource{/home/jorgsk/Dropbox/phdproject/bibtex/jorgsk.bib}
\subsection{Transcription proceeds at the same speed for initial transcription
as for transcription elongation}

We identified the rate constants for initial transcription using a two-step
method. In the first step we performed a broad search by considered the
minimum and maximum values of all rate constants to lie between 1 s$^{-1}$ and
25 s$^{-1}$. By evaluating the model within these values and selecting the
best fitting 1\% of all simulations (see Materials and Methods), we found that
the rate constant for NAC is greater than 7 s$^{-1}$; that the rate
constant for unscrunching and abortive RNA release lies between 1 and 3
s$^{-1}$; and that the rate constant for promoter escape should be greater
than 2.3 s$^{-1}$ (Figure~\ref{fig:parameter_estimation_proper}A), but the
distribution does not show any clear optimal value. Of these three, the
distribution of values for the rate constant of promoter escape does not
show an optimal value, which indicates that time spent in abortive cycling is
not sensitive to the rate constant for promoter escape
(Figure~\ref{fig:parameter_estimation_proper}A).

Having narrowed down the boundaries for the rate constants, we performed a
more focused search in the second step. Since model fitting was insensitive to
the rate constant of promoter escape
(Figure~\ref{fig:parameter_estimation_proper}A), we varied only the rate
constants for the NAC and unscrunching and abortive RNA release in this second
round. The rate constant of promoter escape constant was set to the same as
the NAC. The top 1\% best fitting results from the second round of rate
constant estimation resulted in clear peaks in the distributions for both rate
constants. The rate constant for the NAC showed a peak at 10.4 s$^{-1}$
(Figure \ref{fig:parameter_estimation_proper}B), which compares well with the
speed of transcription after promoter escape of 9.2 nt/s (positively
supercoiled DNA) and 13.3 nt/s (negatively supercoiled DNA) (Revyakin et al.
Figure S14 \cite{revyakin_abortive_2006}). This indicates that the speed of
the NAC is unaffected by both RNAP's attachment to the promoter and DNA
scrunching. The rate constant for unscrunching and abortive release showed a
peak around 1.3 s$^{-1}$, nearly 10 times slower than the NAC (Figure
\ref{fig:parameter_estimation_proper}B). This lower value is in agreement with
experimental evidence indicating that unscrunching and abortive RNA release
are the rate limiting steps for initial
transcription~\cite{revyakin_abortive_2006, margeat_direct_2006}. The optimal
parameters from the peak of distributions result in a good match between model
and measurements (Figure \ref{fig:revyakin_fit}). A full kinetic scheme of the
identified rate constants is given in Figure \ref{fig:estimated_parameters}A.

\begin{figure}
	\begin{center}
      \includegraphics[scale=0.8]{../figures/coarse_and_finegrain_search_GreB_no_extrap.pdf}
	\end{center}
    \caption{Distributions of rate constants after fitting 100000 randomly
      sampled values and selecting the 1000 (1\%) best fits. \textbf{A}: First round
      of screening, where all three rate constants were varied between 1 and
      25 s$^{-1}$. \textbf{B}: Second round of screening, where NAC was varied
      between 6 and 16 s$^{-1}$ and unscrunching and abortive RNA release
      (UAR) was varied between 1 and 3 s$^{-1}$. The promoter escape rate
      constant was held constant at 20 s$^{-1}$ for the second round.}
      \label{fig:parameter_estimation_proper}
\end{figure}

\begin{figure}
	\begin{center}
      \includegraphics[scale=1.0]{../figures/coarse_search-fits_2_3.pdf}
	\end{center}
    \caption{
      Rate constant distributions obtained by fitting the model to
      extrapolated and inconsistent experimental data. \textbf{A}:
      Distributions of top 1\% rate constants after fitting model to
      extrapolated curve for the distribution of time spent in abortive
      cycling \cite{revyakin_abortive_2006}. \textbf{B}: Distributions of top
      1\% rate constants after fitting model using AP values obtained in the
      absence of GreB. In silhouette is shown the distribution obtained with
      matching experimental input from Figure \ref{fig:parameter_estimation_proper}}
      \label{fig:extrap_and_GreB_minus_fit}
\end{figure}

\subsection{Time spent in short durations of abortive cycling does not follow
    an exponential distribution}
Revyakin et al.\ did not measure cycles of abortive initiation lasting shorter
than ~3.5 seconds, but proposed based on the extrapolation curve that 20\% of
abortive cycles are shorter than one second \cite{revyakin_abortive_2006}. In
contrast, the best fit by our kinetic model suggests that
less than 2\% of abortive cycling events last shorter than one second, since
~1 second is required by the model for transcribing the first 11 basepairs
when the rate constant of the NAC is 10.4 s$^{-1}$ (Figure
\ref{fig:revyakin_fit}). However, there could also be mechanisms of initial
transcription not yet understood that would make the exponential extrapolation
valid. To test the implication of assuming the exponential extrapolation, we
repeated the rate constant estimation procedure, but fitted to the
extrapolated curve instead of the measurements. This resulted in a similar
optimal rate constant for unscrunching and abortive RNA release, but a much
larger value for the peak for the NAC was found at 23.4 s$^{-1}$ (Figure
\ref{fig:extrap_and_GreB_minus_fit}), which represents what is required in the
model for 20\% of initial transcription events to last shorter than 1 second.

The validity of such a high rate constant for the NAC may be questioned. A
value of 23.4 s$^{-1}$ is twice as high as that reported for transcription
elongation after promoter escape \cite{revyakin_abortive_2006}. Further, it is
higher than the speed of transcription elongation obtained for substrate
NTP-concentrations ten times as high \cite{bai_mechanochemical_2007}.
Together, this indicates that such a high value for the NAC is
unphysiological, which suggests that it is not valid to extrapolate the
experimental data with an exponential function for abortive cycling lasting
shorter than 3.5 seconds.

\begin{figure}
	\begin{center}
      \includegraphics[scale=0.8]{../illustrations/estimated_parameters_filled.pdf}
	\end{center}
    \caption{Estimated rate constants for initial transcription on the N25
        promoter. The values for backtracking are calculated from the value of
        the NAC and position specific APs using equation
        \eqref{eq:backtrackingcalc}. \textbf{A}: Rate constants of
        backtracking calculated from APs from +GreB experiments. \textbf{B}:
        Rate constants of backtracking calculated from APs from -GreB
        experiments.}
    \label{fig:estimated_parameters}
\end{figure}

\subsection{Estimated rate constants are valid for initial transcription also
in the absence of GreB}
Since the experiments from Revyakin et al. were performed in the presence of
GreB, we originally fitted the model using APs calculated from experiments
also performed in the presence of GreB. However, once fitted, the estimated
constants should be generally valid for initial transcription on the N25
promoter, both in the absence and presence of GreB. When GreB is absent, APs
increase \cite{hsu_initial_2006}, which in our model transfers into larger
rate constants for backtracking (Figure
S\ref{fig:parameter_estimation_scheme}). To test if the rate constants are
valid in the absence of GreB, we compared the model with the half-life of N25
full length (FL) transcript synthesis in the absence of GreB using kinetic data
from Figure 3B in Vo et al.\ \cite{vo_vitro_2003-1}. We fitted the data from
Vo et al.\ to a single exponential function to obtain a half life of FL
transcript synthesis of 13.7 s$^{-1}$. We then ran our model with -GreB AP
values using the rate constants estimated for +GreB AP values, which resulted
in generally higher rate constants of backtracking due to the higher AP values
associated with -GreB. This resulted in a model-predicted half-life of FL
transcript synthesis of 16.7 s$^{-1}$ (Figure~\ref{fig:vo_comparison}), which
is close to value obtained from the experimental data. This shows that the
estimated rate constants are valid for the kinetics of initial transcription
both in the presence and in the absence of GreB, which indicates that they are
valid in general. The kinetic scheme of the estimated rate constants for
initial transcription on the N25 promoter in the absence of GreB is given in
Figure \ref{fig:estimated_parameters}B.

\subsection{Model performance is highly sensitive to inconsistency in experimental data}
When fitting the model we combined experimental data for time spent in
abortive cycling and APs, where the two datasets were GreB-consistent, in the
sense that they were obtained from transcription experiments performed in the
presence of GreB. If the model correctly represents the kinetics of initial
transcription, the estimation of rate constants should be sensitive to the
consistency of the experimental data. To test this, we re-fitted the
model using the -GreB APs instead of the +GreB APs. This resulted in markedly
different distributions of the rate constants
\ref{fig:extrap_and_GreB_minus_fit}B, with the peak of the distributions being
23.6 s$^{-1}$ and 14.7 s$^{-1}$ for the NAC and unscrunching and abortive RNA
release, respectively. This shows that model is sensitive to inconsistent
experimental data during rate constant estimation.

\begin{figure}
    \begin{center}
      \includegraphics[scale=0.7]{../figures/scrunch_times_cumulative_.pdf}
    \end{center}
    \caption{Comparison of model and measurements of time spent in abortive
        cycling before promoter escape. The mean has been found from 100
        simulations of 100 transcription events, with the shaded region
        indicating one standard deviation. Measurements and exponential fit
        are obtained from Revyakin et al. \cite{revyakin_abortive_2006}. The
        left green area indicates the region where abortive cycling would last
        shorter than one second.}
\label{fig:revyakin_fit}
\end{figure}

\begin{figure}
    \begin{center}
        \includegraphics[scale=0.7]{../figures/GreB_minus_kinetics.pdf}
    \end{center}
    \caption{Kinetics of FL transcript synthesis under -GreB conditions.
      Comparison between model and measurements from Vo
      et al. \cite{vo_vitro_2003-1}. To simulate the bulk experiment, we
      used a sufficiently large number of RNAPs (1000) for which there is low
      variability due to the stochastic simulations. To higlight the low
      variability, the mean and standard deviation (shaded region) from 10
      separate simulations are shown.}
\label{fig:vo_comparison}
\end{figure}
