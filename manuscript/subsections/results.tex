%\addbibresource{/home/jorgsk/Dropbox/phdproject/bibtex/jorgsk.bib}
\subsection{Transcription proceeds at the same speed for initial transcription
as for transcription elongation}

We simulated our model and found upper and lower bounds for the parameters
(see Materials and Methods). We found that the minimum feasible value of the rate constant for NAC was 7 s$^{-1}$, and that the rate constant for unscrunching and abortive RNA
release lies between 1 and 3 s$^{-1}$
(Figure~\ref{fig:parameter_estimation_proper}A). We found that although the
rate constant for promoter escape was greater than 2.3 s$^{-1}$, there was no optimal value that yielded a minimal RMSE (Figure 2A, right panel). This lack of optimal value for the kinetics of the promoter
escape $k_e$  indicates that the time spent in abortive cycling is not sensitive to this parameter. Therefore, for the estimation of the remaining parameters $k_n$ and $k_u$, we assumed a fixed value of $k_e=2.3$.  

In the next iteration of the parameter estimation, we selected 1\% of the $k_n$ and $k_u$ values that yielded the  lowest RMSE. Optimal NAC rate constant $k_n$ revealed a peak around 10 s$^{-1}$ (Figure
\ref{fig:parameter_estimation_proper}B), with the mode of the distribution [HVA ER MODE OF DIST? JEG SYNES VI KAN DROPPE DET OG HOLDE OSS TIL TOPPEN PÅ 10] at
10.4 s$^{-1}$ (Table \ref{tab:param_fit_revyakin}). This closely matches the
measured speed of transcription after promoter escape of 9.5 nts$^{-1}$ EHEM! DE BRUKER NT/SEK, VI BRUKER BARE 1/SEK?? DET ER 2 FORSKJELLIGE TING. BEGGE HASTIGHETER, MEN SOM STÅR HER, ER IKKE SAMMENLIGNBAR, ELLER HVA MENER DU? 
(Revyakin et al. Figure S14 \cite{revyakin_abortive_2006}), indicating that
the speed of the NAC is unaffected by both RNAP's attachment to the promoter
and DNA scrunching. The rate constant for unscrunching and abortive release
showed a peak around 1.3 s$^{-1}$, nearly 10 times slower than the NAC (Figure
\ref{fig:parameter_estimation_proper}B, Table \ref{tab:param_fit_revyakin}).
This low value is in agreement with experimental evidence indicating that
unscrunching and abortive RNA release are the rate limiting steps for initial
transcription~\cite{revyakin_abortive_2006, margeat_direct_2006}. A full
kinetic scheme highlighting the identified rate constants is shown in Figure 
\ref{fig:estimated_parameters}A. HVA SKJEDDE MED FIGURE 3? DU KAN IKKE HOPPE FRA 2 TIL 4 OG 5...  JEG BYTTET OM FIGURENE.

\begin{figure}
	\begin{center}
      \includegraphics[scale=0.8]{../figures/coarse_and_finegrain_search_GreB_no_extrap.pdf}
	\end{center}
    \caption{[MANGLER Y AXIS ENHETER, VANLIGVIS ANTALL SIMULERINGER ELLER FORSØK] Distributions of rate constants after fitting 100000 randomly
      sampled values and selecting the 1000 (1\%) best fits. \textbf{A}: First round
      of screening, where all three rate constants were varied between 1 and
      25 s$^{-1}$. \textbf{B}: Second round of screening, where NAC was varied
      between 6 and 16 s$^{-1}$ and unscrunching and abortive RNA release
      (UAR) was varied between 1 and 3 s$^{-1}$. The promoter escape rate
      constant was held constant at 20 s$^{-1}$ for the second round. [IKKE SIKKER PÅ OM HVILKEN TIDSKRIFT VIL AKSEPTERE SKYGGENE MED GRID PÅ FIGURENE SOM DU HAR LAGET HER. DET ER NOK Å GENERERE HIST(X,N), DVS KOLONNER PÅ HVIT BAKGRUNN...}
      \label{fig:parameter_estimation_proper}
\end{figure}
ABOUT FIGURE 2: YOU DO NOT HAVE Y AXIS?? THIS IS HIGHLY IRREGULAR, ALTHOUGH YOU SHOW DISTRIBUTION. ALSO WHEN YOU PRESENT HISTOGRAM YOU MUST SHOW THE DISTRIBUTION (EITHER # SIMULATIONS, OR NORMALIZED 0-1, OR WHATEVER.). SECONDLY, FIRST ROUND AND SECOND ROUND OF SCREENING? I AM A BIT SKEPTICAL TO CALL THAT, I HAVE NEVER SEEN THIS IN PARAMETER IDENTIFICATIONS. MAYBE WE CAN CALL IT FIRST ITERATION, SECOND ITERATION? IT IS NOT UNCOMMON THAT ONE ITERATE IN THE ESTIMATION PROBLEM, NARROWING DOWN THE RANGES UNTIL CONVERGENCE TO AN ISOLATED POINT OR A SET. 

\begin{figure}
	\begin{center}
      \includegraphics[scale=0.8]{../illustrations/estimated_parameters_filled.pdf}
	\end{center}
    \caption{[JEG ANBEFALER Å SKRIVE NUCLEOTIDER AV N25 OVENFOR A OG B SLIK AT DET BLIR A B C. Estimated rate constants for initial transcription on N25 (see text for the rate constant values). The values for backtracking are calculated
      from the value of the NAC using equation \eqref{eq:backtrackingcalc}.
      \textbf{A}: Rate constants of backtracking calculated from APs 
      from +GreB experiments. \textbf{B}: Rate constants of
      backtracking calculated from APs from -GreB experiments.}
    \label{fig:estimated_parameters}
\end{figure}

\begin{figure}
    \begin{center}
      \includegraphics[scale=0.7]{../figures/cumul_scrunch_fit.pdf}
    \end{center}
    \caption{Initial transcription model fitted to distribution of time spent
      in rounds (HVA ER IN ROUNDS??) of abortive cycling before promoter escape. Measurements were obtained
      from Revyakin et al. \cite{revyakin_abortive_2006}. The function that minimized the summed square of residual to the measurements was an exponential function. The green area indicates the
      region where the time period of abortive cycling is shorter than one second.}
\label{fig:revyakin_fit}
\end{figure}

Figure \ref{fig:revyakin_fit} shows the fit of the model (solid curve) using the best
estimated rate constants ($k_n=10.4$, $k_u=1.3$ and $k_e=20$).   [from Table \ref{tab:param_fit_revyakin} VI TRENGER IKKE TABELLEN, DEN HAR BARE EN RAD]. There is a
close match between model and measured values. Also shown in Figure
\ref{fig:revyakin_fit} is an extrapolation of the measurement data below 3.5 seconds, originally
by Revyakin et al. \cite{revyakin_abortive_2006}. The authors did
not measure PROBABILITY OF ABORTIVE CYCLING SHORTER THAN 3 SEC cycles of abortive cycling [CYCLES OF ABORTIVE CYCLING? ER DET IKKE FEIL?] lasting shorter than 3.5 seconds, but
proposed based on this extrapolation curve that 20\% of abortive cycles are
shorter than one second \cite{revyakin_abortive_2006}.  In contrast, our kinetic model suggests that less than 2\% of
abortive cycling rounds ROUNDS IGJEN were shorter than one second (Figure
\ref{fig:revyakin_fit}). We hypothesized that if the extrapolation of the data reflects the true
scrunch-time distribution, then more accurate rate constants should be obtained by
fitting the model to the extrapolated data rather than the method we used above. To
test this, we repeated the rate constant estimation procedure, but minimized the RMSE of the extrapolated curve instead of the measurements [FOR LOW VALUES ONLY? YOU WILL HAVE TO EXPLAIN THIS TO ME, BECAUSE IT IS UNCLEAR. WHEN YOU FIT THE DATA YOU FIT THE WHOLE RANGE, NOT ONLY THE LOW TIME VALUES?]. We found that the estimated rate constant for unscrunching and abortive RNA release remained
similar, but the mean [HERE, GIVE THE MEAN, NOT ONLY THE PEAK] NAC rate
constant yielded significantly higher estimate, [HERE DONT SAY MOST, GIVE THE (ROUND) PERCENTILE, FOR INSTANCE, 90\% OF THE POPULATION WERE GREATER THAN (THRESHOLD) 10, OR 95\% WERE GREATER THAN 15... SEND ME THE RESULTS AND i WILL EASILY CHECK THAT IF YOU WANT] with XX\% of the simulation resulted in values above 10
s$^{-1}$, and with peak of 23.4 s$^{-1}$ (Figure
\ref{fig:extrap_and_GreB_minus_fit}A). As a consequence, the extrapolated data generated  NAC rate values substantially above
the values been observed for transcription elongation
\cite{revyakin_abortive_2006}, which contradict our knowledge [DOEN IT? DO WE REALLY KNOW THAT?  NOBODY HAVE MEASURED THE NAC DURING INITIATION, ONLY DURING ELONGATION, SO HOW CAN WE BE SURE THAT NAC AT INITIATION HAS TO BE THE SAME (APPROX) TO THE ELONGATION? PLEASE EXPLAIN THIS BETTER, OTHERWISE YOU CANNOT REJECT THE HYPOTHESIS]. 


\begin{figure}
	\begin{center}
      \includegraphics[scale=1.0]{../figures/coarse_search-fits_2_3.pdf}
	\end{center}
    \caption{REF TIL DENNE FIGUREN VAR FEIL, JEG FLYTTET DEN HIT.  [MANGLER Y AXIS VERDIER, FJERN SKYGGEN, FJERN GRID. ]
      Rate constant distributions obtained by fitting the model to
      extrapolated experimental data (see text). 
      Here we show the distributions of top 1\% rate constants after fitting the model to \textbf{A}:
      extrapolated curve for the time spent in abortive
      cycling \cite{revyakin_abortive_2006}, and \textbf{B}:  AP values obtained in the
      absence of GreB. Compare to the distribution from Figure \ref{fig:parameter_estimation_proper} (gray lines)}
      \label{fig:extrap_and_GreB_minus_fit}
\end{figure}

\subsection{Estimated rate constants for initial transcription are valid also
in the absence of GreB}
Since the experiments from Revyakin et al. were performed in the presence of
GreB (denoted as +GreB), we estimated the parameters using AP values calculated from bulk experiments performed
in the presence of GreB (denoted as -GreB). However, once calculated, the rate constants we found should be valid for initial
transcription on the N25 promoter, regardless of the presence of
GreB. To test this, we
compared model predictions to the kinetics of N25 full length (FL) transcript
synthesis experimentally found for -GreB \cite{vo_vitro_2003-1}. In the
absence of GreB, AP values were increased \cite{hsu_initial_2006}, which in our model
translates into higher rates of backtracking compared to the +GreB case (Figure
S\ref{fig:parameter_estimation_scheme}). To facilitate a comparison with the
time series from Vo et al., we fitted their data to a single exponential
function to obtain a half life of FL transcript synthesis of 13.7 s. We then
ran the model on -GreB AP values using our original rate constants. This resulted in a model-predicted half-life of
FL transcript synthesis of 15.5 s (Figure~\ref{fig:vo_comparison}), closely
matching the experimental value. This shows that the rate constants we found are valid for the kinetics of initial
transcription both in the presence and in the absence of GreB. The kinetic
scheme of the estimated rate constants for initial transcription on the N25
promoter in the absence of GreB is given in Figure
\ref{fig:estimated_parameters}B.

\subsection{Parameter estimation is highly sensitive to the consistency of experimental data}
When fitting the model we combined experimental data for time spent in
abortive cycling and APs, where both sources of data were consistent in that
they were obtained from initial transcription experiments performed in the
presence of GreB. If the model correctly represents the kinetics of initial
transcription, parameter estimation should be sensitive to the consistency of
experimental data [INCONSISTENCY, SOM HVA? STØY? HVORFOR EGENTLIG SJEKKER VI IKKE DA HVIT STØY (GAUSSIAN) AV DATA, OG KONSISTENSEN AV ESTIMERINGEN? JEG HAR MYE ERFARING MED DETTE, BÅDE FRA KYB OG FRA SISTE ÅRENE]. To test this, we re-fitted the model by using the -GreB APs
instead of the +GreB APs. This introduces a mismatch between the experimental
data from Revyakin et al. to which the model is fitted (+GreB) and the
experimental data used to calculate the rate constant of backtracking (-GreB).
This re-fitting resulted in substantially higher rate 
constants NAC $k_n$ and $k_u$ (Figure \ref{fig:extrap_and_GreB_minus_fit}B). These values do not appear to be feasible [WHY?? THEY WILL ASK].  [I THINK WE SHOULD REWRITE THIS SUBSECTION WITH INDUCED WHITE NOISE, AND CHECK CONSISTENCY OF THE MODEL. THAT IS MORE ACCEPTABLE APPROACH].  

% \begin{table}
%   \begin{center}
%     \begin{tabular}{ccc}
%        \toprule
%        NAC & UAR & Promoter escape \\
%        $10.4$ & $1.3$ & $20$ \\
%     \end{tabular}
%   \end{center}
%   \caption{Fitted rate constants of initial transcription. The values are the
%     modes of the distributions in Figure \ref{fig:parameter_estimation_proper}B.}
%   \label{tab:param_fit_revyakin}
% \end{table}




\begin{figure}
    \begin{center}
        \includegraphics[scale=0.7]{../figures/vo_greb_minus_comparison.pdf}
    \end{center}
    \caption{Kinetics of FL transcript synthesis under -GreB conditions.
      Comparison between model (light green line) and measurements from Vo
      et al. \cite{vo_vitro_2003-1} (green dots). Blue line shows fit of
      measurements to a single exponential function.}
\label{fig:vo_comparison}
\end{figure}
