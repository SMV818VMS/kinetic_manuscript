%\addbibresource{/home/jorgsk/Dropbox/phdproject/bibtex/jorgsk.bib}
\subsection{The forward pathway of transcription proceeds at the same rate
for initial transcription as for elongation}
Revyakin et.\ al measured initial transcription on the N25 promoter in the
presence of GreB and 100 $\mu$M NTP and found that abortive cycling lasts on
average 5 seconds prior to promoter escape~\cite{revyakin_abortive_2006}. We
constructed a kinetic model of initial transcription and fitted it to this
distribution, leveraging the APs quantified from +GreB experiments of initial
transcription by Hsu et. al~\cite{hsu_initial_2006}
(Figure~\ref{fig:revyakin_fit}). The rate constants that resulted in 
optimal fit with data were as follows:~NAC 10.5 s$^{-1} \pm 0.64$; escape
11.84 s$^{-1} \pm 1.81$; and unscrunching 1.58 s$^{-1} \pm 0.14$ (Figure
\ref{fig:parameter_estimation}, Table~\ref{tab:param_fit_revyakin}). The value
for NAC before promoter escape is highly similar to transcription velocity
after promoter escape: 9.5 nt s$^{-1}$ for positively supercoiled DNA and 13.8
nt s$^{-1}$ for negatively supercoiled DNA~\cite{revyakin_abortive_2006}. This
indicates that NAC proceeds at the same rate for both scrunching and
transcription elongation. The comparatively lower rate of 1.58 s$^{-1}$ for
unscrunching and abortive RNA release is consistent with previous reports that
this process is rate limiting for initial
transcription~\cite{margeat_direct_2006, revyakin_abortive_2006}. Revyakin
et.\ al did not measure rounds of abortive cycling that lasted shorter than 1
second, but proposed based on extrapolation of their data that 20\% of
abortive cycles last shorter than one second \cite{revyakin_abortive_2006}.
The best fit to the data by our kinetic model suggests that less than 2\% of
abortive cycling rounds last shorter than 1 second (Figure
\ref{fig:revyakin_fit}). This number can be understood by considering that
with a transcription rate of $10.5$~nt s$^{-1}$ it takes around one second to
reach promoter escape on N25 if backtracking is avoided. Indeed, when fitting
the model to the extrapolation of the Revyakin et. al data, a rate constant
for the NAC of 18.4 s$^{-1} \pm 3.8$ was obtained
(Table~\ref{tab:param_fit_revyakin}), which is higher than what is reported
for transcription elongation for the same experiment
\cite{revyakin_abortive_2006}.

\begin{figure}
	\begin{center}
      \includegraphics[scale=0.8]{../illustrations/estimated_parameters.pdf}
	\end{center}
    \caption{Estimation rate constants for +GreB N25. (NOTE: values are just
    placeholders for now)}
    \label{fig:parameter_estimation}
\end{figure}

\subsection{Estimated rate constants are also valid for -GreB kinetics}
We estimated the rate constants of NAC, escape, and backtracking and abortive
RNA release using APs associated with +GreB. However, once fitted, the rate
constants should be generally valid. The fitted model should therefore be able
to capture the kinetics of transcription on N25 for different APs, such as the
one obtained for transcription under -GreB conditions, for which the abortive
probability is considerably higher \cite{hsu_initial_2006}. To test this, we
used the estimated rate constants to simulate initial transcription for -GreB
conditions. We compared the simulation result with the kinetics of FL
transcript synthesis reported by Vo et.\ al \cite{vo_vitro_2003-1}. The
half-life of FL transcript synthesis, denoted $\tau$, was 15.5 s in the
simulation, and determined to be 13.7 s in the experiment after fitting
measurements to a single exponential (Figure~\ref{fig:vo_comparison}). While these
numbers appear close, there is no other kinetic data for the N25 promoter in
the literature which can tell us the range of values of $\tau$ for initial
transcription on N25. To therefore be able to say something about the
variability of $\tau$, calculated this value from simulations of initial
transcription on N25-anti and N25 A1-anti, two ITS variants of N25 which have
previously been shown to vary compared to N25 in several quantitative
parameters for initial transcription~
\cite{hsu_initial_2006,chan_anti-initial_2001,kammerer_functional_1986}. This
resulted in values of $\tau$ of 61.2 s for N25-anti and 179.6 s for N25
A1-anti, which agrees with these ITS variants having a lower productive yield
than N25 \cite{hsu_initial_2006}. This shows that the experimentally obtained
and simulated values for $\tau$ for N25 -GreB are similar relative to values
(Figure \ref{fig:its_variant_fl_comparison}).

\subsection{Estimated rate constants are sensitive to changes to the +GreB 
APs}
In estimating the rate constants, we have used APs from bulk studies by Hsu
et. al \cite{hsu_initial_2006} together with an abortive cycling distribution
obtained from single-molecule studies by Revyakin et. al
\cite{revyakin_abortive_2006}. This relies on the assumption that the
position-specific probability to abort initial transcription is the same for
the single-molecule experiment as as for the bulk studies. That this holds
true is not known, since the rapid kinetics of initial transcription
has prevented direct observation of the exact position of backtracking and abortive
RNA release \cite{margeat_direct_2006, revyakin_abortive_2006}. One way to
assess the similarity of the two abortive profiles is to quantify the
sensitivity of the estimated rate constants to changes in the APs. We would
argue that the rate constants in Table~\ref{tab:param_fit_revyakin},
specifically the NAC and backtracking and abortive RNA release, are consistent
available literature; therefore, if we can find a dose-response tendency
where modifying the AP values from their +GreB baseline leads to
unphysiological rate constants or otherwise worsened model performance, this
is an indication that the original bulk AP values are also descriptive for the
single-molecule experiment. The first kind of perturbation we tested was to
replace the +GreB APs with -GreB APs, and then repeat the estimation
procedure. This led to an estimate for the rate constant of NAC of 20.9
s$^{-1} \pm 2.9$ (Table~\ref{tab:param_fit_revyakin}). We consider this value
unphysiological since it would suggest that NAC proceeds nearly twice as fast during
scrunching compared to elongation, proceeding at speeds normally found for NTP
concentrations of 1000 $\mu$M~\cite{bai_mechanochemical_2007}, 10 times higher
than actual experimental conditions. Next, we performed a more general
perturbation of the AP values, making them both higher and lower across the
ITS (Figure~\ref{fig:aps_after_adjustment}). This showed first of all that the fit is
best for zero or small perturbations, and gets progressively worse as
perturbations increase (Figure \ref{fig:ap_adjustment}A), which shows that the
original bulk +GreB AP values lie in an optimal region in terms of fitting the
the single-molecule experimental data. The effect of perturbations on the rate
constants were as follows: for higher AP, NAC is forced to increase to
maintain optimal fit, while NAC decreases when AP is lowered, (Figure
\ref{fig:ap_adjustment}B); the rate constant for unscrunching
and abortive release similarly increased as AP increases, to the point where
this step becomes nearly as rapid as the NAC, but for lower AP this value was
not much removed from the baseline (Figure \ref{fig:ap_adjustment}C). The rate
constant for promoter escape was insensitive to changes in AP, indicating that
model performance is insensitive to this parameter (Figure
\ref{fig:ap_adjustment}D). We therefore observed that for an increased AP, the
rate constant for NAC increased to the point of being unphysiological at more
than 20 $s^{-1}$, and the rate constant for the abortive step increased to the
point of being similar in size to that of the NAC, where it is hardly rate
limiting for the process. However, for a reduced AP there is no published data
by which to judge the rate constants as unphysiological. Therefore, we wished
to assess how the rate constants optimal for low APs would affect model
performance. To do so, we used these rate constants to obtain $\tau$ for -GreB
APs, and calculated the size of this value relative to the one obtained from
measurements from Vo et. al \cite{vo_vitro_2003-1}, where we denote the
relative value as $\tau_r$. This showed that when AP is reduced compared to the
original +GreB values, $\tau$ becomes increasingly large comparing to 
experimental value (Figure \ref{fig:ap_adjustment}E), showing that model
performance worsens when AP is lower than the baseline +GreB values.

\begin{table}
  \label{tab:param_fit_revyakin}
  \caption{Fitted rate constants of initial transcription}
  \begin{center}
    \begin{tabular}{lccc}
       \toprule
       & NAC & Escape & Abort \\
       +GreB & $10.1 \pm 0.6$ & $14.6 \pm 1.8$ & $1.6 \pm 0.1$ \\
       -GreB & $20.9 \pm 2.1$ & $15.3 \pm 5.3$ & $17.5 \pm 5.4$ \\
       Extrapolated fit & $18.4 \pm 3.8$ & $16.6 \pm 6.9$ & $1.8 \pm 1.1$ \\
    \end{tabular}
  \end{center}
\end{table}

\begin{figure}
    \begin{center}
      \includegraphics[scale=0.7]{../figures/cumul_scrunch_fit.pdf}
    \end{center}
    \caption{Initial transcription model fitted to distribution of time spent
      in rounds of abortive cycling before promoter escape. Measurements are
      from Revyakin et.\ al \cite{revyakin_abortive_2006}. Fit to measurements
      shows best fit by an exponential function. The green area indicates time
      is less than one second.}
\label{fig:revyakin_fit}
\end{figure}


\begin{figure}
    \begin{center}
      \includegraphics[scale=0.8]{../figures/ap_adjustment.pdf}
    \end{center}
    \caption{Effect on model fit, estimated rate constants, and model
      performance when AP values are modified from baseline. A) Effect on
      model fit; B) effect on rate constant of NAC; C) effect on rate constant of
      the abortive step; D) effect on rate constant of escape; E) effect on
      $\tau_r$, which is the model-obtained $\tau$ relative to the value found from
      measurements from Vo et.\ al \cite{vo_vitro_2003-1} (Figure
      \ref{fig:vo_comparison}).}
\label{fig:ap_adjustment}
\end{figure}


\begin{figure}
    \begin{center}
        \includegraphics[scale=0.7]{../figures/vo_greb_minus_comparison.pdf}
    \end{center}
    \caption{Kinetics of FL transcript synthesis under -GreB conditions.
      Commparison between model (light green line) and measurements from Vo
      et.\ al \cite{vo_vitro_2003-1} (green dots). Blue line shows fit of
      measurements to a single exponential function.}
\label{fig:vo_comparison}
\end{figure}


\begin{figure}
    \begin{center}
      \includegraphics[scale=0.7]{../figures/Time_to_reach_fifty_percent_of_FL.pdf}
    \end{center}
    \caption{The value of $\tau$ obtained from simulation on -GreB N25 is
      similar to the experimentally obtained value, and is distinct from
      values of $\tau$ obtained by simulation on -GreB ITS variants of N25.}
\label{fig:its_variant_fl_comparison}
\end{figure}

% If you end up short with things in the results section, add something here
% about the sensitivity of the parameters. Define sensitivity as how much the
% value can be varied +/- before a 5% change in best fit is reached?

% Can we say that because the NAC is similar to elongation NAC, and that the
% ensemble experiment was so closely matched, that our two main assumptions
% are justified:
    % 1) that each backtracking event leads to an abortive release?
    % 2) that -GreBassisted cleavage and RNA release is rapid enough to be
    % negligeble for the oveall kinetics of the system?

%But how do we interpret the AP for the +GreB experiments? This is a bit of a
%conundrum. We are interpreting them as if GreB results in a reduction of
%the probability of backtracking at each position. That is OK, though, because
%then we only count the unscrunching steps that have proceeded without
%influence by GreB. We thus assume that rescue upon backtracking is fast
%enough.
