%\addbibresource{/home/jorgsk/Dropbox/phdproject/bibtex/jorgsk.bib}
\subsection{The forward pathway of transcription proceeds at the same speed
for initial transcription and for transcription elongation}

To identify the rate constants for initial transcription, we first did an
initial wide screening to identify which values are inconsistent with experimental
findings. In this screening, we considered the minimum and maximum values of
all rate constants to lie between 1 s$^{-1}$ and 25 s$^{-1}$. By evaluating
the model for these parameter ranges (see Materials and Methods), we found
that the rate constant for NAC is at least 7 s$^{-1}$, and that the rate
constant for unscrunching and abortive RNA release lies between 1 and 2.7
s$^{-1}$ (Figure~\ref{fig:parameter_estimation_1}). The screening showed that
the rate constant for promoter escape should be greater than 2.3 s$^{-1}$, but
also showed that there was no strong preference for a particular range of
rate constants, indicating that model performance was not sensitive to this
value (Figure~\ref{fig:parameter_estimation_1}).

Having identified the subset of rate constants that best fit experimental
data, we performed a more detailes screening of these values. Here, we varied
only the rate constants for the NAC and unscrunching and abortive RNA release,
since model performance was insensitive to the rate constant of promoter
escape. This showed \ldots.

To identify the best fitting rate constants, we calculated weighted averages
and weighted standard deviations surrounding the mode of the distributions.
The resulted in resulted in values of 10.5 s$^{-1} \pm 0.64$ for the NAC. This
closely matches the measured speed of transcription after promoter escape of
9.5 nt s$^{-1}$ (Revyakin et al. Figure S14 \cite{revyakin_abortive_2006}),
indicating that the speed of the NAC is unaffected by RNAP's attachment to the
promoter and DNA scrunching. The value for unscrunching and abortive RNA
release.


% XXX Use the weighted mean, or pick tuplets that have the sharpest peak for UAR?
% XXX: remember, when selecting parameter values, select those that give the
% highest peak in the fit, and color them in both graphs, showing not only the
% overall distribution but also the coupling between them. Alternatively show
% this on a heat map if that makes sense.

\begin{figure}
	\begin{center}
      \includegraphics[scale=0.8]{../figures/parameter_estimation_distributions_monte_carlo.pdf}
	\end{center}
    \caption{Distributions of estimated parameters and fits to experimental
      data for Monte Carlo sampled parameter values.  The first row shows the
      distributions for all samples, and the second and third rows show the
      distribution of the best 5\% and 1\% of simulations, respectively.}
      \label{fig:parameter_estimation_1}
\end{figure}

The comparatively
lower rate of 1.58 s$^{-1}$ for unscrunching and abortive RNA release is
consistent with previous findings that showed that this process is rate
limiting for initial transcription~\cite{margeat_direct_2006,
revyakin_abortive_2006}. Revyakin et.\ al did not measure cycles of abortive
cycling lasting shorter than 1 second, but proposed based on extrapolation of
their data that 20\% of abortive cycles last shorter than one second
\cite{revyakin_abortive_2006}.  The best fit to the data by our kinetic model
suggests that less than 2\% of abortive cycling rounds last shorter than 1
second (Figure \ref{fig:revyakin_fit}). To further investigate this
relationship, we fitted the model to the extrapolated distribution from
Revyakin et.\ al, forcing 20\% of abortive cycles to last less than 1 second.
This resulted in a rate constant for NAC of 18.4 s$^{-1} \pm 3.8$ was obtained
(Table~\ref{tab:param_fit_revyakin}), which is higher than what is reported
for transcription elongation for the same experiment
\cite{revyakin_abortive_2006}.

\begin{figure}
	\begin{center}
      \includegraphics[scale=0.8]{../illustrations/estimated_parameters.pdf}
	\end{center}
    \caption{Estimated rate constants for initial transcription on N25 in the
      presence of GreB.}
    \label{fig:estimated_parameters}
\end{figure}

\subsection{Estimated rate constants are also valid for -GreB kinetics}
We estimated the rate constants of NAC, escape, and unscrunching and abortive
RNA release using APs associated with +GreB. However, once fitted, the rate
constants should be generally valid, independent of GreB conditions and AP
values. To therefore validate the model, we compared the model with the
kinetics of full length (FL) transcript synthesis from an experiment by Vo
et.\ al on N25 initial transcription in -GreB conditions at 100 $\mu$M NTP
\cite{vo_vitro_2003-1}. For the experimental value, we fitted a single
exponential function to the measurements by Vo et.\ al to obtain a half life
of FL transcript syntehsis of 13.7 s. For the corresponding -GreB APs, the
model predicts a half-life of 15.5 s, closely matching the experimental value
(Figure~\ref{fig:vo_comparison}), which closely matches the model prediction.

% XXX Make a point out of that ITS variation leads to changes in PY, which can
% be seen by their higher APs. But what is the effect on kinetics? Use kinetic
% model to calculate distribution of half-lives for all ITS variants (include
% the experimental N25 value).
%While the experimental and model half lives are close, there is no other
%kinetic data for the N25 promoter in the literature which can tell us the
%range of values of $\tau$ for initial transcription on N25. To therefore be
%able to say something about the variability of $\tau$, calculated this value
%from simulations of initial transcription on N25-anti and N25 A1-anti, two ITS
%variants of N25 which have previously been shown to vary compared to N25 in
%several quantitative parameters for initial transcription~
%\cite{hsu_initial_2006,chan_anti-initial_2001,kammerer_functional_1986}. This
%resulted in values of $\tau$ of 61.2 s for N25-anti and 179.6 s for N25
%A1-anti, which agrees with these ITS variants having a lower productive yield
%than N25 \cite{hsu_initial_2006}.

\subsection{Estimated rate constants are sensitive to changes to the +GreB 
APs}
In estimating the rate constants, we have used APs from bulk studies by Hsu
et al. \cite{hsu_initial_2006} together with the abortive cycling distribution
obtained from single-molecule studies by Revyakin et al.
\cite{revyakin_abortive_2006}. This relies on the assumption that the
position-specific probability to abort initial transcription is the same for
the single-molecule experiment as as for the bulk studies. That this holds
true is not known, since the rapid kinetics of initial transcription
has prevented direct observation of the exact position of backtracking and abortive
RNA release \cite{margeat_direct_2006, revyakin_abortive_2006}. If there is a
close correspondence between the APs of both experimental techniques, one
would expect that model fitting would be sensitive to changes in AP. For
instance, if by either increasing or decreasing the bulk APs releative to
their baseline values leads to unphysiological rate constants, or
otherwise worsended model performance, this indicates that the bulk
AP values are descriptive for the single-molecule experiment.
To test this, we first repeated the model fitting proceedure using AP values
obtained without the presence of GreB. This led to an estimate for the rate
constant of NAC of 20.9 s$^{-1} \pm 2.9$.
We consider this value unphysiological since it would suggest that NAC
proceeds nearly twice as fast during scrunching compared to elongation,
proceeding at speeds normally found for NTP concentrations of 1000
$\mu$M~\cite{bai_mechanochemical_2007}, 10 times higher than actual
experimental conditions. Next, we performed a more general
perturbation of the AP values, making them both higher and lower across the
ITS (Figure~\ref{fig:aps_after_adjustment}). This showed first of all that the
fit is best for zero or small perturbations, and gets progressively worse as
perturbations increase in either direction (Figure \ref{fig:ap_adjustment}A).
This shows that the original bulk +GreB AP values lie in an optimal region in
terms of fitting the single-molecule experimental data. For NAC, higher APs
led to higher rate constants for NAC, while lower APs led to lower values.
(Figure \ref{fig:ap_adjustment}B). The rate constant for unscrunching and
abortive release increased as AP increases, to the point where this
step becomes nearly as rapid as the NAC, but for lower AP this value remained
simiular to baseline (Figure \ref{fig:ap_adjustment}C). The rate constant for
promoter escape was insensitive to changes in AP, indicating that model
performance is insensitive to this parameter (Figure
\ref{fig:ap_adjustment}D), consisitent with the finding from the original
parameter estimation process (Figure \ref{fig:parameter_estimation_1}).

The perturbation study showed that when AP values were increased,  the rate
constant for NAC increased to the point of being unphysiological at more than
20 $s^{-1}$, and the rate constant for the abortive step increased to the
point of being similar in size to that of the NAC, where it is hardly rate
limiting for the process. However, when AP values were decreased, there is
little evidency by which to judge the resulting rate constants as
unphysiological. Therefore, we wished to assess how the rate constants optimal
for the perturbed APs would affect model performance, especially for the case
of reduced APs. To do so, let the model calculate the half life of FL
synthesis for -GreB APs for the rate constants optimal for each AP
perturbation. This showed that when APs depart from their baseline value, the
optimal rate constants lead to lives of FL synthesis that become increasingly
separarated from the value obtained from measurements from Vo et al.
\cite{vo_vitro_2003-1}. This showes that while the rate constants optimal for
decreased AP may not be judged as unphysiological, they lead to worsensed
model performance compared to the basline baseline +GreB values.

\begin{table}
  \label{tab:param_fit_revyakin}
  \caption{Fitted rate constants of initial transcription}
  \begin{center}
    \begin{tabular}{lccc}
       \toprule
       NAC & Escape & Abort \\
       $10.1$ & $14.6$ & $1.6$ \\
       %-GreB & $20.9$ & $15.3$ & $17.5$ \\
       %Extrapolated fit & $18.4$ & $16.6$ & $1.8$ \\
    \end{tabular}
  \end{center}
\end{table}

\begin{figure}
    \begin{center}
      \includegraphics[scale=0.7]{../figures/cumul_scrunch_fit.pdf}
    \end{center}
    \caption{Initial transcription model fitted to distribution of time spent
      in rounds of abortive cycling before promoter escape. Measurements are
      from Revyakin et.\ al \cite{revyakin_abortive_2006}. Fit to measurements
      shows best fit by an exponential function. The green area indicates the
      region where abortive cycling lasts shorter than one second.}
\label{fig:revyakin_fit}
\end{figure}


\begin{figure}
    \begin{center}
      \includegraphics[scale=0.8]{../figures/ap_adjustment.pdf}
    \end{center}
    \caption{Effect on model fit, estimated rate constants, and model
      performance when +GreB AP values are modified from baseline. A) Effect on
      model fit; B) effect on rate constant of NAC; C) effect on rate constant of
      the abortive step; D) effect on rate constant of escape; E) effect on
      $\tau_r$, which is the model-obtained $\tau$ relative to the value found from
      measurements from Vo et.\ al \cite{vo_vitro_2003-1} (Figure
      \ref{fig:vo_comparison}).}
\label{fig:ap_adjustment}
\end{figure}


\begin{figure}
    \begin{center}
        \includegraphics[scale=0.7]{../figures/vo_greb_minus_comparison.pdf}
    \end{center}
    \caption{Kinetics of FL transcript synthesis under -GreB conditions.
      Commparison between model (light green line) and measurements from Vo
      et.\ al \cite{vo_vitro_2003-1} (green dots). Blue line shows fit of
      measurements to a single exponential function.}
\label{fig:vo_comparison}
\end{figure}


\begin{figure}
    \begin{center}
      \includegraphics[scale=0.7]{../figures/Time_to_reach_fifty_percent_of_FL.pdf}
    \end{center}
    \caption{The value of $\tau$ obtained from simulation on -GreB N25 is
      similar to the experimentally obtained value, and is distinct from
      values of $\tau$ obtained by simulation on -GreB ITS variants of N25.}
\label{fig:its_variant_fl_comparison}
\end{figure}

% If you end up short with things in the results section, add something here
% about the sensitivity of the parameters. Define sensitivity as how much the
% value can be varied +/- before a 5% change in best fit is reached?

% Can we say that because the NAC is similar to elongation NAC, and that the
% ensemble experiment was so closely matched, that our two main assumptions
% are justified:
    % 1) that each backtracking event leads to an abortive release?
    % 2) that -GreBassisted cleavage and RNA release is rapid enough to be
    % negligeble for the oveall kinetics of the system?

%But how do we interpret the AP for the +GreB experiments? This is a bit of a
%conundrum. We are interpreting them as if GreB results in a reduction of
%the probability of backtracking at each position. That is OK, though, because
%then we only count the unscrunching steps that have proceeded without
%influence by GreB. We thus assume that rescue upon backtracking is fast
%enough.
