%\addbibresource{/home/jorgsk/Dropbox/phdproject/bibtex/jorgsk.bib}
\SUBSECTION{Transcription proceeds at a similar average speed for initial
transcription as for transcription elongation}

We identified the rate constants for initial transcription using a two-step
method (see Methods). We found in the first step that the rate constant for
NAC is greater than 6/s; that the rate constant for unscrunching and abortive
RNA release lies between 1/s and 3/s; and that the rate constant for promoter
escape should be greater than 2.3/s
(\FIG~\ref{fig:parameter_estimation_proper}A). The distribution for the
promoter escape rate constant (right panel) did not exhibit any optimal peak,
which indicates that the time that RNAP spent in abortive cycling is not
sensitive to the rate constant for promoter escape. This is reasonable since
the reaction for promoter escape occurs downstream the sequence and does not
affect abortive cycling.   

In the second step, we varied the rate constants for NAC between 6/s and 14/s,
and UAR between 1/s and 3/s, based on the distributions of these
values from the first step. Because it did not affect abortive cycling,
we fixed rate constant of promoter escape to 20/s. The top 1\% best fitting
results from the second round of rate constant estimation resulted in clear
peaks in the distributions for both rate constants. The rate constant for the
NAC showed a peak at 10.6/s (\FIG~\ref{fig:parameter_estimation_proper}B),
consistent with the measured speed of elongating RNAP (9.2 nt/s on positively
supercoiled DNA and 13.3 nt/s on negatively supercoiled DNA
\cite{revyakin_abortive_2006}). The rate constant for UAR peaked at 1.4/s,
nearly 10 times slower than the NAC. This low value is consistent with
experimental evidence, suggesting that the unscrunching and abortive RNA
release processes are rate limiting in the initial
transcription~\cite{revyakin_abortive_2006, margeat_direct_2006}. Overall, the
rate constants we estimated resulted in an accurate predictive model where
most experimental measurements fell within one standard deviation of the mean
model prediction (\FIG~\ref{fig:revyakin_fit}). 

% \begin{figure}[h]
%     \begin{center}
%       \includegraphics{../illustrations/estimated_parameters_filled}
%     \end{center}
%     \caption{ {\bf Estimated rate constants for initial transcription on the N25
%       promoter.} The rate constants for the NAC and unscrunching and abortive
%       RNA release were obtained by parameter estimation (see text). The values
%       for backtracking were calculated from the value of the NAC and position
%       specific APs using equation \eqref{eq:backtrackingcalc}. Rate constants
%       for backtracking calculated for -GreB APs are given in parenthesis.}
%     \label{fig:estimated_parameters}
% \end{figure}

\SUBSECTION{Short duration scrunching does not follow an exponential
distribution as was previously reported}
  
Our model predicts that less than 4\% of scrunching events are shorter than
one second (Fig. 4). These quick events represent RNAP transcribing the 11
required base pairs without backtracking and reach promoter escape at a NAC
$K_n$ of 10.6/s (\FIG~\ref{fig:revyakin_fit}). This prediction is lower than
the 20\% abortive cycling predicted previously from extrapolation of data
(\cite{revyakin_abortive_2006}).  

To test the validity of our predictions, we first assumed that the previous
prediction, which was obtained by extrapolation of an exponential fitting
curve to experimental data, was accurate. We then repeated the parameter
estimation (see Methods) but now the RMSE was calculated by the predicted
$\hat T$ from the fitted curve, not the data measurements (see also
\FIG~\ref{fig:param_estimation_scheme} and Algorithm 1). Fitting to the
extrapolated data resulted in an overestimated rate constant NAC
($K_n= 23.4$/s, see \FIG~\ref{fig:extrap_and_GreB_minus_fit}). Not only that
this $K_n$ value is twice as high as that reported for transcription
elongation \cite{revyakin_abortive_2006}, but it is higher than the speed of
transcription elongation obtained when substrate NTP-concentrations were in
saturation (ten times as high) \cite{bai_mechanochemical_2007}. Furthermore,
the value of $K_u$ was highly overestimated (16.94, which is 11 times higher
than what was estimated from measurement data). 


\SUBSECTION{Estimated rate constants describe initial transcription also
in the absence of GreB}

We estimated model parameters using two independent data sets (APs and
scrunching duration, see Fig. \FIG~\ref{fig:extrap_and_GreB_minus_fit}), both
obtained in the presence of GreB. APs are increased when GreB is absent
\cite{hsu_initial_2006}, expressed in our model by higher backtracking values
$K_{b,i}$ (Eq. \ref{eq:backtrackingcalc}). The rate constants that we
estimated with data from +GreB, should also be valid for initial
transcription on the N25 promoter in the absence of GreB.  

To test the predictive ability of the model, we compared the predictions of
the model with $k_b$ calculated from -GreB APs \cite{hsu_initial_2006} to the
kinetics N25 full length (FL) transcript synthesis using obtained under $100\
\mu M$ NTP at 37 $^{\circ}$C in the absence of GreB \cite{vo_vitro_2003-1}. We
expected RNAP to exhibit more abortive cycles compared to when $k_b$ was
calcualted from +GreB APs, leading to a lower rate of full length product
synthesis. We stress that this kinetic data consisted of six measurements
only, and contained dynamics we did not model, in which unproductive complexes
synthesize abortive RNA continuously \cite{vo_vitro_2003-1}. However, these
limitations should not compromise the test of validity.
 
The comparison of the model predictions (blue solid in
\FIG~\ref{fig:vo_comparison}) to the measured -GreB data (squares) indicate
the estimated rate constants (obtained from +GreB data) are valid also for
transcription initiation in the absence of GreB. The rate constants of
backtracking associated with -GreB AP values are given in
\FIG~\ref{fig:estimated_parameters} (Supplementary Figures). The model
predicted the half-life of full length transcript synthesis in -GreB
conditions to be 17.4 s. In comparison, a half-life of 7.0 seconds was
obtained for full length transcript synthesis for +GreB conditions
(\FIG~\ref{fig:vo_comparison}). 
% 
% IS THE FOLLOWING NECESSARY? i THINK WE CAN DROP IT AND FIGURE 8, ELSE, WHAT
% IS THE MESSAGE OF THIS PARAGRAPH AND FIGURE 8? WHAT DO YOU WANT TO TELL AND
% THE CONTRIBUTION TO THE OVERALL STORY? THERE IS NO FURTHER REFERENCE TO THAT
% IN THE DISCUSSION. pLEASE DROP THIS PARAGRAPH.  Having thus tested the model
% on -GreB data, we calculated the half-life of FL transcript synthesis for
% all promoter variants in the DG100 series from Hsu et al./
% \cite{hsu_initial_2006}. The DG100 dataset contains 43 variants of the N25
% promoter with sequence variation in the region from +3 to +20 which causes a
% large variation in the productive yield between variants
% \cite{hsu_initial_2006}. The simulations resulted in a more than 10-fold
% variation in predicted half-life of FL synthesis, ranging from 14.7 seconds
% (DG146a) to 199 seconds (DG137a) (\FIG~\ref{fig:dg100_halflives}). Since
% productive yield is strongly associated with abortive probability, there is
% a near 1-to-1 correlation between the half-life of FL synthesis and the
% productive yield associated with each promoter variant (r=0.98, p<1e-30).
% 
\SUBSECTION{Model parameter estimation requires correct combination of
experimental data for fitting}

We originally estimated the rate constants using two datasets: APs
\cite{hsu_initial_2006} and duration of abortive cycling
\cite{revyakin_abortive_2006}, both obtained in the presence of +GreB (see
\FIG~\ref{fig:parameter_estimation_proper} and Methods). In other words, AP
and $\hat T_j$ data were consistent in terms of the presence (or absence) of
GreB. As discussed previously, GreB stimulates RNA cleavage in backtracked
complexes. If the model correctly represents the kinetics of initial
transcription, any re-estimation of rate constants using inconsistent data
(mix of datasets obtained in +GreB and -Greb conditions) should not lead to a
predictive model. To test our hypothesis, we re-estimated the
kinetic parameters of the model, now using -GreB APs \cite{hsu_initial_2006}
instead of the +GreB APs while we kept the same $\hat T_j$ data
from Revyakin et al. (right dataset in
\FIG~\ref{fig:param_estimation_scheme}). The resulted distributions were
significantly (Wilcoxon test, $p<10^{-10}$) different compared to the
consistent +GreB case (Figure 5 XXX NADI, FIG. 5 OMHANDLDER TIL EXTRAPOLATED
FIT, IKKE FIT TIL INCONSISTENT GREB), with the mean distributions of
unscrunching and abortive RNA release ($K_u$)
moved from 1.5/s (original blue dash vertical line) to 17.3/s (gray dash line)
and the NAC ($K_n$) moved from 10.6 /s to 23.3 /s. The new NAC value is
double the reported for transcription elongation after promoter escape
\cite{revyakin_abortive_2006}. Even if we assume these results are correct,
the value for unscrunching and abortive RNA release is similar to reported
transcription speeds, and therefore is not rate limiting for the initial
transcription process, which contradicts experimental findings
\cite{revyakin_abortive_2006, margeat_direct_2006}. We thus conclude that the
model is sensitive to inconsistent experimental data, and this strengthens its
validity.
