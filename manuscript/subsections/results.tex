%\addbibresource{/home/jorgsk/Dropbox/phdproject/bibtex/jorgsk.bib}
\subsection{The forward pathway of transcription proceeds at the same speed
for initial transcription and for transcription elongation}

To identify the rate constants for initial transcription, we first performed
an initial general screening to identify boundaries where rate constants
become inconsistent with experimental findings. In this screening, we
considered the minimum and maximum values of all rate constants to lie between
1 s$^{-1}$ and 25 s$^{-1}$. By evaluating the model for these parameter ranges
(see Materials and Methods), we found that the rate constant for NAC is at
least 7 s$^{-1}$, and that the rate constant for unscrunching and abortive RNA
release lies between 1 and 3 s$^{-1}$
(Figure~\ref{fig:parameter_estimation_proper}A). The screening showed that the
rate constant for promoter escape should be greater than 2.3 s$^{-1}$, but
also showed that there was no strong preference for a particular range of rate
constants. The lack of a clear selection for a specific range of values
indicates that initial transcription kinetics is not sensitive to the rate
constant for promoter escape (Figure~\ref{fig:parameter_estimation_proper}A).

Having identified boundaries for the rate constants, we performed a more
detailed screening. Since model performance was insensitive to the rate
constant of promoter escape (Figure~\ref{fig:parameter_estimation_proper}A),
we varied only the rate constants for the NAC and unscrunching and abortive
RNA release in this second round. Filtering for the top 1\% best fitting
results showed clear distributions for both rate constants. The
rate constant for the NAC showed a peak around 10 s$^{-1}$ (Figure
\ref{fig:parameter_estimation_proper}B), with the mode of the distribution at
10.4 s$^{-1}$ (Table \ref{tab:param_fit_revyakin}). This closely matches the
measured speed of transcription after promoter escape of 9.5 nt/s (Revyakin et
al. Figure S14 \cite{revyakin_abortive_2006}), indicating that the speed of
the NAC is unaffected by both RNAP's attachment to the promoter and DNA
scrunching. The rate constant for unscrunching and abortive release showed a
peak around 1.3 s$^{-1}$, nearly 10 times slower than the NAC (Figure
\ref{fig:parameter_estimation_proper}B, Table \ref{tab:param_fit_revyakin}).
This low value is consistent with experimental evidence indicating that
unscrunching and abortive RNA release are the rate limiting steps for initial
transcription~\cite{margeat_direct_2006, revyakin_abortive_2006}. A full
kinetic scheme with these rate constants is shown in Figure
\ref{fig:estimated_parameters}A.

\begin{figure}
	\begin{center}
      \includegraphics[scale=0.8]{../figures/coarse_and_finegrain_search_GreB_no_extrap.pdf}
	\end{center}
    \caption{Distributions of rate constants after fitting 100000 randomly
      sampled values and selecting the 1000 (1\%) best fits. \textbf{A}: First round
      of screening, where all three rate constants were varied between 1 and
      25 s$^{-1}$. \textbf{B}: Second round of sceening, where NAC was varied
      between 6 and 16 s$^{-1}$ and unscrunching and abortive RNA release
      (UAR) was varied between 1 and 3 s$^{-1}$. The promoter escape rate
      constant was held constant at 20 s$^{-1}$ for the second round.}
      \label{fig:parameter_estimation_proper}
\end{figure}

\begin{figure}
	\begin{center}
      \includegraphics[scale=0.8]{../figures/coarse_search-fits_2_3.pdf}
	\end{center}
    \caption{
      Rate constant distributions obtained by fitting the model to
      extrapolated and inconsistent experimental data. \textbf{A}:
      Distributions of top 1\% rate constants after fitting model to
      extrapolated curve for the distribution of time spent in abortive
      cycling \cite{revyakin_abortive_2006}. \textbf{B}: Distributions of top
      1\% rate constants after fitting model using AP values obtained in the
      absence of GreB. In silhuette is shown the distribution obtained with
      matching experimental input from Figure \ref{fig:parameter_estimation_proper}}

      \label{fig:extrap_and_GreB_minus_fit}
\end{figure}

Figure \ref{fig:revyakin_fit} shows a comparison between of the distribution
of time spent in abortive cycling between the measured data from Revyakin et
al. \cite{revyakin_abortive_2006} and the model, where the rate given in Table
\ref{tab:param_fit_revyakin} have been used. As can be seen, there is a close
match between model and measured values. Also shown in Figure
\ref{fig:revyakin_fit} is an extrapolation of the measurement data. Revyakin
et.\ al did not measure cycles of abortive cycling lasting shorter than ~3.5
seconds, but proposed based on this extrapolation curve that 20\% of abortive
cycles are shorter than one second \cite{revyakin_abortive_2006}. In contrast,
the best fit by our kinetic model to the measurements suggests that less than
2\% of abortive cycling rounds last shorter than one second (Figure
\ref{fig:revyakin_fit}). If the extrapolation of the data reflects the true
scrunch-time distribution, the model should perform better when fitted to the
extrapolated data. To test this, we repeated the rate constant estimation
proceedure, but fitted to the extrapolated curve instead of the measurements.
When doing this, the estimated rate constant for unscrunching and
abortive RNA release were similar to when fitting to the actual measured data,
but the NAC rate constant was estimated to be much higher, with most values
being above 10 s$^{-1}$, and with the distribution peak 23.4 s$^{-1}$ (Figure
\ref{fig:extrap_and_GreB_minus_fit}). Thus, fitting to the extrapolated data
necessitates a NAC during initial transcription at speeds far above what has
been observed for transcription elongation \cite{revyakin_abortive_2006}. 

\begin{figure}
	\begin{center}
      \includegraphics[scale=0.8]{../illustrations/estimated_parameters_filled.pdf}
	\end{center}
    \caption{Estimated rate constants for initial transcription on N25. The
      values for the rate constant of the NAC, uncrunching and abortive RNA
      release, and promoter escape are as given in Table
      \ref{tab:param_fit_revyakin}. The values for backtracking are calculated
      from the value of the NAC using equation \eqref{eq:backtrackingcalc}.
      \textbf{A}: Rate constants of backtracking calculated for AP values
      obtained in the presence of GreB. \textbf{B}: Rate constants of
      backtracking calculated for AP values obtained in the absence of GreB. }
    \label{fig:estimated_parameters}
\end{figure}

\subsection{Estimated rate constants are valid for initial transcription also
in the absence of GreB}
Since the experiments from Revyakin et al. were performed in the presence of
GreB, we had used APs calculated from bulk experiments performed in the presence of
GreB for calculating the rate constant of backtracking. However, once fitted,
the rate constants reported in Table \ref{tab:param_fit_revyakin} should be
generally valid for initial transcription on the N25 promoter, irrespective of
presence or absence of GreB. To investigate the general validity of the
estimated rate constants, we calculated rate constants of backtracking from AP
values obtained in the absence of GreB and compared the result with
experiments of the kinetics of N25 full length (FL) transcript synthesis in
the absence of GreB \cite{vo_vitro_2003-1}. In the absence of GreB, APs
increase \cite{hsu_initial_2006}, which in our model transfers into higher
rates of backtracking compared to the +GreB case (Figure
S\ref{fig:parameter_estimation_scheme}). To be able to compare with the time
series from Vo et al., we fitted their data to a single exponential function
to obtain a half life of FL transcript synthesis of 13.7 s. We then ran the
model on -GreB AP values using the rate constants in Table
\ref{tab:param_fit_revyakin}. This resulted in a model-predicted half-life of
FL transcript synthesis of 15.5 s (Figure~\ref{fig:vo_comparison}), closely
matching the experimental value. This shows that the rate constants in Table
\ref{tab:param_fit_revyakin} are valid for the kinetics of initial
transription both in the presence and in the absence of GreB. The kinetic
scheme of the estimated rate constants for initial transcription on the N25
promoter in the absence of GreB is given in Figure
\ref{fig:estimated_parameters}B.

\subsection{Parameter estimation is highly sensitive to the consistency of experimental data}
When fitting the model we combined experimental data for time spent in
abortive cycling and APs, where both sources of data were consistent in that
they were obtained from initial transcription experiments performed in the
presence of GreB. If the model correctly represents the kinetics of initial
transcription, parameter estimation should be sensitive to the consistency of
the experimental data. To test this, we re-fitted the model, now using APs
that were obtained in the absence of GreB when calculating backtracking rates.
This introduces a mismatch between the experimental data to which the model is
fitted (+GreB) and the experimental data used to calculate the rate constant
of backtracking (-GreB). This re-fitting resulted in markedly different
distributions of the rate constants com \ref{fig:extrap_and_GreB_minus_fit},
with the mode of the distributions being 23.6 and 14.7 for the NAC and
unscrunching and abortive RNA release, respectively, which we consider to be
outside a physiologically believable range for thise rate constants. This
shows that model correctly reacts to inconsistent experimental data in
parameter estimation.

\begin{table}
  \label{tab:param_fit_revyakin}
  \caption{Fitted rate constants of initial transcription. For NAC and
  unscrunching and abortive RNA release, these values are the modes of the
  distributions in Figure \ref{fig:parameter_estimation_proper}B. The rate constant of
  promoter escape was selected from Figure
\ref{fig:fig:parameter_estimation_proper}B.}.
  \begin{center}
    \begin{tabular}{ccc}
       \toprule
       NAC & Unscrunching and abortive RNA release & Promoter escape \\
       $10.4$ & $1.3$ & $20$ \\
    \end{tabular}
  \end{center}
\end{table}


\begin{figure}
    \begin{center}
      \includegraphics[scale=0.7]{../figures/cumul_scrunch_fit.pdf}
    \end{center}
    \caption{Initial transcription model fitted to distribution of time spent
      in rounds of abortive cycling before promoter escape. Measurements are
      from Revyakin et.\ al \cite{revyakin_abortive_2006}. Fit to measurements
      shows best fit by an exponential function. The green area indicates the
      region where abortive cycling lasts shorter than one second.}
\label{fig:revyakin_fit}
\end{figure}


\begin{figure}
    \begin{center}
        \includegraphics[scale=0.7]{../figures/vo_greb_minus_comparison.pdf}
    \end{center}
    \caption{Kinetics of FL transcript synthesis under -GreB conditions.
      Commparison between model (light green line) and measurements from Vo
      et.\ al \cite{vo_vitro_2003-1} (green dots). Blue line shows fit of
      measurements to a single exponential function.}
\label{fig:vo_comparison}
\end{figure}

