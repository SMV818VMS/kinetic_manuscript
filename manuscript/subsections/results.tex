%\addbibresource{/home/jorgsk/Dropbox/phdproject/bibtex/jorgsk.bib}
\subsection{The forward pathway of transcription proceeds at the same speed
for initial transcription and for transcription elongation}

To identify the rate constants for initial transcription, we first did an
initial wide screening to identify which values are inconsistent with
experimental findings. In this screening, we considered the minimum and
maximum values of all rate constants to lie between 1 s$^{-1}$ and 25
s$^{-1}$. By evaluating the model for these parameter ranges (see Materials
and Methods), we found that the rate constant for NAC is at least 7 s$^{-1}$,
and that the rate constant for unscrunching and abortive RNA release lies
between 1 and 3 s$^{-1}$ (Figure~\ref{fig:parameter_estimation_proper}A). The
screening showed that the rate constant for promoter escape should be greater
than 2.3 s$^{-1}$, but also showed that there was no strong preference for a
particular range of rate constants. This indicates that the kinetics of the
system is not sensitive to the rate constant for promoter escape
(Figure~\ref{fig:parameter_estimation_proper}A).

Having identified a reduced range for the rate constants that best fit
experimental data, we performed a more detailed screening. Since model
performance was insensitive to the rate constant of promoter escape
(Figure~\ref{fig:parameter_estimation_proper}A), we varied only the rate
constants for the NAC and unscrunching and abortive RNA release. After
filtering out the top 1\% of all simulations, the distributions of both rate
constants showed clear peaks. The rate constant for the NAC showed a peak
around 10 s$^{-1}$ (Figure \ref{fig:parameter_estimation_proper}B), with the
mode of the distribution at 9 s$^{-1}$ (Table \ref{tab:param_fit_revyakin}).
This closely matches the measured speed of transcription after promoter escape
of 9.5 nt/s (Revyakin et al. Figure S14 \cite{revyakin_abortive_2006}),
indicating that the speed of the NAC is unaffected by both RNAP's attachment
to the promoter and DNA scrunching. The rate constant for unscrunching and
abortive release showed a peak around 1.2 s$^{-1}$, nearly 10 times slower
than the NAC (Figure \ref{fig:parameter_estimation_proper}B, Table
\ref{tab:param_fit_revyakin}). This low value is consistent with experimental
evidence indicating that unscrunching and abortive RNA release are the rate
limiting steps for initial transcription~\cite{margeat_direct_2006,
revyakin_abortive_2006}. A full scheme of the estimated rate constants is
given in Figure \ref{fig:estimated_parameters}A.

\begin{figure}
	\begin{center}
      \includegraphics[scale=0.8]{../figures/coarse_and_finegrain_search_GreB_no_extrap.pdf}
	\end{center}
    \caption{Distributions of rate constants after fitting 100000 randomly
      sampled values and selecting the 1000 best fits. \textbf{A}: First round
      of screening, where all three rate constants were varied between 1 and
      25 s$^{-1}$. \textbf{B}: Second round of sceening, where NAC was varied
      between 6 and 16 s$^{-1}$ and unscrunching and abortive RNA release
      (UAR) was varied between 1 and 3 s$^{-1}$. The rate constant for
      promoter escape was held constant at 20 s$^{-1}$ for the second round.}
      \label{fig:parameter_estimation_proper}
\end{figure}

\begin{figure}
	\begin{center}
      \includegraphics[scale=0.8]{../figures/coarse_search-fits.pdf}
	\end{center}
    \caption{Parameter fitting to extrapolated data and to inconsistent
      experimental data.}
      \label{fig:extrap_and_GreB_minus_fit}
\end{figure}

Figure \ref{fig:revyakin_fit} shows a comparison between of the distribution
of time spent in abortive cycling for the model and the measured data from
Revyakin et al. \cite{revyakin_abortive_2006}, where the simulation has been
performed using the rate constants given in Table
\ref{tab:param_fit_revyakin}. As can be seen, there is a close match between
model and measured values. Also shown in Figure \ref{fig:revyakin_fit} is an
extrapolation of the measurement data. Revyakin et.\ al did not measure cycles
of abortive cycling lasting shorter than ~3.5 seconds, but proposed based on
the extrapolation curve that 20\% of abortive cycles are shorter than one
second \cite{revyakin_abortive_2006}. In contrast, the best fit by our kinetic
model to the measurements suggests that less than 2\% of abortive cycling
rounds last shorter than one second (Figure \ref{fig:revyakin_fit}). If the
extrapolation of the data reflects the true scrunch-time distribution, the
model should perform well also when fitted to the extrapolated data. To test
this, we repeated the model fitting proceedure using extrapolated curve
instead of the measurements. In contrast to when fitting to the actual
measured data, there was no clear screening of the parameter space, but a
general preference for large rate constants (Figure
\ref{fig:extrap_and_GreB_minus_fit}). The distribution modes were 21.3 and 1.2
for the rate constant of the NAC and unscrunching and abortive RNA release,
respectively.
% XXX have to rewrite this.

\begin{figure}
	\begin{center}
      \includegraphics[scale=0.8]{../illustrations/estimated_parameters_filled.pdf}
	\end{center}
    \caption{Estimated rate constants for initial transcription on N25. The
      values for the rate constant of the NAC, uncrunching and abortive RNA
      release, and promoter escape are as given in Table
      \ref{tab:param_fit_revyakin}. The values for backtracking are calculated
      from the value of the NAC using equation \eqref{eq:backtrackingcalc}.
      \textbf{A}: Rate constants of backtracking calculated for AP values
      obtained in the presence of GreB. \textbf{B}: Rate constants of
      backtracking calculated for AP values obtained in the absence of GreB. }
    \label{fig:estimated_parameters}
\end{figure}

\subsection{Estimated rate constants are valid for initial transcription also
in the absence of GreB}
Since the experiments from Revyakin et al. were performed in the presence of
GreB, we had used APs calculated from bulk experiments performed in the presence of
GreB to estimate the rate constants in Table \ref{tab:param_fit_revyakin}.
However, once fitted, the rate constants should be generally valid for initial
transcription on the N25 promoter, irrespective of presence or absence of
GreB. To investigate the general validity of the estimated rate constants, we
compared the model with experiments performed by Vo et al.  who obtained a
kinetics of N25 full length (FL) transcript synthesis in the absence of GreB
\cite{vo_vitro_2003-1}. The effect of the absence of GreB is a marked increase
in AP values \cite{hsu_initial_2006}, which in our model will be reflected in
a higher rates of backtracking compared to the +GreB case (Figure
S\ref{fig:parameter_estimation_scheme}). To be able to compare with the time
series from Vo et al., we fitted their data to a single exponential function
to obtain a half life of FL transcript synthesis of 13.7 s. We then ran the
model on -GreB AP values using the rate constants in Table
\ref{tab:param_fit_revyakin}, using the same rate constant for NAC before and
after promoter escape. This resulted in a model-predicted half-life of FL
transcript synthesis of 15.5 s (Figure~\ref{fig:vo_comparison}), closely
matching the experimental value. The kinetic scheme of the estimated rate
constants for initial transcription on the N25 promoter in the absence of GreB
is given in Figure \ref{fig:estimated_parameters}B.

\subsection{Parameter estimation is highly sensitive to the consistency of experimental data}
When fitting the model we combined experimental data for time spent
in abortive cycling and APs, where both sources of data
came from initial transcription experiments performed in the presence of GreB.
If the model correctly represents the kinetics of the system, parameter
estimation should be sensitive to the consistency of the experimental data. To
test this, we re-fitted the model, now using APs that were obtained in
the absence of GreB to calculate backtracking rates. This introduces a
mismatch between the experimental data to which the model is fitted (+GreB)
and the experimental data used to calculate the rate constant of backtracking
(-GreB). This re-fitting resulted in markedly different distributions of the 
rate constants com \ref{fig:extrap_and_GreB_minus_fit}, with the mode of the
distributions being 23.6 and 14.7 for the NAC and unscrunching and abortive
RNA release, respectively, which we consider to be outside a physiologically
believable range for thise rate constants.

\begin{table}
  \label{tab:param_fit_revyakin}
  \caption{Fitted rate constants of initial transcription. For NAC and
  unscrunching and abortive RNA release, these values are the modes of the
  distributions in Figure \ref{fig:parameter_estimation_proper}B. The rate constant of
  promoter escape was selected from Figure
\ref{fig:fig:parameter_estimation_proper}B.}.
  \begin{center}
    \begin{tabular}{ccc}
       \toprule
       NAC & Unscrunching and abortive RNA release & Promoter escape \\
       $10.4$ & $1.3$ & $20$ \\
    \end{tabular}
  \end{center}
\end{table}


\begin{figure}
    \begin{center}
      \includegraphics[scale=0.7]{../figures/cumul_scrunch_fit.pdf}
    \end{center}
    \caption{Initial transcription model fitted to distribution of time spent
      in rounds of abortive cycling before promoter escape. Measurements are
      from Revyakin et.\ al \cite{revyakin_abortive_2006}. Fit to measurements
      shows best fit by an exponential function. The green area indicates the
      region where abortive cycling lasts shorter than one second.}
\label{fig:revyakin_fit}
\end{figure}


\begin{figure}
    \begin{center}
        \includegraphics[scale=0.7]{../figures/vo_greb_minus_comparison.pdf}
    \end{center}
    \caption{Kinetics of FL transcript synthesis under -GreB conditions.
      Commparison between model (light green line) and measurements from Vo
      et.\ al \cite{vo_vitro_2003-1} (green dots). Blue line shows fit of
      measurements to a single exponential function.}
\label{fig:vo_comparison}
\end{figure}

