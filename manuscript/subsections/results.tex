%\addbibresource{/home/jorgsk/Dropbox/phdproject/bibtex/jorgsk.bib}
\subsection(Reaction rates during initial transcription)
Several single-molecule studies of initial transcription have been undertaken
\cite{margeat_direct_2006,revyakin_abortive_2006}. These serve as the best
source for determining the rates of initial transcription. Revyakin et. al
measured the time spent in abortive cycling before promoter escape for 100
individual transcription events \cite{revyakin_abortive_2006}. These
experiments were performed on the N25 promoter with 100 $\my$M NTP in the
precense of GreB. We used the results of Revyakin et. al to calculate the
cumulative probability of achieving promoter escape and fitted the rate
constants of NAC, promoter escape, and unscrunching (Figure
\ref{fig:revyakin_fit}). For this, we used abortive probabilities for N25 with
GreB as found by Hsu et al. \cite{hsu_initial_2006}. We found the best fit to
the experimental data for a NAC at 10.5 nt/s, 8/s rate constant for promoter
escape, and 2/s rate constant for unscrunching and abortive RNA release.

\begin{figure}
	\begin{center}
        \includegraphics{../figures/cumul_scrunch_fit.pdf}
	\end{center}
    \caption{Fit of model to kinetic data.}
    \label{fig:revyakin_fit}
\end{figure}

We compared with single-molecule experiments. What they did. What were the
conditions. We used GreB+ APs from Hsu 2006. How we fitted the model. These
were the results: nac, abort, escape (Fig1). This fits with their results.
Strong indication that during abortive transcription, the nucleotide addtion
cycle proceeds at the same speed as for transcription elongation.  promoter
escape transcription is likely to proceed with the same speed.  Noticed that
only 5 pct were less than 1 second with optimal fit. We ran 10000 simulations
of single RNAP and look at the average (Fig2). Shows that with this fit, up to
xx pct may fit within th 1 sec window.

%[Recall that you are estimating reaction rate constants: 10/s, not the rate.
%The Rate will be d[RNAP_5]/dt = k*RNAP_4 = 10/s * RNAP_4, for example.]

Comparison with ensemble studies. Vo did simulation with 100 [NTP], as for
Revyakin, but without GreB. So we tried without GreB. Worked great (see here).
Showing that the rates obtained by single-molecule experiments works also for
ensenble transcription studies. This shows that the AP profile, obtained
without necessitating timeseries information, is sufficient to find the
kinetics of initial transcription.

The results indicate that the AP profile is sufficient. So far, the AP profile
has been varied by GreB+/-. But AP profile will also change with mutation in
the initial trascripbed sequence. To see the effect of ITS mutations on the
kinetics of initial transctiption, we examined the kinetics 43 N25 promoter
variants from Hsu et. al. This shows that mutations to the ITS leads to large
variability in the time needed to reach promoter escape, indicated by $\tau$
which is the time when 50 pct of FL product has been produced for N25,
N25$_{\text{anti}}$, and N25A1$_{\text{anti}}$ (Fig3A). Overall, $\tau$ varied
from 12s (DG001) to 3min xxx s (DG002) (Figure 3B). Using the APs to calculate
the propensity of backtracking allows calculation of the frequency of
backtracking during initial transcription. For transcription elongation, the
rate of backtracking is at X/bp. For the N25 variants, it varied from x/bp to
y/bp (Figure S1).
