%\addbibresource{/home/jorgsk/Dropbox/phdproject/bibtex/jorgsk.bib}
\subsection{The forward pathway of transcription proceeds at the same speed
for initial transcription and for transcription elongation}

To identify the rate constants for initial transcription, we first did an
initial wide screening to identify which values are inconsistent with experimental
findings. In this screening, we considered the minimum and maximum values of
all rate constants to lie between 1 s$^{-1}$ and 25 s$^{-1}$. By evaluating
the model for these parameter ranges (see Materials and Methods), we found
that the rate constant for NAC is at least 7 s$^{-1}$, and that the rate
constant for unscrunching and abortive RNA release lies between 1 and 2.7
s$^{-1}$ (Figure~\ref{fig:parameter_estimation_1}). The screening showed that
the rate constant for promoter escape should be greater than 2.3 s$^{-1}$, but
also showed that there was no strong preference for a particular range of
rate constants, indicating that model performance was not sensitive to this
value (Figure~\ref{fig:parameter_estimation_1}).

Having identified the subset of rate constants that best fit experimental
data, we performed a more detailes screening of these values. In this round,
we varied only the rate constants for the NAC and unscrunching and abortive
RNA release, since model performance was insensitive to the rate constant of
promoter escape (Figure~\ref{fig:parameter_estimation_1}). After filtering out
the top 1\% of all simulations, the distributions of both rate constants
showed clear peaks. The rate constant for the NAC showed a peak around 10
s$^{-1}$ (Figure \ref{fig:high_res_search}), with the mode of the distribution
at 9 s$^{-1}$ (Table \ref{tab:param_fit_revyakin}). This closely matches the measured
speed of transcription after promoter escape of 9.5 nt/s (Revyakin et
al. Figure S14 \cite{revyakin_abortive_2006}), indicating that the speed of
the NAC is unaffected by both RNAP's attachment to the promoter and DNA
scrunching. The rate constant for unscrunching and abortive release showed a
peak around 1.2 s$^{-1}$, nearly 10 times slower than the NAC (Figure
\ref{fig:high_res_search}B, Table \ref{tab:param_fit_revyakin}). This low
value is consistent with experimental evidence indicating that unscrunching
and abortive RNA release are the rate limiting steps for initial
transcription~\cite{margeat_direct_2006, revyakin_abortive_2006}. A full
kinetic scheme of the estimated rate constants is given in Figure
\ref{fig:estimated_parameters}A.

\begin{figure}
	\begin{center}
      \includegraphics[scale=0.8]{../figures/parameter_estimation_distributions_monte_carlo.pdf}
	\end{center}
    \caption{Distributions of estimated parameters and fits to experimental
      data for Monte Carlo sampled parameter values. The first row shows the
      distributions for all samples, and the second and third rows show the
      distribution of the best 5\% and 1\% of simulations, respectively.}
      \label{fig:parameter_estimation_1}
\end{figure}

\begin{figure}
	\begin{center}
      \includegraphics[scale=0.8]{../figures/high_res_search_for_rate_constants.pdf}
	\end{center}
    \caption{Distributions of estimated parameters. \textbf{A}: Distribution
      of the rate constant for the NAC from the top 1\% of simulations in terms
      of fit to experimental data. \textbf{B}: Distribution of the rate constant
      for unscrunching and abortive RNA release from the top 1\% of
      simulations in terms of fit to experimental data.}
      \label{fig:high_res_search}
\end{figure}

Figure \ref{fig:revyakin_fit} shows a comparison between of the distribution
of time spent in abortive cycling for the model and the measured data from
Revyakin et al. \cite{revyakin_abortive_2006}, where the simulation has been
performed using the rate constants given in Table
\ref{tab:param_fit_revyakin}. As can be seen, there is a close match between
model and measured values. Also shown in Figure \ref{fig:revyakin_fit} is an
extrapolation of the measurement data. Revyakin et.\ al did not measure cycles
of abortive cycling lasting shorter than ~3.5 seconds, but proposed based on
the extrapolation curve that 20\% of abortive cycles are shorter than one
second \cite{revyakin_abortive_2006}. In contrast, the best fit by our kinetic
model to the measurements suggests that less than 2\% of abortive cycling
rounds last shorter than one second (Figure \ref{fig:revyakin_fit}). If the
extrapolation of the data reflects the true scrunch-time distribution, the
model should perform well also when fitted to the extrapolated data. To test
this, we repeated the model fitting proceedure using extrapolated curve
instead of the measurements. In contrast to when fitting to measurment data,
there was no clear screening of the parameter space, but a general preference
for large rate constants (Figure S\ref{fig:coarse_extrap_fit}). The
distribution modes were 4.3 and 20.4 for the rate constant of the NAC and
unscrunching and abortive RNA release, respectively.

\begin{figure}
	\begin{center}
      \includegraphics[scale=0.8]{../illustrations/estimated_parameters_filled.pdf}
	\end{center}
    \caption{Estimated rate constants for initial transcription on N25. The
      values for the rate constant of the NAC, uncrunching and abortive RNA
      release, and promoter escape are as given in Table
      \ref{tab:param_fit_revyakin}. The values for backtracking are calculated
      from the value of the NAC using equation \eqref{eq:backtrackingcalc}.
      \textbf{A}: Rate constants of backtracking calculated for AP values
      obtained in the presence of GreB. \textbf{B}: Rate constants of
      backtracking calculated for AP values obtained in the absence of GreB. }
    \label{fig:estimated_parameters}
\end{figure}

\subsection{Estimated rate constants are valid for initial transcription also
in the absence of GreB}
Since the experiments from Revyakin et al. were performed in the presence of
GreB, we had used APs calculated from bulk experiments performed in the presence of
GreB to estimate the rate constants in Table \ref{tab:param_fit_revyakin}.
However, once fitted, the rate constants should be generally valid for initial
transcription on the N25 promoter, irrespective of presence or absence of
GreB. To investigate the general validity of the estimated rate constants, we
compared the model with experiments performed by Vo et al.  who obtained a
kinetics of N25 full length (FL) transcript synthesis in the absence of GreB
\cite{vo_vitro_2003-1}. The effect of the absence of GreB is a marked increase
in AP values \cite{hsu_initial_2006}, which in our model will be reflected in
a higher rates of backtracking compared to the +GreB case (Figure
S\ref{fig:parameter_estimation_scheme}). To be able to compare with the time
series from Vo et al., we fitted their data to a single exponential function
to obtain a half life of FL transcript synthesis of 13.7 s. We then ran the
model on -GreB AP values using the rate constants in Table
\ref{tab:param_fit_revyakin}, using the same rate constant for NAC before and
after promoter escape. This resulted in a model-predicted half-life of FL
transcript synthesis of 15.5 s (Figure~\ref{fig:vo_comparison}), closely
matching the experimental value. The kinetic scheme of the estimated rate
constants for initial transcription on the N25 promoter in the absence of GreB
is given in Figure \ref{fig:estimated_parameters}B.

\subsection{The initial transcription model is highly sensitive to
inconsistent experimental data}
As a final test of the model, we wished to evaluate the effect of
inconsistent input data. We had used experimental data obtained in the
presence of GreB to show that the model fits experimental data and predicts
rate constants for the NAC and unscrunching and abortive RNA release that are
consistent with physiological constaints and previous literature (Figure
\ref{revyakin_fit} and Table \ref{tab:param_fit_revyakin}). If this model is a
good representation of the real kinetics of the system it should be sensitive
to inconsistent input data. To test this, we re-fitted the model using the
scrunch-time distribution from Revyakinn et al. combined with AP values
obtained in the absence of GreB. This resulted in a selection for very large
rate constants \ref{fig:inconsistent_estimationn}, with the mode of the
distributions being 20.3 and 8.3 for the NAC and unscrunching and abortive RNA
release, respectively.
  
% XXX It's all true and right, but we're leaving this one out for now
%While the experimental and model half lives are close, there is no other
%kinetic data for the N25 promoter in the literature which can tell us the
%range of values of $\tau$ for initial transcription on N25. To therefore be
%able to say something about the variability of $\tau$, calculated this value
%from simulations of initial transcription on N25-anti and N25 A1-anti, two ITS
%variants of N25 which have previously been shown to vary compared to N25 in
%several quantitative parameters for initial transcription~
%\cite{hsu_initial_2006,chan_anti-initial_2001,kammerer_functional_1986}. This
%resulted in values of $\tau$ of 61.2 s for N25-anti and 179.6 s for N25
%A1-anti, which agrees with these ITS variants having a lower productive yield
%than N25 \cite{hsu_initial_2006}.

% XXX leaving this one out for now
%\subsection{Estimated rate constants are sensitive to changes to the +GreB 
%APs}
%In estimating the rate constants, we have used APs from bulk studies by Hsu
%et al. \cite{hsu_initial_2006} together with the abortive cycling distribution
%obtained from single-molecule studies by Revyakin et al.
%\cite{revyakin_abortive_2006}. This relies on the assumption that the
%position-specific probability to abort initial transcription is the same for
%the single-molecule experiment as as for the bulk studies. That this holds
%true is not known, since the rapid kinetics of initial transcription
%has prevented direct observation of the exact position of backtracking and abortive
%RNA release \cite{margeat_direct_2006, revyakin_abortive_2006}. If there is a
%close correspondence between the APs of both experimental techniques, one
%would expect that model fitting would be sensitive to changes in AP. For
%instance, if by either increasing or decreasing the bulk APs releative to
%their baseline values leads to unphysiological rate constants, or
%otherwise worsended model performance, this indicates that the bulk
%AP values are descriptive for the single-molecule experiment.
%To test this, we first repeated the model fitting proceedure using AP values
%obtained without the presence of GreB. This led to an estimate for the rate
%constant of NAC of 20.9 s$^{-1} \pm 2.9$.
%We consider this value unphysiological since it would suggest that NAC
%proceeds nearly twice as fast during scrunching compared to elongation,
%proceeding at speeds normally found for NTP concentrations of 1000
%$\mu$M~\cite{bai_mechanochemical_2007}, 10 times higher than actual
%experimental conditions. Next, we performed a more general
%perturbation of the AP values, making them both higher and lower across the
%ITS (Figure~\ref{fig:aps_after_adjustment}). This showed first of all that the
%fit is best for zero or small perturbations, and gets progressively worse as
%perturbations increase in either direction (Figure \ref{fig:ap_adjustment}A).
%This shows that the original bulk +GreB AP values lie in an optimal region in
%terms of fitting the single-molecule experimental data. For NAC, higher APs
%led to higher rate constants for NAC, while lower APs led to lower values.
%(Figure \ref{fig:ap_adjustment}B). The rate constant for unscrunching and
%abortive release increased as AP increases, to the point where this
%step becomes nearly as rapid as the NAC, but for lower AP this value remained
%simiular to baseline (Figure \ref{fig:ap_adjustment}C). The rate constant for
%promoter escape was insensitive to changes in AP, indicating that model
%performance is insensitive to this parameter (Figure
%\ref{fig:ap_adjustment}D), consisitent with the finding from the original
%parameter estimation process (Figure \ref{fig:parameter_estimation_1}).

%The perturbation study showed that when AP values were increased,  the rate
%constant for NAC increased to the point of being unphysiological at more than
%20 $s^{-1}$, and the rate constant for the abortive step increased to the
%point of being similar in size to that of the NAC, where it is hardly rate
%limiting for the process. However, when AP values were decreased, there is
%little evidency by which to judge the resulting rate constants as
%unphysiological. Therefore, we wished to assess how the rate constants optimal
%for the perturbed APs would affect model performance, especially for the case
%of reduced APs. To do so, let the model calculate the half life of FL
%synthesis for -GreB APs for the rate constants optimal for each AP
%perturbation. This showed that when APs depart from their baseline value, the
%optimal rate constants lead to lives of FL synthesis that become increasingly
%separarated from the value obtained from measurements from Vo et al.
%\cite{vo_vitro_2003-1}. This showes that while the rate constants optimal for
%decreased AP may not be judged as unphysiological, they lead to worsensed
%model performance compared to the basline baseline +GreB values.

\begin{table}
  \label{tab:param_fit_revyakin}
  \caption{Fitted rate constants of initial transcription. For NAC and
  unscrunching and abortive RNA release, these values are the modes of the
  distributions in Figure \ref{fig:high_res_search}. The rate constant of
  promoter escape was selected from Figure \ref{fig:fig:parameter_estimation_1}}.
  \begin{center}
    \begin{tabular}{ccc}
       \toprule
       NAC & Unscrunching and abortive RNA release & Promoter escape \\
       $9.0$ & $1.2$ & $20$ \\
    \end{tabular}
  \end{center}
\end{table}

\begin{figure}
    \begin{center}
      \includegraphics[scale=0.7]{../figures/cumul_scrunch_fit.pdf}
    \end{center}
    \caption{Initial transcription model fitted to distribution of time spent
      in rounds of abortive cycling before promoter escape. Measurements are
      from Revyakin et.\ al \cite{revyakin_abortive_2006}. Fit to measurements
      shows best fit by an exponential function. The green area indicates the
      region where abortive cycling lasts shorter than one second.}
\label{fig:revyakin_fit}
\end{figure}


% XXX leaving this one out for now
%\begin{figure}
    %\begin{center}
      %\includegraphics[scale=0.8]{../figures/ap_adjustment.pdf}
    %\end{center}
    %\caption{Effect on model fit, estimated rate constants, and model
      %performance when +GreB AP values are modified from baseline. A) Effect on
      %model fit; B) effect on rate constant of NAC; C) effect on rate constant of
      %the abortive step; D) effect on rate constant of escape; E) effect on
      %$\tau_r$, which is the model-obtained $\tau$ relative to the value found from
      %measurements from Vo et.\ al \cite{vo_vitro_2003-1} (Figure
      %\ref{fig:vo_comparison}).}
%\label{fig:ap_adjustment}
%\end{figure}


\begin{figure}
    \begin{center}
        \includegraphics[scale=0.7]{../figures/vo_greb_minus_comparison.pdf}
    \end{center}
    \caption{Kinetics of FL transcript synthesis under -GreB conditions.
      Commparison between model (light green line) and measurements from Vo
      et.\ al \cite{vo_vitro_2003-1} (green dots). Blue line shows fit of
      measurements to a single exponential function.}
\label{fig:vo_comparison}
\end{figure}

% XXX leaving this one out.
%\begin{figure}
    %\begin{center}
      %\includegraphics[scale=0.7]{../figures/Time_to_reach_fifty_percent_of_FL.pdf}
    %\end{center}
    %\caption{The value of $\tau$ obtained from simulation on -GreB N25 is
      %similar to the experimentally obtained value, and is distinct from
      %values of $\tau$ obtained by simulation on -GreB ITS variants of N25.}
%\label{fig:its_variant_fl_comparison}
%\end{figure}

% If you end up short with things in the results section, add something here
% about the sensitivity of the parameters. Define sensitivity as how much the
% value can be varied +/- before a 5% change in best fit is reached?

% Can we say that because the NAC is similar to elongation NAC, and that the
% ensemble experiment was so closely matched, that our two main assumptions
% are justified:
    % 1) that each backtracking event leads to an abortive release?
    % 2) that -GreBassisted cleavage and RNA release is rapid enough to be
    % negligeble for the oveall kinetics of the system?

%But how do we interpret the AP for the +GreB experiments? This is a bit of a
%conundrum. We are interpreting them as if GreB results in a reduction of
%the probability of backtracking at each position. That is OK, though, because
%then we only count the unscrunching steps that have proceeded without
%influence by GreB. We thus assume that rescue upon backtracking is fast
%enough.
