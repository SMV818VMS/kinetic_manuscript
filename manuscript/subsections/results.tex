%\addbibresource{/home/jorgsk/Dropbox/phdproject/bibtex/jorgsk.bib}
\subsection{Reaction rates during initial transcription}
Revyakin et.\ al measured initial transcription on the N25
promoter in the presence of GreB and with 100 $\mu$M NTP to find the
distribution of time for which RNAP is undergoing abortive cycling, and found
that it repetitive abortive cycling lasts on average around 5 seconds prior to
reaching promoter escape~\cite{revyakin_abortive_2006}. We constructed a
kinetic model, using the abortive profile from ensemble studies of
initial transcription on +GreB N25 from Hsu et. al~\cite{hsu_initial_2006},
and fitted rate constants for NAC, promoter escape, and unscrunhcing to the
cumulative distribution of time spent during abortive cycling as found by
Revyakin et.\ al (Figure~\ref{fig:revyakin_fit}). This resulted in the
following rate constants:~NAC $10.5/s \pm 0.64$; escape $11.84/s \pm 1.81$;
and unscrunching $1.58/s \pm 0.14$. The value for NAC before promoter escape
is highly similar to the transcription velocity after promoter escape: 9.5
nt/s for positively supercoiled DNA and 13.8 nt/s for negatively supercoiled
DNA~\cite{revyakin_abortive_2006}. This indicates that NAC proceeds at the
same rate for both scrunching and transcription elongation. The comparatively
lower rate of 1.58/s for unscrunching, including abortive RNA release, is
consistent with previous reports that this process is rate limiting for
initial transcription~\cite{margeat_direct_2006, revyakin_abortive_2006}.
Revyakin et.\ al did not measure rounds of abortive cycling that lasted
shorter than 1 second, but proposed based on extrapolation of their data that
20\% of abortive cycles last shorter than one second
\cite{revyakin_abortive_2006}. The best fit to the data by our kinetic model
suggests that less than 2\% of abortive cycling rounds last shorter than 1
second (Figure \ref{fig:revyakin_fit}), because one second is the average time
needed to reach promoter escape on N25 with a transcription rate of
$10.5$~nt/s for those few initiation attempts do not experience backtracking.

% Also, comment on the finding that 20% of scrunches last less than 1 second.
% Use the optimal parameters and sample 1000 times, find the average + std of
% # of scrunches that last less than 1 second. You found that 2.3% scrunches
% lasted than 1 sec +/- 1.8% allowing for the variation in 100 copy number
% reaction. (But perhaps not talk about that here?)
% XXX OK, include this next, and then include the bar-figure of N25 below. The
% sizzle up the parameter estimation text, and then move on to ..
% introduction?

%# put before assaying the method. Then, put Vo result in perspective by
%comparing with N25A1, N25anti and N25-A1anti
While the rate coefficients were estimated from experimental data under +GreB
conditions, which is associated with relatively low APs, the model itself,
after fitting, does not any more depend on a specific abortive profile, and
should as such be valid for initial transcription on N25 also for -GreB
conditions, under which there is a much higher probability to abort
transcripts \cite{hsu_initial_2006}. To therefore assess the performance of
the model when given a different abortive profile than for which the model was
fitted, we simulated initial transcription for -GreB conditions on N25 to
compare the kinetics of FL transcript synthesis to that found by Vo et.\
al \cite{vo_vitro_2003-1}. We found a half-life of FL transcript synthesis,
denoted $\tau$, of 15.5 seconds, similar to the experimental result which we
found by interpolation to an exponential function to be 13.7 seconds
(Figure~\ref{fig:vo_match}). While these numbers appear close, there is no
other kinetic data for the N25 promoter in the literature which can tell us
about the variability of $\tau$ on N25 under conditions that lead to
alteration of the abortive profile. Therefore, we chose to compare with
$\tau$ values from simulated initial transcription for N25 promoter variants
that have a modified ITS, which also alters the abortive profile
\cite{hsu_initial_2006}. We simulated the kinetics of FL transcript synthesis
for N25 variants N25-anti and N25A1-anti, which have previously been shown to
vary from N25 in several quantitative parameters for initial transcription~
\cite{hsu_initial_2006,chan_anti-initial_2001,kammerer_functional_1986}. This
showed that the simulated and experimentally determined $\tau$ values are
highly similar compared to the values for the other N25 variants
\ref{fig:}<++>.

The parameter estimation procedure relies on that the abortive profile for the
single-molecule experiment, which cannot be measured with those techniques due
to the rapid kinetics of the process \cite{margeat_direct_2006,
revyakin_abortive_2006}, corresponds to the one found by ensemble studies. To
investigate the validity of this assumption, we wished to see how the parameter
estimation process would work with an abortive profile that did not correspond
to the actual experimental conditions of Revyakin et. al. We therefore
repeated the parameter estimation with the abortive profile of N25 for -GreB
conditions. The abortive profile without GreB is marked by a much higher
probability to abort transcription after synthesis of a +6 transcript,
reflecting increased abortive cycling in the absence of
GreB~\cite{hsu_initial_2006}. This led to an estimate for the rate constant of
NAC of $23.2/s \pm 0.4$ (Table~\ref{tab:param_fit_revyakin}), which is an
unlikely value since it would presume that NAC proceeds twice as fast during
scrunching than during elongation, proceeding at speeds normally found for NTP
concentrations of 1000 $\mu$M~\cite{bai_mechanochemical_2007}, 10 times higher
than actual experimental conditions. This shows that only the N25 +GreB
ensemble abortive profile, matching experimental conditions, leads to a
credible parameter fit. To further investigate the sensitivity to the abortive
profile, we increased in the AP at each position in the +GreB N25 abortive
profile with +3\%, +9\% and +15\%, and for each increase repeated the
parameter estimation processes. This showed that as AP increases, a larger
rate constant for NAC is required to reach optimal fit with experimental data;
an increase in AP of 3\% at each position necessitated a rate constance for
the NAC of 12.7/s, and for an increase of 9\% a rate of 15/s was required to
optimally fit the data (Figure \ref{fig:ap_adjustment.pdf}A). In the same
manner, we decreased AP with the same percentages and found that the rate
constant of NAC decreased in response (Figure \ref{fig:ap_adjustment.pdf}A).
This shows that modification to the experimentally obtained AP in either
direction leads to values for the rate constant of NAC that depart from the
rate of transcription elongation.

\begin{table}
    \label{tab:param_fit_revyakin}
    \begin{center}
        \caption{Fitted rate constants of initial transcription}
        \begin{tabular}{cccc}
            \toprule
            Greb & NAC & Escape & Abort \\
            $+$ & $10.1 \pm 0.6$ & $11.8 \pm 1.8$ & $1.6 \pm 0.1$ \\
            $-$ & $23.2 \pm 0.4$ & $15.3 \pm 1.5$ & $16.3 \pm 1.4$ \\
    \end{tabular}
    \end{center}
\end{table}


\begin{figure}
    \begin{center}
        \includegraphics[scale=0.7]{../figures/cumul_scrunch_fit.pdf}
    \end{center}
    \caption{Fit of model to single-molecule kinetic data.}
\label{fig:revyakin_fit}
\end{figure}


\begin{figure}
    \begin{center}
        \includegraphics[scale=0.7]{../figures/ap_adjustment.pdf}
    \end{center}
    \caption{Perturbation of AP changes optimal fit}
\label{fig:ap_adjustment}
\end{figure}

\begin{figure}
    \begin{center}
        \includegraphics[scale=0.7]{../figures/vo_greb_minus_comparison.pdf}
    \end{center}
    \caption{Comparison with ensemble kinetic data.}
\label{fig:vo_comparison}
\end{figure}

\begin{figure}
    \begin{center}
        \includegraphics[scale=0.7]{../figures/Time_to_reach_fifty_percent_of_FL.pdf}
    \end{center}
    \caption{The value of $\tau$ obtained from simulation on -GreB N25 is
    similar to the experimentally obtained value, and is distinct from values of
    $\tau$ obtained by simulation on -GreB ITS variants of N25.}
\label{fig:vo_comparison}
\end{figure}

%The findings in Figure \ref{fig:revyakin_fit} and \ref{fig:vo_comparison} give
%a strong indication that using the abortive profile is a robust method for
%modelling model initial transcription. Thus far, the abortive profile has been
%changed by considering experiments performed in the presence and absence of
%GreB.  However, the abortive profile will also change with mutations in the
%initial transcribed sequence (ITS) \cite{hsu_initial_2006}. To see the effect
%of ITS mutations on the kinetics of initial transcription, we obtained from
%the model the time needed to synthesize 50\% of FL product for the 43 N25
%promoter variants from Hsu et.\ al. This showed that ITS mutations lead to
%variation $\tau$ ranging from 11.1 seconds (DG146a) to 3min 40s
%(DG137a)(Figure~\ref{fig:hsu_hl_synthesis}.

% If you end up short with things in the results section, add something here
% about the sensitivity of the parameters. Define sensitivity as how much the
% value can be varied +/- before a 5% change in best fit is reached?

% Can we say that because the NAC is similar to elongation NAC, and that the
% ensemble experiment was so closely matched, that our two main assumptions
% are justified:
    % 1) that each backtracking event leads to an abortive release?
    % 2) that -GreBassisted cleavage and RNA release is rapid enough to be
    % negligeble for the oveall kinetics of the system?

%But how do we interpret the AP for the +GreB experiments? This is a bit of a
%conundrum. We are interpreting them as if GreB results in a reduction of
%the probability of backtracking at each position. That is OK, though, because
%then we only count the unscrunching steps that have proceeded without
%influence by GreB. We thus assume that rescue upon backtracking is fast
%enough.
