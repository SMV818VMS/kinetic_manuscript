%\addbibresource{/home/jorgsk/Dropbox/phdproject/bibtex/jorgsk.bib}
\SUBSECTION{Transcription proceeds at a similar average speed for initial
transcription as for transcription elongation}

We identified the rate constants for initial transcription using a two-step
method (see Methods). We found in the first step that the rate constant for
NAC is greater than 6/s; that the rate constant for unscrunching and abortive
RNA release lies between 1/s and 3/s; and that the rate constant for promoter
escape is greater than 2.3/s
(\FIG~\ref{fig:parameter_estimation_two_iterations}A). The distribution for the
promoter escape rate constant (right panel) did not exhibit any optimal peak,
which indicates that the time that RNAP spends in abortive cycling is not
sensitive to the rate constant for promoter escape. This is reasonable since
the reaction for promoter escape occurs downstream the sequence and does not
affect the duration of abortive cycling.   

In the second step, we varied the rate constant for the NAC between 6/s and
14/s, and UAR between 1/s and 3/s, based on the distributions of these values
from the first step (\FIG~\ref{fig:parameter_estimation_two_iterations}A). Because it
did not affect abortive cycling, we fixed the rate constant of promoter escape
to 20/s. The top 1\% best fitting results from the second round of estimation
resulted in clear peaks in the distributions for both rate constants. The rate
constant for the NAC showed a peak at 10.6/s
(\FIG~\ref{fig:parameter_estimation_two_iterations}B), consistent with the measured
speed of elongating RNAP (9.2 nt/s on positively supercoiled DNA and 13.3 nt/s
on negatively supercoiled DNA \cite{revyakin_abortive_2006}). The rate constant
for UAR peaked at 1.4/s, nearly 10 times slower than the NAC. This low value is
consistent with experimental evidence, suggesting that unscrunching and
abortive RNA release are rate limiting for initial
transcription~\cite{revyakin_abortive_2006, margeat_direct_2006}. Overall, the
rate constants we estimated resulted in an accurate predictive model where most
experimental measurements fell within one standard deviation of the mean model
prediction (\FIG~\ref{fig:revyakin_fit}). 

\SUBSECTION{Short duration scrunching does not follow an exponential distribution}
  
Our model predicts that less than 4\% of scrunching events are shorter than
one second (\FIG~\ref{fig:revyakin_fit}). These quick events represent RNAP
transcribing the 11 required base pairs for promoter escape without
backtracking with a NAC $k_n$ of 10.6/s
(\FIG~\ref{fig:revyakin_fit}). This prediction is lower than the 20\%
predicted previously from extrapolation of data
(\cite{revyakin_abortive_2006}).  

To test the validity of our predictions, we first assumed that the previous
prediction, obtained by extrapolation of an exponential fitting
curve to measurements, was accurate. We then repeated the parameter
estimation (see Methods) but now the RMSE was calculated by the predicted
$\hat T$ from the fitted curve, not the data measurements (see also
\FIG~\ref{fig:param_estimation_scheme} and Algorithm 1). Fitting to the
extrapolated data resulted in an overestimated rate constant for the NAC
($k_n= 23.4$/s). Not only that this $k_n$ value is twice as high as that
reported for transcription elongation \cite{revyakin_abortive_2006}, but it is
higher than the speed of transcription elongation obtained when substrate
NTP-concentrations are saturated (ten times as high)
\cite{bai_mechanochemical_2007}. Furthermore, the value of $k_u$ was highly
overestimated (16.94, 11 times higher than what we estimated from measurement
data).


\SUBSECTION{Estimated rate constants describe initial transcription also
in the absence of GreB}

We estimated model parameters using two independent data sets (APs and
scrunching duration), both obtained in the presence of GreB. APs are increased
when GreB is absent \cite{hsu_initial_2006}, expressed in our model by higher
backtracking values $k_{b,i}$ (Eq. \ref{eq:backtrackingcalc}). The rate
constants that we estimated with data from +GreB, should also be valid for
initial transcription on the N25 promoter in the absence of GreB.  

To test the predictive ability of the model, we compared the predictions of
the model with $k_{b,i}$ calculated from -GreB APs \cite{hsu_initial_2006} to the
kinetics N25 full length (FL) transcript synthesis in the absence of GreB
\cite{vo_vitro_2003-1}. We expected RNAP to exhibit more abortive cycles
compared to when $k_b$ was calcualted from +GreB APs, leading to a lower rate
of full length product synthesis. We stress that this kinetic data consisted
of six measurements only, and contained dynamics we did not model, in which
unproductive complexes synthesize abortive RNA continuously
\cite{vo_vitro_2003-1}. However, these limitations should not compromise the
test of validity.
 
The comparison of the model predictions (blue solid line in
\FIG~\ref{fig:vo_comparison}) to the measured -GreB data (squares) indicate
the estimated rate constants (obtained from +GreB data) are valid also for
transcription initiation in the absence of GreB. The model
predicted the half-life of full length transcript synthesis in -GreB
conditions to be 17.4 s. In comparison, this value was 7.0 seconds under +GreB
conditions (\FIG~\ref{fig:vo_comparison}). 

\SUBSECTION{Model parameter estimation requires correct combination of
experimental data for fitting}

The APs \cite{hsu_initial_2006} and transcription data $\hat T$
\cite{revyakin_abortive_2006} used for rate constant estimation were
consistent in terms of the presence of GreB. As a test to the validity of the
model, any re-estimation of rate constants using inconsistent data (mix of
datasets obtained in +GreB and -Greb conditions) should not lead to plausible
results. To test our hypothesis, we re-estimated the kinetic parameters of the
model, now using -GreB APs \cite{hsu_initial_2006} instead of the +GreB APs
while we kept the same $\hat T_j$ data from Revyakin et al.\ (right dataset in
\FIG~\ref{fig:param_estimation_scheme}). The resulting distributions were
significantly different (Wilcoxon test, $p<10^{-10}$) compared to the
consistent +GreB case (\FIG~\ref{fig:GreB_minus_fit}), with the mean
distributions of unscrunching and abortive RNA release ($k_u$) moved from
1.5/s (original blue dash vertical line) to 17.3/s (gray dash line) and the
NAC ($k_n$) moved from 10.6 /s to 23.3 /s. The new NAC value is twice that
reported for transcription elongation after promoter escape
\cite{revyakin_abortive_2006}. Even if we assume these results are correct,
the value for unscrunching and abortive RNA release is similar to reported
transcription speeds, and therefore is not rate limiting for the initial
transcription process, which contradicts experimental findings
\cite{revyakin_abortive_2006, margeat_direct_2006}. We thus conclude that the
model is sensitive to inconsistent experimental data, and this strengthens its
validity.
