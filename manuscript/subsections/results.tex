%\addbibresource{/home/jorgsk/Dropbox/phdproject/bibtex/jorgsk.bib}
\subsection{The forward pathway of transcription proceeds at the same speed
for initial transcription as for elongation}
Revyakin et.\ al measured initial transcription on the N25
promoter in the presence of GreB and with 100 $\mu$M NTP to find the
distribution of time for which RNAP is undergoing abortive cycling, and found
that it repetitive abortive cycling lasts on average around 5 seconds prior to
reaching promoter escape~\cite{revyakin_abortive_2006}. We constructed a
kinetic model, using the abortive profile from bulk studies of
initial transcription on +GreB N25 from Hsu et. al~\cite{hsu_initial_2006},
and fitted rate constants for NAC, promoter escape, and unscrunhcing to the
cumulative distribution of time spent during abortive cycling as found by
Revyakin et.\ al (Figure~\ref{fig:revyakin_fit}). This resulted in the
following rate constants:~NAC $10.5/s \pm 0.64$; escape $11.84/s \pm 1.81$;
and unscrunching $1.58/s \pm 0.14$. The value for NAC before promoter escape
is highly similar to the transcription velocity after promoter escape: 9.5
nt/s for positively supercoiled DNA and 13.8 nt/s for negatively supercoiled
DNA~\cite{revyakin_abortive_2006}. This indicates that NAC proceeds at the
same rate for both scrunching and transcription elongation. The comparatively
lower rate of 1.58/s for unscrunching, including abortive RNA release, is
consistent with previous reports that this process is rate limiting for
initial transcription~\cite{margeat_direct_2006, revyakin_abortive_2006}.
Revyakin et.\ al did not measure rounds of abortive cycling that lasted
shorter than 1 second, but proposed based on extrapolation of their data that
20\% of abortive cycles last shorter than one second
\cite{revyakin_abortive_2006}. The best fit to the data by our kinetic model
suggests that less than 2\% of abortive cycling rounds last shorter than 1
second (Figure \ref{fig:revyakin_fit}), because one second is the average time
needed to reach promoter escape on N25 with a transcription rate of
$10.5$~nt/s for those few initiation attempts do not experience backtracking.

\subsection{Estimated rate constants describe kinetics of initial
    transcription also in the absence of GreB}
While the estimated values of NAC, escape, and scrunching led to a good match
to the data to which they were fitted, they were estimated for a specific
abortive profile, namely that of +GreB N25. If the estimated values are
generally valid, the fitted model should be able to capture the kinetics of
transcription on N25 for different abortive profiles, such as the one
obtained for transcription under -Greb conditions, for which the abortive
probability is considerably higher \cite{hsu_initial_2006}. To therefore
assess the performance of the model when given a different abortive profile
than for which the model was fitted, we simulated initial transcription for
-GreB conditions on N25 and compared with the kinetics of FL transcript
synthesis found by Vo et.\ al \cite{vo_vitro_2003-1}. The model gave a
half-life of FL transcript synthesis, denoted $\tau$, of 15.5 seconds, which
is similar to the experimental result which we found by interpolation to an
exponential function to be 13.7 seconds (Figure~\ref{fig:vo_match}). While
these numbers appear close, there is no other kinetic data for the N25
promoter in the literature which can tell us about the variability of $\tau$
under different abortive profiles. Therefore, we chose to compare with $\tau$
values from simulated initial transcription on N25 promoter variants that have
a modified ITS, which also alters the abortive profile
\cite{hsu_initial_2006}. We simulated the kinetics of FL transcript synthesis
for N25 variants N25-anti and N25A1-anti, which have previously been shown to
vary from N25 in several quantitative parameters for initial transcription~
\cite{hsu_initial_2006,chan_anti-initial_2001,kammerer_functional_1986}. This
showed that the simulated and experimentally determined $\tau$ values are
highly similar compared to the values for the other N25 variants \ref{fig:}.

\subsection{Abortive profile obtained by bulk studies is valid for
single-molecule initial transcription}
In estimating the rate constants, we have used an abortive profile from
bulk studies together with an abortive cycling distribution obtained from
single-molecule studies. This relies on the assumption that the abortive
profile for the single-molecule experiments is the same as the one obtained by
bulk studies. That this holds is not known, since the rapid kinetics of
initial transcription prevents direct observation of backtracking and abortive
RNA release \cite{margeat_direct_2006, revyakin_abortive_2006}. One way to
test if the abortive profiles are similar is to repeat the parameter
estimation process with an abortive profile that differs from the one of +GreB
N25. We therefore repeated the parameter estimation with the abortive profile
of N25 for -GreB conditions. This led to an estimate for the rate constant of
NAC of $23.2/s \pm 0.4$ (Table~\ref{tab:param_fit_revyakin}), which would
presume that NAC proceeds twice as fast during scrunching than during
elongation, proceeding at speeds normally found for NTP concentrations of 1000
$\mu$M~\cite{bai_mechanochemical_2007}, 10 times higher than actual
experimental conditions. This shows that the +GreB N25 bulk abortive
profile, but not the -GreB N25 profile, leads to a credible parameter fit. To
further investigate the sensitivity of parameter estimation to the abortive
profile, we increased the AP at each position in the +GreB N25 abortive
profile with +3\%, +9\% and +15\%, and for each increase re-estimated
parameters. This showed that as AP increases, a larger rate constant for NAC
is required to reach optimal fit with experimental data; an increase in AP of
3\% at each position necessitated a rate constance for the NAC of 12.7/s, and
for an increase of 9\% a rate of 15/s was required to optimally fit the data
(Figure \ref{fig:ap_adjustment.pdf}A). In the same manner, we decreased AP
with the same percentages and found that the rate constant of NAC decreased in
response (Figure \ref{fig:ap_adjustment.pdf}A).

\begin{table}
  \label{tab:param_fit_revyakin}
  \caption{Fitted rate constants of initial transcription}
  \begin{center}
    \begin{tabular}{cccc}
       \toprule
       Greb & NAC & Escape & Abort \\
       $+$ & $10.1 \pm 0.6$ & $11.8 \pm 1.8$ & $1.6 \pm 0.1$ \\
       $-$ & $23.2 \pm 0.4$ & $15.3 \pm 1.5$ & $16.3 \pm 1.4$ \\
    \end{tabular}
  \end{center}
\end{table}


\begin{figure}
    \begin{center}
      \includegraphics[scale=0.7]{../figures/cumul_scrunch_fit.pdf}
    \end{center}
    \caption{Initial transcription model fitted to distribution of time spent
      in rounds of abortive cycling before promoter escape. Measurements are
      from Revyakin et. al \cite{revyakin_abortive_2006}. Fit to measurements
      shows best fit by an exponential function. To the left in green is
      indicated the area of the plot where time is less than one second.}
\label{fig:revyakin_fit}
\end{figure}


\begin{figure}
    \begin{center}
        \includegraphics[scale=0.7]{../figures/ap_adjustment.pdf}
    \end{center}
    \caption{Perturbation of AP changes optimal fit.  A: Change of best fit
      for rate constant of NAC. B: $\tau$ from initial transcription with
    -GreB with best fitted rate constants.}
\label{fig:ap_adjustment}
\end{figure}


\begin{figure}
    \begin{center}
        \includegraphics[scale=0.7]{../figures/vo_greb_minus_comparison.pdf}
    \end{center}
    \caption{Kinetics of FL transcript synthesis under -GreB conditions.
      Commparison between model (light green line) and measurements from Vo
      et.\ al \cite{vo_vitro_2003-1} (green dots). Blue line shows fit of
      measurements to a single exponential function.}
\label{fig:vo_comparison}
\end{figure}


\begin{figure}
    \begin{center}
      \includegraphics[scale=0.7]{../figures/Time_to_reach_fifty_percent_of_FL.pdf}
    \end{center}
    \caption{The value of $\tau$ obtained from simulation on -GreB N25 is
      similar to the experimentally obtained value, and is distinct from
      values of $\tau$ obtained by simulation on -GreB ITS variants of N25.}
\label{fig:vo_comparison}
\end{figure}

% If you end up short with things in the results section, add something here
% about the sensitivity of the parameters. Define sensitivity as how much the
% value can be varied +/- before a 5% change in best fit is reached?

% Can we say that because the NAC is similar to elongation NAC, and that the
% ensemble experiment was so closely matched, that our two main assumptions
% are justified:
    % 1) that each backtracking event leads to an abortive release?
    % 2) that -GreBassisted cleavage and RNA release is rapid enough to be
    % negligeble for the oveall kinetics of the system?

%But how do we interpret the AP for the +GreB experiments? This is a bit of a
%conundrum. We are interpreting them as if GreB results in a reduction of
%the probability of backtracking at each position. That is OK, though, because
%then we only count the unscrunching steps that have proceeded without
%influence by GreB. We thus assume that rescue upon backtracking is fast
%enough.
