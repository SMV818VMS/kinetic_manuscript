%\addbibresource{/home/jorgsk/Dropbox/phdproject/bibtex/jorgsk.bib}
\subsection{Transcription proceeds at the same speed for initial transcription
as for transcription elongation}

To identify the rate constants for initial transcription we used two
iterations of rate constant estimation. In the first iteration we considered the
minimum and maximum values of all rate constants to lie between 1 s$^{-1}$ and
25 s$^{-1}$. By evaluating the model within these values (see Materials and
Methods), we found that the rate constant for NAC is greater than 7 s$^{-1}$,
and that the rate constant for unscrunching and abortive RNA release lies
between 1 and 3 s$^{-1}$ (Figure~\ref{fig:parameter_estimation_proper}A).  The
rate constant for promoter escape should be greater than 2.3 s$^{-1}$, but the
distribution does not show any clear optimal value. The lack of an optimal
value indicates that time spent in abortive cycling is not sensitive to the
rate constant for promoter escape
(Figure~\ref{fig:parameter_estimation_proper}A).

Having thus obtained more narrow boundaries for the rate constants, we
performed a more focused search. Since model fitting was insensitive to the
rate constant of promoter escape
(Figure~\ref{fig:parameter_estimation_proper}A), we varied only the rate
constants for the NAC and unscrunching and abortive RNA release in this second
round. The top 1\% best fitting results from the second round of screening
showed clear distributions for both rate constants.  The rate constant for the
NAC showed a peak around 10 s$^{-1}$ (Figure
\ref{fig:parameter_estimation_proper}B), with the peak of the distribution at
10.4 s$^{-1}$. This compares well with the speed of transcription after
promoter escape of 9.2 nt/s (positively supercoiled DNA) and 13.3 nt/s
(negatively supercoiled DNA) (Revyakin et al. Figure S14
\cite{revyakin_abortive_2006}), which indicates that the speed of the NAC is
unaffected by both RNAP's attachment to the promoter and DNA scrunching. The
rate constant for unscrunching and abortive release showed a peak around 1.3
s$^{-1}$, nearly 10 times slower than the NAC (Figure
\ref{fig:parameter_estimation_proper}B). This lower value is in agreement with
experimental evidence indicating that unscrunching and abortive RNA release
are the rate limiting steps for initial
transcription~\cite{revyakin_abortive_2006, margeat_direct_2006}. A full
kinetic scheme highlighting the identified rate constants is given in Figure
\ref{fig:estimated_parameters}A.

\begin{figure}
	\begin{center}
      \includegraphics[scale=0.8]{../figures/coarse_and_finegrain_search_GreB_no_extrap.pdf}
	\end{center}
    \caption{Distributions of rate constants after fitting 100000 randomly
      sampled values and selecting the 1000 (1\%) best fits. \textbf{A}: First round
      of screening, where all three rate constants were varied between 1 and
      25 s$^{-1}$. \textbf{B}: Second round of screening, where NAC was varied
      between 6 and 16 s$^{-1}$ and unscrunching and abortive RNA release
      (UAR) was varied between 1 and 3 s$^{-1}$. The promoter escape rate
      constant was held constant at 20 s$^{-1}$ for the second round.}
      \label{fig:parameter_estimation_proper}
\end{figure}

\begin{figure}
	\begin{center}
      \includegraphics[scale=1.0]{../figures/coarse_search-fits_2_3.pdf}
	\end{center}
    \caption{
      Rate constant distributions obtained by fitting the model to
      extrapolated and inconsistent experimental data. \textbf{A}:
      Distributions of top 1\% rate constants after fitting model to
      extrapolated curve for the distribution of time spent in abortive
      cycling \cite{revyakin_abortive_2006}. \textbf{B}: Distributions of top
      1\% rate constants after fitting model using AP values obtained in the
      absence of GreB. In silhouette is shown the distribution obtained with
      matching experimental input from Figure \ref{fig:parameter_estimation_proper}}
      \label{fig:extrap_and_GreB_minus_fit}
\end{figure}

Figure \ref{fig:revyakin_fit} shows the fit of the model using the estimated
rate constants. There is a close match between model and measured values. Also
shown in Figure \ref{fig:revyakin_fit} is an extrapolation of the measurement
data, originally made by Revyakin et al. \cite{revyakin_abortive_2006}.
Revyakin et al.\ did not measure cycles of abortive initiation lasting shorter
than ~3.5 seconds, but proposed based on the extrapolation curve that 20\% of
abortive cycles are shorter than one second \cite{revyakin_abortive_2006}.  In
contrast, the best fit by our kinetic model to the measurements suggests that
less than 2\% of abortive cycling before promoter escape last shorter than one
second (Figure \ref{fig:revyakin_fit}). If the extrapolation of the data
reflects the true scrunch-time distribution, more accurate rate constants
should be obtained by fitting the model to the extrapolated data rather than
the measured data. To test this, we repeated the rate constant estimation
procedure, but fitted to the extrapolated curve instead of the measurements.
This showed that the estimated rate constant for unscrunching and abortive RNA
release was similar to when fitting to the measured data, while 86\% of 
the NAC rate constant were estimated to be above 15 s$^{-1}$, with the
distribution peak 23.4 s$^{-1}$ (Figure \ref{fig:extrap_and_GreB_minus_fit}A).
The extrapolated data therefore would imply that the NAC proceeds during
initial transcription at speeds high above what has been observed for
transcription elongation \cite{revyakin_abortive_2006}.

\begin{figure}
	\begin{center}
      \includegraphics[scale=0.8]{../illustrations/estimated_parameters_filled.pdf}
	\end{center}
    \caption{Estimated rate constants for initial transcription on N25. The
        values for backtracking are calculated from the value of the NAC using
        equation \eqref{eq:backtrackingcalc}. \textbf{A}: Rate constants of
        backtracking calculated from APs from +GreB experiments. \textbf{B}:
        Rate constants of backtracking calculated from APs from -GreB
        experiments.}
    \label{fig:estimated_parameters}
\end{figure}

\subsection{Estimated rate constants are valid for initial transcription also
in the absence of GreB}
Since the experiments from Revyakin et al. were performed in the presence of
GreB, we originally fitted the model using APs calculated from bulk
experiments performed also in the presence of GreB. However, once fitted, the
estimated constants should be generally valid for initial transcription on the
N25 promoter, both with GreB present and absent. When GreB is absent, APs
increase \cite{hsu_initial_2006}, which in our model transfers into larger
rate constants for backtracking (Figure
S\ref{fig:parameter_estimation_scheme}). To test if the rate constants are
valid in the absence of GreB, we compared the model with the half-life of N25
full length (FL) transcript synthesis in the absence of GreB from kinetic data
from Figure 3B in Vo et al.\ \cite{vo_vitro_2003-1}. We fitted the data from Vo et
al.\ to a single exponential function to obtain a half life of FL transcript
synthesis of 13.7 s$^{-1}$. We then ran our model with -GreB AP values using
the rate constants estimated for +GreB AP values, which resulted in generally
higher rate constants of backtracking due to the higher AP values associated
with -GreB. This resulted in a mean model-predicted half-life of FL transcript
synthesis of 16.7 s$^{-1}$ (Figure~\ref{fig:vo_comparison}), which is close to
value found from the exponential fit. This shows that the estimated rate
constants are valid for the kinetics of initial transcription both in the
presence and in the absence of GreB, which indicates that they are generally
valid. The kinetic scheme of the estimated rate constants for initial
transcription on the N25 promoter in the absence of GreB is given in Figure
\ref{fig:estimated_parameters}B.

\subsection{Rate constant estimation requires consistent experimental data}
When fitting the model we combined experimental data for time spent in
abortive cycling and APs, where the two datasets were GreB-consistent, in the
sense thatin that they were obtained from initial transcription experiments
performed in the presence of GreB. If the model correctly represents the
kinetics of initial transcription, the estimation of rate constants should
react if fitting was performed with data that was GreB-inconsistent. To test
this, we re-fitted the model using the -GreB APs instead of the +GreB APs.
This re-fitting resulted in markedly different distributions of the rate
constants \ref{fig:extrap_and_GreB_minus_fit}B, with the peak of the
distributions being 23.6 s$^{-1}$ and 14.7 s$^{-1}$ for the NAC and
unscrunching and abortive RNA release, respectively, which we consider to be
outside a physiologically valid range for these rate constants. This shows
that model is sensitive to inconsistent experimental data during rate constant
estimation.


\begin{figure}
    \begin{center}
      \includegraphics[scale=0.7]{../figures/scrunch_times_cumulative_.pdf}
    \end{center}
    \caption{Initial transcription model fitted to distribution of time spent
      in abortive cycling before promoter escape. The mean has been found from
      100 simulations of 100 transcription events, with the shaded region
      indicating one standard deviation. Measurements are
      from Revyakin et al. \cite{revyakin_abortive_2006}. Fit to measurements
      shows best fit by an exponential function. The left green area indicates
      the region where abortive cycling would last shorter than one second.}
\label{fig:revyakin_fit}
\end{figure}


\begin{figure}
    \begin{center}
        \includegraphics[scale=0.7]{../figures/GreB_minus_kinetics.pdf}
    \end{center}
    \caption{Kinetics of FL transcript synthesis under -GreB conditions.
      Comparison between model and measurements from Vo
      et al. \cite{vo_vitro_2003-1}. The mean has been found from 10
      simulations of 1000 transcription events, with the shaded region
      indicating one standard deviation.}
\label{fig:vo_comparison}
\end{figure}

