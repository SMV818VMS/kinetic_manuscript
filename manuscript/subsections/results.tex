%\addbibresource{/home/jorgsk/Dropbox/phdproject/bibtex/jorgsk.bib}
\subsection{Reaction rates during initial transcription}
Several single-molecule studies of initial transcription have been undertaken
\cite{margeat_direct_2006,revyakin_abortive_2006,kapanidis_retention_2005}.
By following single transcription events with high accuracy, the results of
these studies can be well suited for informing computational models of
transcription. Revyakin et.\ al measured initial transcription on the N25
promoter in the presence of GreB and with 100 $\mu$M NTP to find that
individual RNAPs under these conditions spend on average around 5 seconds in
repetitive abortive cycling prior to reaching promoter
escape~\cite{revyakin_abortive_2006}. We used the abortive probabilities from
ensemble experiments of initial transcription on N25 in the presence of GreB
from Hsu et. al~\cite{hsu_initial_2006} to fit a computational model of
initial transcription to the cumulative distribution of time spent during
abortive cycling (Figure~\ref{fig:revyakin_fit}). This allowed us to estimate
average rate constants for the NAC, promoter escape, and unscrunhcing (see
Methods). This resulted in the following rate constants:~NAC $10.5/s \pm
0.64$; escape $11.84/s \pm 1.81$; and unscrunching $1.58/s \pm 0.14$. The
value for NAC is highly similar to the rate of NAC for elongation found by
Revyakin et.\ al: 9.5 nt/s for positively supercoiled DNA and 13.8 nt/s for
negatively supercoiled DNA~\cite{revyakin_abortive_2006}. This indicates that
NAC proceeds at the same rate during scrunching as for transcription
elongation. The comparatively lower rate of 1.58/s for unscrunching is
consistent with previous reports that this process is rate limiting for
abortive cycling~\cite{margeat_direct_2006, revyakin_abortive_2006}.
Importantly, this result indicates that the abortive profile obtained from
ensemble experiments is valid for the single molecule experiments of Revyakin
et.\ al, even though the limited time-resolution of these experiments cannot
resolve abortive events needed to obtain the abortive profile
\cite{revyakin_abortive_2006}.

To test the sensitivity of the paramter estimation process to the specific
abortive profile, we repeated the fitting of the data from Revyakin et.\ al
but this time to the abortive profile of N25 as obtained without GreB. The
profile without GreB is marked by a much high probabilty to abort
transcription after synthesis of a +6 transcript~\cite{hsu_initial_2006},
which shows that additional rounds of abortive cycling are likely when GreB is
absent. This fit resulted in the a value of $23.2/s \pm 0.4$ for the
rate constant of NAC (Table~\ref{tab:param_fit_revyakin}), which would
indicate that the NAC is faster during scrunching than during elongation, and
proceeds at speeds normally found for NTP concentrations 10 times higher than
actual experimental conditions \cite{bai_mechanochemical_2007}. This shows
that the parameter estimation is sensitive to the ensemble abortive profile,
and that only the profile which matches the single-molecule experiment in
terms of presence of GreB can be used to estimate rate constants that are
consistent with previous findings.
 
\begin{table} \label{tab:param_fit_revyakin}
    \begin{center}
        \caption{Fitted rate constants of initial transcription}
        \begin{tabular}{ccc}
            \toprule
            Greb & NAC & Escape & Abort \\
            $+$ & $10.1 \pm 0.6$ & $11.8 \pm 1.8$ & $1.6 \pm 0.1$\\
            $-$ & $23.2 \pm 0.4$ & $15.3 \pm 1.5$ & $16.3 \pm 1.4$\\
        \end{tabular}
    \end{center}
\end{table}

\begin{figure}
    \begin{center}
        \includegraphics{../figures/cumul_scrunch_fit.pdf}
    \end{center}
    \caption{Fit of model to single-molecule kinetic data.}
\label{fig:revyakin_fit}
\end{figure}

Having estimated rate constants for initial transcription, we wanted to
compare these results with previously published kinetic data on the N25
promoter. Ensemble kinetic experiments of initial transcription on N25 at 100
$\mu$M NTP have been published by Vo et.\ al which show the accumulation of FL
transcript for two different approaches toward single-round transcription
\cite{vo_vitro_2003-1}, both performed in the absence of GreB (denoted -GreB).
Using the rate constants obtained from single-molecule experiments, we found
that the estimated half-life of FL synthesis was 15.5 seconds, closely
matching the experimental result which we interpolated to be 13.7 seconds
(Figure \ref{fig:vo_match}). This shows that the rate constants from
Table~\ref{tab:param_fit_revyakin}, fitted to single-molecule +GreB
experiments, are descriptive for the kinetics of FL synthesis for -GreB
ensemble experiments when using a -GreB profile.

To again test the sensitivity of model performance for the specific abortive
profile, we repeated the simulation using the +GreB abortive profile. This
resulted in a half-life of FL synthesis of xx seconds, which again shows that
the model performs well only for an abortive profile that matches experimental
conditions.

\begin{figure}
    \begin{center}
        \includegraphics{../figures/vo_greb_minus_comparison.pdf}
    \end{center}
    \caption{Comparison with ensemble kinetic data.}
\label{fig:vo_comparison}
\end{figure}

The findings in Figure \ref{fig:revyakin_fit} and \ref{fig:vo_comparison} give
a strong indication that using the abortive profile is a robust method for
modelling model initial transcription. Thus far, the abortive profile has been
changed by considering experiments performed in the presence and absence of
GreB.  However, the abortive profile will also change with mutations in the
initial transcribed sequence (ITS) \cite{hsu_initial_2006}. To see the effect
of ITS mutations on the kinetics of initial transcription, we obtained from
the model the time needed to synthesize 50\% of FL product for the 43 N25
promoter variants from Hsu et.\ al. This showed that ITS mutations lead to
variation $\tau$ ranging from 11.1 seconds (DG146a) to 3min 40s
(DG137a)(Figure~\ref{fig:hsu_hl_synthesis}.

% Are there more results?
% You can calculate that the frequency of backtracking is 0.X/s, which is a
% lot more than the 0.0001/s for transcription elongation. This shows that 
%Using the APs to calculate the propensity of backtracking allows calculation
%of the frequency of backtracking during initial transcription. For
%transcription elongation, the rate of backtracking is at X/bp. For the N25
%variants, it varied from x/bp to y/bp (Figure S1).

% you could quantify the effect of AP by comparing tau for N25 -GreB by adding
% (or subtract) 10%, 20%, 30% etc to the APs to see the effect on the
% prediction of tau. Assuming that the distance from tau to the one estimated
% is within accepted limites, you can see how much an increase in AP
% (alternatively, backtracking frequency) affects the veracity of the results.

% Can we say that because the NAC is similar to elongation NAC, and that the
% ensemble experiment was so closely matched, that our two main assumptions
% are justified:
    % 1) that each backtracking event leads to an abortive release?
    % 2) that -GreBassisted cleavage and RNA release is rapid enough to be
    % negligeble for the oveall kinetics of the system?

%Thought experiment. What would happen if in reality there was a lot of really
%slow backtracking that did not produce abortive transcripts? Let's say, only
%1/N were actually produced, where N is number of nts, the rest are stuck in
%long-lived pauses. If this were the case, it would mean that the probability
%to backtrack would have to be a lot higher, since right now only a fraction of
%the complexes that backtrack actually contribute to the abortive probability.
%But we have shown that by changing the abortive probability, there is a large
%change in the dynamics of the system.

%I would say that because the NAC and unscrunching steps match previous
%findings, the assumption that GreB assisted cleavea and RNA release is rapid
%is valid.

%But how do we interpret the AP for the +GreB experiments? This is a bit of a
%conundrum. We are interpreting them as if GreB results in a reduction of
%the probability of backtracking at each position. That is OK, though, because
%then we only count the unscrunching steps that have proceeded without
%influence by GreB.
