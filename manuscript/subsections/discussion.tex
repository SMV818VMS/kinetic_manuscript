%Points to discuss:

% You have a lot of ``this indicates .. '' in the results part; perhaps move
% some of them here?

%* What errors are we incurring by using AP to find rate of backtracking? What
%are the assumptions (For GreB +/-).
    %- Assumption1: backtracking leads to rapid abortive RNA release, and each
    %backtracking event leads to abortive RNA release at similar rates. This
    %has got to be an approximation, judging from the action of GreB.
    %- Assumption2: rescue by GreB is fast

%In spite of these simplifying assumption, the model has done well.

%while at the same time being faster than 1/s which is the
%time-resolution of ~\cite{margeat_direct_2006, revyakin_abortive_2006}
% All values should be considered averages. For example, margeat did not
% resolve unscrunhcing for a leq 7 transcript with their 400ms = 2.5/s
% resolution. On average faster than the 1/s which was the time resolution of
% Revyakin et. al who could not resolve unscrunching with their 1/s
% resolution.

% Not 20% of events last less than 1 seond, but we have to consider that their
% results were obtained with positively and negatively wound DNA on a plasmid,
% while the abortive profile is obtained 

% For HL synthesis, mention (once more?) that 

% Mention that the Vo et. al experiments have some re-initiaion, which
% explains why the experimental values reach maximum more slowly.

% For comparing with FL of Vo, there is another caveat. Lilian's experiments
% are steady state for 10 minutes, which means that APs might be a bit higher
% than what they should be due to the unproductive complexes. FL kinetics is
% reflecting only the productive complexes in the beginnign, who probably have
% a lower probablity to abort at +2 and +3. Damn it it would be nice to find
% out about this. Should you give it another shot? You might actually have had
% it wrong last time: vo figure does not give AP for unproductive, but rather
% abortive %. Can you calculate AP from this? Yes you can, assuming 0 FL
% transcript (although, it may be produced, but just very very slowly). The
% method would be: use APs from unproductive; estimate unproductives at 10%
% +/- 5%, and estimate productive AP at +2, +3, and +4, where 85% of
% unproductive AP is produced. But if you do that, would you have to go back
% to re-estimate N25 stuff with lower APs in the beginning? It might actually
% lead to a better fit, especially in the beginning! Well I'd say that this
% has now become more plausible, but it's work for another time, for another
% manuscript, which you'll probably never have time to work on :S!! It means
% that the nature of unproductive and productive initial transcription may not
% be uncovered for a very long time! :S
% XXX In conclusion: do not mention anything about this here, but leave it
% open for future work. It would make the story more complicated.

%Importantly, this result indicates that the abortive
%profile obtained from ensemble experiments is valid for the single molecule
%experiments of Revyakin et.\ al, even though the limited time-resolution of
%these experiments cannot resolve abortive events needed to obtain the abortive
%profile \cite{revyakin_abortive_2006}.

% Interesting, but not perhaps something to mention. RNAP seems to be
% stochastic in its speed of trancsription, but with little variability for a
% given RNAP, and with a lot of variability between RNAPs. This indicates that
% the stochastic regime used here (sometimes fast, sometimes slow) is not so
% descriptive of RNAP translocation. But the productive/unproductive kinetics
% do not support this gradual difference. So there!

%Thought experiment. What would happen if in reality there was a lot of really
%slow backtracking that did not produce abortive transcripts? Let's say, only
%1/N were actually produced, where N is number of nts, the rest are stuck in
%long-lived pauses. If this were the case, it would mean that the probability
%to backtrack would have to be a lot higher, since right now only a fraction of
%the complexes that backtrack actually contribute to the abortive probability.
%But we have shown that by just slightly changing the abortive probability,
%the NAC does not ``fit well'' any more.

%I would say that because the NAC and unscrunching steps match previous
%findings, the assumption that GreB assisted cleavea and RNA release is rapid
%is valid.

%The model will not fit a -GreB abortive profile to a scruch-distribution
%generated under +GreB conditions.
