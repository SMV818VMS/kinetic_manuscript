%\addbibresource{/home/jorgsk/Dropbox/phdproject/bibtex/jorgsk.bib}
In this work we have used a kinetic model to show that promoter-bound
transcription by RNAP before promoter escape proceeds at an average speed of
around 10.6 nt/s
(\FIGS~\ref{fig:estimated_parameters}~and~\ref{fig:revyakin_fit}), highly
similar to the speed found for transcription elongation after promoter escape
\cite{revyakin_abortive_2006}. We found that the steps of unscrunching and
abortive RNA release are nearly ten times slower, proceeding at around 1.4/s.
These rate constants were validated by comparing with kinetic data obtained
under experimental conditions different to those to which the model was fitted
(\FIG~\ref{fig:vo_comparison}).

% Explain the meaning of the study
Knowing the rate constants of initial transcription is important for the
overall understanding of the mechanism of promoter-bound RNA synthesis and
abortive cycling. It has been proposed that the cause of both abortive RNA
release and promoter escape is the release of built-up energy caused by the
strain of the increasingly enlarged DNA bubble \cite{straney_stressed_1987,
hsu_promoter_2002, revyakin_abortive_2006}. While the precise nature of this
strain has not been identified, it is clear that it is unrelated to the free
energy of the DNA-DNA bubble or the RNA-DNA hybrid \cite{hsu_initial_2006,
skancke_sequence-dependent_2015}. It has instead been suggested that this
strain manifests itself through bulges in the template-strand of DNA, which
may contribute to the obligate displacement of different parts of $\sigma$
before promoter escape \cite{winkelman_crosslink_2015}. In our study, we have
shown that this strain does not have a marked impact on the rate of
transcription. In other words, our model predicts that the reactions in the
NAC (translocation, NTP binding, and pyrophosphorolysis) are on average not
delayed because of strain in RNAP and the presence of scrunched DNA. We have
further showed that the time spent in short abortive cycles or single
scrunching events ($< 2.5$ seconds) does not follows an exponential
distribution, as originally suggested
\cite{revyakin_abortive_2006}, since this would require an unphysiologically
large rate constant of the NAC of 23.4/s in order to fit the
distribution curve (\FIG~\ref{fig:extrap_and_GreB_minus_fit}A). Instead, when
using the rate constant of the NAC obtained by simply fitting the measured
data, we predicted a transient delay of about one second before the first
promoter escape events come to pass (\FIG~\ref{fig:revyakin_fit}).

% Relate the findings to similar studies
% Perhaps combine with limitation? Mention that Tang got position-specific
A limitation of the current study is that it only describes the average rate
of initial transcription, since the published transcription data is
insufficient to resolve variation at basepair resolution. However, the speed
of transcription is known to vary depending on sequence context
\cite{bai_mechanochemical_2007,malinen_active_2012}. Further, structural
changes during initial transcription, such as displacement of $\sigma_{3,2}$
by the growing RNA, may contribute to initiation-specific variation in the
speed of RNA chain elongation. In a study on initial transcription on the T7
RNAP, Tang et al. used high-resolution transcription data to quantify how the
speed of transcription varies for each translocation step up to promoter
escape \cite{tang_real-time_2009}. While the T7 RNAP is sufficiently different
to the \textit{E. coli} RNAP studied here that these results are not directly
transferable, we expect that some variation in the speed of transcription will
occur in bacterial RNAP as well. At the same time, the average speed of
transcription we identified was highly similar to the average speed identified
for transcription elongation upon promoter escape. This indicates that if
there are steps during initial transcription where the speed is reduced, these
steps must either be few, or if they are several, the reduction in speed must
be small. In other words, our model is consistent with a picture where the
strain of the growing DNA bubble and steric clashes with $\sigma$ do not
greatly affect the NAC. Instead, our results are consistent with a model
where transcription during initiation proceeds on average similar to
elongation except for a higher probability to backtrack.

With the present available data the model can not distinguish the likely
variation in the rate constant of unscrunching and abortive RNA release from
ITCs with RNAs of varying lengths. Since is known that GreB cannot contribute to
the rescue of backtracked complexes with RNAs shorter than 5 nt
\cite{hsu_initial_2006}, it can be expected that there is an
RNA-length-dependence in the rate of unscrunching and abortive RNA release.
Work on elongation complexes have shown that a weakened RNA-DNA hybrid
prevents backtracking \cite{nudler_rnadna_1997}, making it likely that
unscrunching takes longer for a complete compared to a partially formed
RNA-DNA hybrid. To resolve this number more precisely, experimental techniques
are needed that can resolve backtracking and abortive RNA release at the time
scale in which they occur. Alternatively, as increasingly complex molecular
simulations of RNAP dynamics are obtained \cite{silva_millisecond_2014}, the
details of the rapid kinetics of abortive cycling may be instead first be
obtained using numerical methods alone.

In conclusion, this work supports a model of initial transcription where the
forward rate of transcription is highly similar to the rate for transcription
elongation, and where the backtracking and abortive release step is nearly 10
times slower. Since model fitting resulted in physiologically meaningful rate
constants, and since these rate constants held up to independent kinetic data
(\FIG~\ref{fig:vo_comparison}), this supports that the underlying assumption
that APs reflect the probability to backtrack is sound. This suggests that APs
may be considered for future modelling work to investigate the mechanisms
behind the large variation in abortive RNA release that occurs from one
template position to the next \cite{hsu_initial_2006}.
