%\addbibresource{/home/jorgsk/Dropbox/phdproject/bibtex/jorgsk.bib}

In this work we have used a kinetic model to show that promoter-bound
transcription by RNAP before promoter escape proceeds at an average speed of
around 10.6 nt/s
(\FIGS~\ref{fig:parameter_estimation_two_iterations}~and~\ref{fig:revyakin_fit}), highly
similar to the speed found for transcription elongation after promoter escape
\cite{revyakin_abortive_2006}. We found that the steps of unscrunching and
abortive RNA release are nearly ten times slower, proceeding at 1.5 nt/s. We
validated these rate constants by comparing with kinetic data obtained under
experimental conditions different to those to which the model was fitted
(\FIG~\ref{fig:vo_comparison}).

% Explain the meaning of the study
Knowing the rate constants of initial transcription is important for the
overall understanding of the mechanism of promoter-bound RNA synthesis and
abortive cycling. It has been proposed that the cause of both abortive RNA
release and promoter escape is the release of built-up energy caused by the
strain of the increasingly enlarged DNA bubble \cite{straney_stressed_1987,
hsu_promoter_2002, revyakin_abortive_2006}. While the precise nature of this
strain has not yet been identified, it is most likely unrelated to the free
energy of the DNA-DNA bubble or the RNA-DNA hybrid \cite{hsu_initial_2006,
skancke_sequence-dependent_2015}. Winkelman et al.\ suggested that this strain
manifests itself through bulges in the template-strand of DNA, which may
contribute to the obligate displacement of different parts of $\sigma$ before
promoter escape \cite{winkelman_crosslink_2015}. Our model's prediction of
similar RNAP velocities before and after promoter escape has a clear
implication. Because the NAC proceeds at a similar rate during initiation and
elongation, and the strain that emerges as the DNA bubble grows does not exist
in elongating RNAPs, we conclude that the rates of NAC reactions
(translocation, NTP binding, and pyrophosphorolysis) during initial
transcription are not strongly affected by the existence of this strain.

We further showed that the time spent in short abortive cycles or single
scrunching events ($< 2.5$ seconds) does not follow an exponential function
as was originally suggested \cite{revyakin_abortive_2006}, since this would
require an unfeasible rate constant of the NAC of above 23 nt/s. Instead, when
using the rate constant of the NAC obtained by fitting the measured data, we
predicted a transient delay of about one second before the first promoter
escape events come to pass (\FIG~\ref{fig:revyakin_fit}).

% Relate the findings to similar studies
% Perhaps combine with limitation? Mention that Tang got position-specific
A limitation of the current study is that it only describes the average rate
of initial transcription, owing to a lack of available experimental data
needed to resolve variation at basepair resolution. However, it is generally
known that the speed of transcription varies depending on sequence context
\cite{bai_mechanochemical_2007,malinen_active_2012}. Furthermore, structural
changes during initial transcription, such as displacement of $\sigma_{3,2}$
by the growing RNA, may contribute to initiation-specific variation in the
speed of RNA chain elongation. In a study on initial transcription on the T7
RNAP, Tang et al.\ used high-resolution transcription data and showed that the
speed of transcription varied for each translocation step up to promoter
escape \cite{tang_real-time_2009}. Although the T7 RNAP structure is different
than the \textit{E.\ coli} RNAP we studied, variations in the speed of
bacterial RNAP translocation may be expected as well from the above listed
reasons. The mean speed of transcription that we identified however was highly
similar to the mean speed identified post promoter escape. This implies the
following: 1) the speed is reduced over only small number of nucleotides during
initial transcription and does not affect the mean speed prior to promoter
escape; 2) the reduction in speed is not significant enough to affect the mean
speed; 3) the speed reduction for some nucleotides is very strong, but is
compensated by rapid speed for other nucleotides so the mean remains
unaffected. There is of yet no experimental support in the latter argument for
rapid steps during initiation, but it can be tested with a sequence-sensitive
model using specific sequences that consists of 'rapid' and slow transcription
speeds. The precise speed of transcription at different nucleotide position
during initial transcription is subject to further research.

With the present available data the model can not distinguish the likely
variation in the rate constant of unscrunching and abortive RNA release from
ITCs with RNAs of varying lengths. Since is known that GreB cannot contribute
to the rescue of backtracked complexes with RNAs shorter than 5 nt
\cite{hsu_initial_2006}, it can be expected that there is an
RNA-length-dependence in the rate of unscrunching and abortive RNA release.
Work on elongation complexes have shown that a weakened RNA-DNA hybrid
prevents backtracking \cite{nudler_rnadna_1997}, making it likely that
unscrunching takes longer for a fully formed compared to a partially formed
RNA-DNA hybrid. To resolve this rate constant more precisely, experimental
techniques are needed that can resolve backtracking and abortive RNA release
at the time scale in which they occur. Alternatively, as increasingly complex
molecular simulations of RNAP dynamics are obtained
\cite{silva_millisecond_2014}, the details of the rapid kinetics of abortive
cycling may be instead first be obtained using numerical methods alone.

In conclusion, our model is consistent with a picture in which the strain of
the growing DNA bubble and steric clashes with $\sigma$ do not greatly affect
the NAC. Instead, our results are consistent with a model where the speed of
transcription during initiation is on average similar to elongation. Our model
supports the underlying assumption that APs reflect the probability to
backtrack. We suggest that APs may be considered for future modelling work in
order to investigate the mechanisms behind the large variation in abortive RNA
release that occurs from one template position to the next
\cite{hsu_initial_2006}.
