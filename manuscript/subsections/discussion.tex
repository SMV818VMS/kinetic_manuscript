Points to discuss:

%* What errors are we incurring by using AP to find rate of backtracking? What
%are the assumptions (For GreB +/-).
    %- Assumption1: backtracking leads to rapid abortive RNA release, and each
    %backtracking event leads to abortive RNA release at similar rates. This
    %has got to be an approximation, judging from the action of GreB.
    %- Assumption2: rescue by GreB is fast

%In spite of these simplifying assumption, the model has done well.


%while at the same time being faster than 1/s which is the
%time-resolution of ~\cite{margeat_direct_2006, revyakin_abortive_2006}
% All values should be considered averages. For example, margeat did not
% resolve unscrunhcing for a leq 7 transcript with their 400ms = 2.5/s
% resolution. On average faster than the 1/s which was the time resolution of
% Revyakin et. al who could not resolve unscrunching with their 1/s
% resolution.

% Not 20% of events last less than 1 seond, but we have to consider that their
% results were obtained with positively and negatively wound DNA on a plasmid,
% while the abortive profile is obtained 

% For HL synthesis, mention (once more?) that 

% Mention that the Vo et. al experiments have some re-initiaion, which
% explains why the experimental values reach maximum more slowly.
