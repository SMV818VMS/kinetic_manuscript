%\addbibresource{/home/jorgsk/Dropbox/phdproject/bibtex/jorgsk.bib}
In this work we have used a kinetic model to show that promoter-bound
transcription by RNAP proceeds at around 10 nts$^{-1}$
(Figure~\ref{fig:revyakin_fit}, Table~\ref{tab:param_fit_revyakin}), highly
similar to the speed under identical reaction conditions for
transcription elongation \cite{revyakin_abortive_2006}. We found that 
the steps of unscrunching and abortive RNA release are nearly ten times
slower, proceeding at around 1.3 s$^{-1}$ (Table~\ref{tab:param_fit_revyakin}).
These rate constants were validated by comparing with kinetic data obtained
under experimental conditions different than those to which the model was
fitted (Figure~\ref{fig:vo_comparison}).

% Explain the meaning of the study
Knowing the rate constants of initial transcription adds to the overall
understanding of the mechanism of promoter-bound RNA synthesis and abortive
cycling. It has been proposed that the cause of both abortive RNA release and
promoter escape is the release of built-up energy caused by the strain of the
increasingly enlarged DNA bubble
\cite{straney_stressed_1987,hsu_promoter_2002,revyakin_abortive_2006}. Since
then, it has been found that this strain seems unrelated to the free energy of
the DNA-DNA bubble \cite{hsu_initial_2006, skancke_sequence-dependent_2015},
and that the strain seems largely unaffected by nicks and gaps in the
scrunched DNA bubble \cite{samanta_insights_2013}. In our study, we have shown
that this strain additionally does not have a marked impact on the rate of
transcription (Table~\ref{tab:param_fit_revyakin}). In other words, our model
predicts that the rate constants of translocation, NTP binding, and
pyrophosphorolysis do not differ greatly between transcription with promoter
bound and elongating RNAPs. We have further showed that the extrapolation
of time spent in abortive scrunching as described by Revyakin et al.
necessitates speeds of the NAC above that which has been found for
transcription elongation (Figure \ref{fig:extrap_and_GreB_minus_fit}A).
Instead, the best fitting rate constants predict a transient delay of about
one second before the first promoter escapes are reached (Figure
\ref{fig:revyakin_fit}).

% Relate the findings to similar studies
The most similar study to the work presented here is a study of the kinetics
of initial transcription with the T7 RNAP by Tang et al.
\cite{tang_real-time_2009}. Using time-series data, they discerned rate
constants for the individual translocation steps, showing a 10-fold variation
for the NAC rate constant, from 6 nts$^{-1}$ to 60 nts$^{-1}$
\cite{tang_real-time_2009}. This raises the question if there is a similar
variation in the rate constant of the NAC for \textit{E. coli} RNAP. We 
argue that this is not the case, since the average rate constant we obtained
coincided closely with the speed during transcription elongation. The speed
during transcription elongation should be close to the upper bound of what may
be achieved for promoter-bound RNAP, which means that if the averages
coincide, there is little room for individual steps at a speed much slower than
10 nts$^{-1}$. Therefore, if there are any individual NACs during
initial transcription that are slow, they are not likely to be much slower
than the average. Our results therefore support a model where there is no
clear bottleneck step for the NAC during initial transcription, implying that
the strain of the growing DNA bubble and steric clashes with $\sigma$ do
not greatly affect the NAC. Instead, transcription during initiation proceeds
similarly as for elongation, except for a higher probability to backtrack,
which has been shown to have a sequence-dependent context
\cite{skancke_sequence-dependent_2015}.

% Consider alternative explanations of the findings

% Acknowledge the study's limitation
While the average is likely a good measure for the NAC, a limitation of the
current study is that it does not discern variation in the rate of unscrunching
and abortive RNA release from initial complexes with different RNAs. It is
known that GreB cannot contribute to the rescue of backtracked complexes with
RNAs shorter than 5 nt \cite{hsu_initial_2006}, implying a length-dependence
in the speed of unscrunching and abortive RNA release. Work on elongation
complexes have shown that a weakened RNA-DNA hybrid prevents backtracking
\cite{nudler_rnadna_1997}, making it likely that unscrunching takes longer for
a complete compared to a partially formed RNA-DNA hybrid. At the same time,
the rate constant of 1.2 s$^{-1}$ identified here just above the ~1 s$^{-1}$
temporal resolution of the experiment of Revyakin et al., where this reaction
step could not be discerned \cite{revyakin_abortive_2006}, implying that the
value of 1.2 s$^{-1}$ represents a lower boundary. To resolve this number more
precisely, experimental techniques are needed that can resolve backtracking
and abortive RNA release at the time scale in which they occur. Alternatively,
as increasingly complex molecular simulations of RNAP dynamics are obtained
\cite{silva_millisecond_2014}, the details of the rapid kinetics of abortive
cycling may be instead first be obtained using numerical methods alone.

A key assumption in our model is that the APs calculated from the abundance of
aborted RNA indicate the probability of the initial backtracking step
(Figure~\ref{fig:model_and_rates}). There remains however the possibility that
the actual probability to backtrack is higher than what is observed from the
APs. This would occur if complexes remain in long backtracked
pauses, as has been observed for transcription elongation
\cite{shaevitz_backtracking_2003}. If this phenomenon occurs, it should further
be more prominent in experiments performed in the absence of GreB, since GreB
has been found to reduce the duration of backtracked pauses
\cite{shaevitz_backtracking_2003}. However, we showed that our model fits data
obtained both from +GreB and -GreB experiments (Figures
\ref{fig:revyakin_fit}, \ref{fig:vo_comparison}). This indicates that the APs
are sufficient to describe the backtracking probability during initial
transcription, and that if long backtracked pauses exist, they do not make a
large contribution to the kinetics of initial transcription.

In conclusion, this work supports a model of initial transcription where the
forward rate of transcription is highly similar to the rate for transcription
elongation, and where the backtracking and abortive release step is more than
5 times slower. Since model fitting resulted in physiologically reasonable
rate constants (Table~\ref{tab:param_fit_revyakin}), and since these rate
constants hold up to independent kinetic data
(Figure~\ref{fig:vo_comparison}), this supports that the assumptions
underlying the model are sound. Therefore, it is likely that the APs
reflect probabilities to backtrack during initial transcription, and it is
probable that the APs are the same for the single-molecule experiments of
Revyakin et al.\ \cite{revyakin_abortive_2006} and the steady state
transcription experiments by Hsu et al.\ \cite{hsu_initial_2006}.
 
% Q:You mentioned something to Itziar, something that you decided not to
% include! Yes, separating the productive from the unproductive transcripts.
% You can do this by analyzing the kinetics from a single round of
% transcription.
