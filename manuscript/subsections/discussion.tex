%\addbibresource{/home/jorgsk/Dropbox/phdproject/bibtex/jorgsk.bib}
In this work we have used a kinetic model to show that promoter-bound
transcription by RNAP before promoter escape proceeds at around 10.4 nt/s
(\FIGS~\ref{fig:revyakin_fit} and \FIG~\ref{fig:estimated_parameters}), highly
similar to the speed obtained for transcription elongation after promoter
escape \cite{revyakin_abortive_2006}. We found that the steps of unscrunching
and abortive RNA release are nearly ten times slower, proceeding at around 1.4
s$^{-1}$. These rate constants were validated by comparing with kinetic data
obtained under experimental conditions different to those to which the model
was fitted (\FIG~\ref{fig:vo_comparison}).

% Explain the meaning of the study
Knowing the rate constants of initial transcription is important for the overall
understanding of the mechanism of promoter-bound RNA synthesis and abortive
cycling. It has been proposed that the cause of both abortive RNA release and
promoter escape is the release of built-up energy caused by the strain of the
increasingly enlarged DNA bubble \cite{straney_stressed_1987,
hsu_promoter_2002, revyakin_abortive_2006}. While the precise nature of this
strain has not been identified, it is clear that it is unrelated to the free
energy of the DNA-DNA bubble or the RNA-DNA hybrid \cite{hsu_initial_2006,
skancke_sequence-dependent_2015}. It has instead been suggested that this
strain manifests itself through bulges in the template-strand of DNA, which
may contribute to the obligate displacement of different parts of $\sigma$
before promoter escape \cite{winkelman_crosslink_2015}. In our study, we have
shown that this strain does not have a marked impact on the rate of
transcription. In other words, our model predicts that the rate constant of
the NAC (translocation, NTP binding, and pyrophosphorolysis) does not differ
greatly between transcription for promoter bound and elongating RNAP. We have
further showed that it is unlikely that the time spent in short abortive
cycles (< 3.5 seconds) follows an exponential distribution, as originally
suggested by Revyakin et al. \cite{revyakin_abortive_2006}, since this would
require an unphysiologically large rate constant of the NAC of 23.4 $^{-1}$ in
order to fit the distribution curve
(\FIG~\ref{fig:extrap_and_GreB_minus_fit}A). Instead, using the rate constant
of the NAC obtained by simply fitting to the measured data, the distribution
of time spent in abortive cycling is predicted to contain a transient delay of
about one second before the first promoter escape events come to pass
(\FIG~\ref{fig:revyakin_fit}).

% Relate the findings to similar studies
The most similar study to the work presented here is a study of the kinetics
of initial transcription with the T7 RNAP by Tang et al.
\cite{tang_real-time_2009}. Using time-series data, they discerned rate
constants for the individual translocation steps, showing a 10-fold variation
for the NAC rate constant, from 6 nt/s to 60 nt/s
\cite{tang_real-time_2009}. In our model, we have used an average value for
the NAC at all translocation steps. This raises the question if there is a
similar variation in the rate constant of the NAC for \textit{E. coli} RNAP.
Since the average rate constant we obtained coincided closely with the speed
during transcription elongation, we would argue that a large variation is
not likely for \textit{E. coli}. Assuming that the speed during transcription
elongation represents an upper limit of what may be achieved for
promoter-bound RNAP, there is little room for individual steps at a speed much
lower than 10 nt/s, since this would have to be compensated for by other
steps where the speed is much higher than 10 nt/s in order to maintain the
average value. Therefore, if there are any individual NACs during initial
transcription that are slow, for example from slow translocation during
displacement of $\sigma_{3,2}$, they are not likely to be much slower than the
average. Our results therefore support a model where there is no strong
bottleneck step for the NAC during initial transcription, implying that the
strain of the growing DNA bubble and steric clashes with $\sigma$ do not
greatly affect the NAC. Instead, we propose that transcription during
initiation to proceeds similarly as for elongation except for a higher
probability to backtrack.

% Acknowledge the study's limitation
A limitation of the current study is that it does not capture the likely
variation in the rate constant of unscrunching and abortive RNA release from
ITCs with RNAs of varying lengths. It is known that GreB cannot
contribute to the rescue of backtracked complexes with RNAs shorter than 5 nt
\cite{hsu_initial_2006}, which implies an RNA-length-dependence in the rate of
unscrunching and abortive RNA release. Work on elongation complexes have shown
that a weakened RNA-DNA hybrid prevents backtracking
\cite{nudler_rnadna_1997}, making it likely that unscrunching takes longer for
a complete compared to a partially formed RNA-DNA hybrid. At the same time,
the rate constant of 1.4 s$^{-1}$ identified here just above the ~1 s$^{-1}$
temporal resolution of the experiment of Revyakin et al., where this reaction
step could not be discerned \cite{revyakin_abortive_2006}, implying that the
value of 1.4 s$^{-1}$ represents a lower boundary. To resolve this number more
precisely, experimental techniques are needed that can resolve backtracking
and abortive RNA release at the time scale in which they occur. Alternatively,
as increasingly complex molecular simulations of RNAP dynamics are obtained
\cite{silva_millisecond_2014}, the details of the rapid kinetics of abortive
cycling may be instead first be obtained using numerical methods alone.

A key assumption in our model is that the APs calculated from the abundance of
aborted RNA indicate the probability of the initial backtracking step. There
remains however the possibility that the actual probability to backtrack is
higher than what is observed from the APs. This would be the case if ITCs
can remain in long backtracked pauses, as has been observed for elongating
complexes \cite{shaevitz_backtracking_2003}. Revyakin et al.\
included GreB in their experiments specifically to avoid the possibiltiy of
long backtracked pauses \cite{revyakin_abortive_2006}. Therefore, it is
likely that the rate constants estimated from their data are unaffected by this
phenomenon. Since the rate constants also led to good fits in combination with
APs calculated from -GreB experiments (\FIG~\ref{fig:vo_comparison}) we
consider the APs to give a sufficient description of backtracking probability
during initial transcription.

In conclusion, this work supports a model of initial transcription where the
forward rate of transcription is highly similar to the rate for transcription
elongation, and where the backtracking and abortive release step is more than
5 times slower. Since model fitting resulted in physiologically reasonable
rate constants, and since these rate constants hold up to independent kinetic
data (\FIG~\ref{fig:vo_comparison}), this supports that the assumptions
underlying the model are sound. Therefore, it is likely that the APs reflect
probabilities to backtrack during initial transcription. This supports using
calculated APs, and not just bulk quantification, as a measure when
investigating the mechanisms behind the large variation in abortive RNA
synthesis from one template position to the next \cite{hsu_initial_2006}.
