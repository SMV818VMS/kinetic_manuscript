%\addbibresource{/home/jorgsk/Dropbox/phdproject/bibtex/jorgsk.bib}
In this paper we have built and fitted a model of initial transcription to a
combination of bulk and single-molecule experimental data. This allowed us
to pin down the essentaial rate constants of the process, showing that the
speed of transcription highly similar for initial transcription and
transcription elongation (Figure~\ref{fig:revyakin_fit},
Table~\ref{tab:param_fit_revyakin}). We validated the rate constants by
comparing with published kinetic data of initial transcription obtained under
experimental conditions different than for how the rate constants were
obtained (Figure~\ref{fig:vo_comparison}), and we showed that the underlying
assumption of a common abortive profile between single-molecule and bulk
experiments is likely correct, as departure from this profile led to
unphysiological rate constants and worsening of model performance
(Figure~\ref{fig:ap_adjustment}).

The model of initial transcription used in this work uses the same rate
constant for NAC and unscrunching and abortive release at each template
position (Figure~\ref{fig:model_and_rates}). Therefore, the rate constants
presented here must be considered average values. Is this a reasonable
assumption? For NAC, a kinetic study of initial transcription with the T7 RNAP
showed large variation for this value at each translocation step
\cite{tang_real-time_2009}. Here, we found that the average value was similar
to the average rate of transcription elongation for the same concentration of
NTP. Since there is no obvious mechanism that would increase transcription
rate during initial transcription, we find it likely that the variability in
speed of transcription is also similar between initiation and elongation. This
then indicates that there is no bottleneck step during forward initial
transcription, but that it proceeds as for transcription elongation except
that there is a higher probability to backtrack. On the other hand, we expect
a more position-based variability in the rate constant for unscrunching and
abortive RNA release. Presumably, backtracking happens more readily and
abortive release is more rapid for an incomplete RNA-DNA hybrid
\cite{nudler_rnadna_1997,komissarova_shortening_2002}, making this step slower
for complexes with a nascent RNA longer than 10 nt. However, it can also be
speculated that the increased stability from a complete hybrid is matched by
increased instability caused by the increasingly enlarged DNA bubble.
Ultimately, to resolve this number, experimental techniques are needed that
can resolve backtracking and abortive RNA release at the time scale where they
occur. Alternatively, as increasingly complex molecular simulations of RNAP
dynamics are obtained \cite{silva_millisecond_2014}, the details of this
process may be obtained using numerical methods alone.

Our model assumes that APs calculated from the abundance of aborted RNA are
equal to the probability of the initial backstep
(Figure~\ref{fig:model_and_rates}). Can we assess the quality of this
assumption based on model performance? The alternative would be that some
complexes remain in long backtracked pauses, and only slowly or never
abortively release RNA. If this were the case, then the probability to
backtrack would be higher than what is indicated by the AP, since the AP would
only reflect those backtracked complexes that do abort. However, our results
indicate that this is not the case. We showed that if APs are increased, the
reaction rate constants required to fit experimental data become
unphysiological (Figure~\ref{fig:ap_adjustment}). Therefore, we find that a
model of initial transcription where the calculated APs represent the
probability to backtrack is consistent with experimental data.

In conclusion, this work supports a model of initial transcription where the
forward rate of transcription is the same as for transcription elongation, and
where the backtracking and abortive release step is more than 5 times slower.
Since model fitting resulted in physiologically reasonable rate constants
(Table~\ref{tab:param_fit_revyakin}), and since these rate constants hold up
to independent kinetic data (Figure~\ref{fig:vo_comparison}), we propose that
assumptions underlying the model are sound. Therefore, we propose that the APs
may be interpreted as probabilities to backtrack during initial transcription,
and our findings suggest that these probabilities are the same for the
single-molecule experiments of Revyakin et.\ al \cite{revyakin_abortive_2006}
and the steady state transcription experiments by Hsu et.\ al
\cite{hsu_initial_2006}.

% Q:You mentioned something to Itziar, something that you decided not to
% include! Yes, separating the productive from the unproductive transcripts.
% You can do this by analyzing the kinetics from a single round of
% transcription.
