%\addbibresource{/home/jorgsk/Dropbox/phdproject/bibtex/jorgsk.bib}
In this paper we have built and fitted a model of initial transcription to a
combination of bulk and single-molecule data of transcription in the presence
of GreB. This allowed us to pin down the essentaial rate constants of the
process, showing that the speed of initial transcription is the same for
initial transcription as for transcription elongation (Fig1, Table1). We
validated the rate constants by comparing with published kinetic data of
initial transcription without GreB (Fig2), and we showed that the assumption
of a common abortive profile between single-molecule and bulk experiments is
correct, as departure from this profile led to unphysiological rate constants
and worsening of model performance (Fig3).

\subsection{Rate constants of initial transcription}
The model of initial transcription used in this work uses the same rate
constant for NAC and unscrunching and abortive release at each template
position (Fig X). Therefore, the rate constants must be considered average
values. For NAC, a kinetic study of initial transcription with the T7 RNAP
showed large variation for this value at each translocation step
\cite{tang_real-time_2009}. At the same time, the value found in this study
was similar to that of elongation for the same concentration of NTP, which
indicates a low variability for rate of initial transcription for bacterial
RNAP on N25. Instead, we suspect that there might be more variability for the
value for unscrunching and abortive RNA release. Presumably, backtracking
happens more readily and abortive release is more rapid for an incomplete
RNA-DNA hybrid \cite{nudler_rnadna_1997,komissarova_shortening_2002}, making
this step slower for complexes with a nascent RNA longer than 10 nt. However,
it can also be speculated that the increased stability from a complete hybrid
is matched by increased instability caused by the increasingly enlarged DNA
bubble.

We found that the rate constant for promoter escape was not sensitive to
perturbation of APs (Fig 3C). This may be expected since this value represents
only a single step during initial transcription, while the other two rate
constants fitted affect transcription from open complex to promoter escape
(Fig methods).

\subsection{Backtracking during initial transcription}
Our model assumes that AP values calculated from abundance of aborted RNA are
equivalent to the probability of the initial backstep (Fig methods).
Can we assess the quality of this assumption based on model performance? To
begin with, the APs have a slightly different interpretation for experiments
performed with and without GreB. In the absence of GreB, it may be assumed
that backtracked complexes are rarely rescued, given that the intrinsic
endonucleolytic transcript cleavage rate of RNAP is slow
\cite{laptenko_transcript_2003}, such that abortive RNA release is obligate
after the initial backstep. However, in the presence of GreB, backtracked
complexes may be rescued by RNA cleavage; therefore the AP reflects the
probability to both backstep and avoid rescue by GreB. However, by assuming
that GreB-mediated RNA cleavage is much faster than translocation, one may
interpret the +GreB APs as effective probabilities of backtracking.

We may assess the quality of this assumption by considering the alternative,
which is that some complexes persist in extended backtracked pauses, and only
slowly or never release RNA. If this were the case, then the probability to
backtrack would be even higher than what is indicated by the AP, since the AP
would only reflect those backtracked complexes that do not pause. However, we
showed that when fitting the model to artificially increased APs, reaction
rate constants become unphysiological (Figure 3). Based on this, we find that
a model of initial transcription where the calculated APs represent the
probability to backtrack is consistent with experimental data..

In conclusion, this work supports a model of initial transcription where the
forward rate of transcription is the same as for transcription elongation, and
where the backtracking and abortive release step is more than 5 times slower.
Further, from i) the ability of the model to fit rate
constants to physiologically sound values that are consistent with previous
literature (Fig1, Table1), and ii) the validation of those rate constants
against independent kinetic data, we propose that the estimated rate constants
for NAC and backtracking abortive release, as well as the assumptions
underlying the model are sound. Therefore, we propose that the APs may be
interpreted as probabilities to backtrack during initial transcription
(effective probabilities in the case of +GreB), and we propose that these
probabilities are the same for the single-molecule experiments of Revyakin
et. al \cite{revyakin_abortive_2006} and the steady state transcription
experiments by Hsu et. al \cite{hsu_initial_2006}.

% Q:You mentioned something to Itziar, something that you decided not to
% include! Yes, separating the productive from the unproductive transcripts.
% You can do this by analyzing the kinetics from a single round of
% transcription.

