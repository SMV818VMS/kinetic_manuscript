%\addbibresource{/home/jorgsk/Dropbox/phdproject/bibtex/jorgsk.bib}
In this paper we have built a model of initial transcription and fitted rate
constants of key processes using a combination of bulk and single-molecule
experimental data. This allowed us to pin down the essential rate constants of
the process, showing that the speed of transcription highly similar for
initial transcription and transcription elongation
(Figure~\ref{fig:revyakin_fit}, Table~\ref{tab:param_fit_revyakin}). We
validated the rate constants by comparing with published kinetic data of
initial transcription obtained under experimental conditions different than
for how the rate constants were obtained (Figure~\ref{fig:vo_comparison}), and
we showed that the underlying assumption of a common abortive profile between
single-molecule and bulk experiments is likely correct, as departure from this
profile led to unphysiological rate constants and worsening of model
performance (Figure~\ref{fig:ap_adjustment}).

The model of initial transcription used in this work uses the same rate
constant for NAC and unscrunching and abortive release at each template
position (Figure~\ref{fig:model_and_rates}). Therefore, the rate constants
presented here must be considered average values. Given the complex nature of
initial transcription, one may ask if the use of average rate constants is a reasonable
assumption. For T7 RNAP, a kinetic study showed a 10-fold variation for the
NAC rate constant, from 6 nt$^{-s}$ to \cite{tang_real-time_2009}. We would
argue that this is not the case for bacterial RNAP. The average rate constant
we found for NAC matched closely the average speed of transcription during
elongation. Since it is unlikely that the NAC is faster for some steps during
initiation, the closely matching averages require that the NAC cannot be much
slower either. Our results therefore support at model where there is no
bottleneck step during for the NAC during initial transcription, but that
transcription during initiation is the same as for elongation, except that
there is a higher probability to backtrack. This implies that the strain of
the growing DNA bubble and the steric clashes with $\sigma$ do not greatly
affect the NAC during initiation. On the other hand, it is likely that these
processes contribute to the increased rate of backtracking during initiation
compared to those found for elongation.

% XXX new section:
% Add something about the value of NAC, talk about the distribution (it's
% quite wide). Mention the issue of supercoiling, and the case of linear
% fragments. Talk about distributions in general: Abortive and backtracking
% has a very narrow distribution around 1-2, while NAC is more wide, but with
% a clear peak for the highest values around 11.4 whatev. Limitation in
% resolving this resolution is partially in the low number of single
% nucleotide experiments, 100, for which there is plenty of variation in the
% speed of NAC, at lest for elongation.

Our model assumes that APs calculated from the abundance of aborted RNA are
equal to the probability of the initial backstep
(Figure~\ref{fig:model_and_rates}). The alternative would be that some
complexes, especially in the absence of GreB, remain in long backtracked
pauses, as has been observed for transcription elongation
\cite{shaevitz_backtracking_2003}. If long backtracked pauses were common
during initiation, the APs would only reflect those backtracked
complexes that abort, so that that the actual probability to backtrack would
be higher than what is indicated by the APs. However, we showed that rate
constants obtained by fitting a model to +GreB APs are descriptive for the
kinetics of initial transcription without GreB
(Figure~\ref{fig:vo_comparison}), where such long backtracked pauses may take
place. This result indicates that such long backtracked pauses, if they exist,
do not make a large contribution to the kinetics of initial transcription.

The rate constant for unscrunching and abortive RNA release was found to have
a narrow optimum at 1.7 $\pm 0.4$. We speculate that this value may vary for
different template positions. Presumably, backtracking happens more readily
and abortive release is more rapid for an incomplete RNA-DNA hybrid
\cite{nudler_rnadna_1997,komissarova_shortening_2002}. This would imply that
this step is slower for complexes that have obtained with a full-length
RNA-DNA hybrid. For N25, this may not be so relevant, since promoter escape
happens at +11, shortly after a full hybrid has been obtained. To resolve
this number more precisely, experimental techniques are needed that can
resolve backtracking and abortive RNA release at the time scale in which they
occur. Alternatively, as increasingly complex molecular simulations of RNAP
dynamics are obtained \cite{silva_millisecond_2014}, the details of this
process may be instead first be obtained using numerical methods alone.

In conclusion, this work supports a model of initial transcription where the
forward rate of transcription is highly similar to the rate for transcription
elongation, and where the backtracking and abortive release step is more than
5 times slower. Since model fitting resulted in physiologically reasonable
rate constants (Table~\ref{tab:param_fit_revyakin}), and since these rate
constants hold up to independent kinetic data
(Figure~\ref{fig:vo_comparison}), this supports that the assumptions
underlying the model are sound. Therefore, it is likely that the APs
reflect probabilities to backtrack during initial transcription, and it is
probable that the APs are the same for the single-molecule experiments of
Revyakin et.\ al \cite{revyakin_abortive_2006} and the steady state
transcription experiments by Hsu et.\ al \cite{hsu_initial_2006}.
 
% Q:You mentioned something to Itziar, something that you decided not to
% include! Yes, separating the productive from the unproductive transcripts.
% You can do this by analyzing the kinetics from a single round of
% transcription.
