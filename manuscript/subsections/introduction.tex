%\addbibresource{/home/jorgsk/Dropbox/phdproject/bibtex/jorgsk.bib}
Transcription of DNA by RNAP is at the core of cellular metabolism and is a
highly regulated process, especially at the level of initiation. In bacteria,
the initiation of transcription begins when RNAP, associated with a $\sigma$
factor, identifies and binds to a promoter region of DNA, where it forms a DNA
bubble and an RNAP-DNA open complex \cite{saecker_mechanism_2011}. During the
first steps of transcription, RNAP is still bound to the promoter and
therefore translocates by the mechanism of scrunching, where DNA is pulled
into RNAP to expose the enzyme's active site for NTP binding
\cite{revyakin_abortive_2006, kapanidis_initial_2006}. As the DNA bubble grows
by one base pair for each synthesized NTP, strain is formed in the initiating
complex as RNAP has to accommodate the resulting DNA bulges
\cite{straney_stressed_1987, kapanidis_initial_2006,
winkelman_crosslink_2015}. This strain can be released either through promoter
escape, which occurs when the nascent RNA has
reached 8-15 nt in length, \cite{carpousis_cycling_1980,
hsu_vitro_2003, tang_real-time_2009, hsu_initial_2006}, or through abortive
release of the RNA. Abortive RNA release is thought to follow after RNAP has
backtracked, a reverse motion where RNAP releases scrunched
DNA (also known as unscrunching) \cite{hsu_escherichia_1995,
feng_grea-induced_1994, hsu_initial_2006}. After abortive RNA release, RNAP
falls back to the open complex formation where transcription may be
re-initiated \cite{carpousis_cycling_1980}. The extent of repeated RNA
synthesis and abortive release, together known as abortive cycling, varies
depending on the sequence context of both the core promoter region and the
initial transcribed sequence \cite{hsu_initial_2006, hsu_promoter_2002,
vo_vitro_2003, skancke_sequence-dependent_2015}. Regardless of sequence
context, the extent of abortive RNA release tends to be reduced in the
presence of GreB, a protein factor which stimulates RNA cleavage in
backtracked complexes \cite{hsu_initial_2006}.

While much has been discovered about the individual reaction steps of initial
transcription in bacteria, a detailed description of the kinetics of the
process is lacking. An interesting aspect of the kinetics of initial
transcription is that transcription is performed by an RNAP that is tethered to
DNA and is strained from accommodating an enlarged and growing DNA bubble.
This is in contrast to elongation, where RNAP transcribes with a fixed DNA
bubble, moving along a DNA template. Little is known about how the kinetics of
transcription differs between promoter-bound and promoter-free RNAP. Early
work on initial transcription kinetics identified abortive cycling as the rate
limiting step for full length RNA production on short templates
\cite{stefano_lac_1979, munson_abortive_1981}. Recent knowledge has come
largely from single-molecule experiments \cite{revyakin_abortive_2006,
kapanidis_initial_2006, tang_real-time_2009, kapanidis_retention_2005,
margeat_direct_2006}. Due to its rapid kinetics, it has not been possible to
measure directly the speed of initial transcription, or the steps of
unscrunching and abortive RNA release \cite{revyakin_abortive_2006,
margeat_direct_2006}. This is in contrast to transcription elongation, where
the speed of transcription is more readily quantified \cite{wang_force_1998,
tolic-norrelykke_diversity_2004, bai_mechanochemical_2007}. On a qualitative
level, however, it is known that backtracking and abortive RNA release during
initial transcription are slow steps relative to forward translocation and RNA
synthesis \cite{revyakin_abortive_2006, margeat_direct_2006}. And while the
individual reaction steps have been too fast to measure, Revyakin et al.\ has
quantified the total time RNAP spends scrunching before reaching promoter
escape \cite{revyakin_abortive_2006}; on the N25 promoter, RNAP was measured
to spend between 2.5 and 20 seconds abortively cycling over 11 bp of DNA before
achieving promoter escape \cite{revyakin_abortive_2006}. Abortive cycles, or
single scrunch-events that lead directly to promoter escape, shorter than 2.5
seconds were not measured, but were proposed based on extrapolation of data to
follow an exponential fit \cite{revyakin_abortive_2006}.

No kinetic models of bacterial RNAP have been published that use experimental
data to estimate rate constants during initial transcription. Tang et al.
have estimated rate constants ranging from 6 s$^{-1}$ to 60 s$^{-1}$ for
initial transcription by T7 RNAP \cite{tang_real-time_2009}. Since T7 RNAP is
structurally different from bacterial RNAP, it is not clear how this
corresponds to transcription initiation in bacteria. Xue et al. published a
kinetic model for initial transcription for bacterial RNAP, but did not fit
the model's parameters to experimental data, instead relying on model
parameters obtained from transcription elongation data
\cite{xue_kinetic_2008}. This highlights the need for a model that can
identify the rate constants of initial transcription in bacterial RNAP based
on relevant experimental data.

While single-molecule experiments have revealed highly detailed information
about the movement and position of single RNAPs during initial transcription,
traditional bulk experiments are still used to quantify the relative molar
abundance of abortive transcripts. Knowing the relative abundance of each
abortive RNA species, one may calculate probabilities of producing an
abortive transcript of a given length, known as the abortive probability
(AP) \cite{hsu_promoter_2002, hsu_quantitative_1996}. Since abortive RNA
release is thought to follow from backtracking, the APs provide a
position-specific indication of the likelihood to backtrack during initial
transcription. In the present study, we have used a kinetic model of initial
transcription by \textit{E.\ coli} RNAP to find the rate constants of initial
transcription using a novel method that combines the APs obtained from steady
state bulk experiments with the time-distribution of abortive cycling from
single-molecule experiments. With this method, we inferred the speed of
initial transcription (i.e.\ the rate constant of the nucleotide addition
cycle (NAC)) and the rate constant for unscrunching and abortive RNA release.
Our results indicate two important notions: 1) the average speed transcription
is similar for promoter-bound RNAP and elongating RNAP following promoter
escape, and 2) the rate of the NAC is 10 times faster than unscrunching and
abortive RNA release, in agreement previous experimental evidence. We
additionally predict a non-exponential shape of the distribution of abortive
cycles lasting shorter than 2.5 seconds which accounts for the $\sim1$ second
required by RNAP to synthesize an RNA of sufficient length to undergo promoter
escape.
