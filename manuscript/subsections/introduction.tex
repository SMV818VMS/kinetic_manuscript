%\addbibresource{/home/jorgsk/Dropbox/phdproject/bibtex/jorgsk.bib}
Transcription of DNA by RNAP is at the core of cellular metabolism and is a
highly regulated process, especially at the level of initiation. In bacteria,
the initiation of transcription begins when RNAP, associated with a $\sigma$
factor, identifies and binds to a promoter region of DNA, where it forms a DNA
bubble and an RNAP-DNA open complex \cite{saecker_mechanism_2011}. During the
first steps of transcription, RNAP is still bound to the promoter and
therefore translocates by the mechanism of scrunching, where DNA is pulled
into RNAP to expose the enzyme's active site for NTP binding
\cite{revyakin_abortive_2006, kapanidis_initial_2006}. As the DNA bubble grows
by one base pair for each synthesized NTP, strain is formed in the initiating
complex as RNAP has to accommodate the resulting DNA bulges
\cite{straney_stressed_1987, kapanidis_initial_2006,
winkelman_crosslink_2015}. This strain can be released either through promoter
escape, which occurs when the nascent RNA has
reached 8-15 nt in length, \cite{carpousis_cycling_1980,
hsu_vitro_2003, tang_real-time_2009, hsu_initial_2006}, or through abortive
release of the RNA. Abortive RNA release is thought to follow after RNAP has
backtracked, a reverse motion where RNAP releases scrunched
DNA (also known as unscrunching) \cite{hsu_escherichia_1995,
feng_grea-induced_1994, hsu_initial_2006}. After abortive RNA release, RNAP
falls back to the open complex formation where transcription may be
re-initiated \cite{carpousis_cycling_1980}. The extent of repeated RNA
synthesis and abortive release, together known as abortive cycling, varies
depending on the sequence context of both the core promoter region and the
initial transcribed sequence \cite{hsu_initial_2006, hsu_promoter_2002,
vo_vitro_2003, skancke_sequence-dependent_2015}. Regardless of sequence
context, the extent of abortive RNA release tends to be reduced in the
presence of GreB, a protein factor which stimulates RNA cleavage in
backtracked complexes \cite{hsu_initial_2006}.

While much has been discovered about the individual reaction steps of initial
transcription in bacteria, a detailed description of the kinetics of the
process is lacking. An interesting aspect of the kinetics of initial
transcription is that transcription is performed by an RNAP that is tethered to
DNA and is strained from accommodating an enlarged and growing DNA bubble.
This is in contrast to elongation, where RNAP transcribes with a fixed DNA
bubble, moving along a DNA template. Little is known about how the kinetics of
transcription differs between promoter-bound and promoter-free RNAP. Early
work on initial transcription kinetics identified abortive cycling as the rate
limiting step for full length RNA production on short templates
\cite{stefano_lac_1979, munson_abortive_1981}. Recent knowledge has come
largely from single-molecule experiments \cite{revyakin_abortive_2006,
kapanidis_initial_2006, tang_real-time_2009, kapanidis_retention_2005,
margeat_direct_2006}. Due to its rapid kinetics, it has not been possible to
measure directly the speed of initial transcription, or the steps of
unscrunching and abortive RNA release \cite{revyakin_abortive_2006,
margeat_direct_2006}. This is in contrast to transcription elongation, where
the speed of transcription is easier to quantify \cite{wang_force_1998,
tolic-norrelykke_diversity_2004, bai_mechanochemical_2007}. On a qualitative
level, however, it is known that backtracking and abortive RNA release is slow
relative to forward translocation and RNA synthesis
\cite{revyakin_abortive_2006, margeat_direct_2006}. And while the individual
reaction steps have been too fast to measure, Revyakin et al.\ has quantified
the total time RNAP spends in abortive cycling before reaching promoter escape
\cite{revyakin_abortive_2006}; on the N25 promoter, RNAP spends between 3.5
and 20 seconds abortively cycling over 11 bp of DNA before achieving promoter
escape \cite{revyakin_abortive_2006}. Abortive cycles lasting shorter than 3.5
seconds have not been measured, but Revyakin et al.\ proposed by extrapolation
of their data that the frequency of abortive cycles lasting between 3.5 and 0
seconds follows an exponential function \cite{revyakin_abortive_2006}.  

While single-molecule experiments have revealed highly detailed information
about the movement and position of single RNAPs during initial transcription,
traditional bulk experiments are still used to quantify the relative molar
abundance of abortive transcripts. Knowing the relative abundance of each
abortive RNA species, one may calculate probabilities of producing an
abortive transcript of a given length, known as the abortive probability
(AP) \cite{hsu_promoter_2002, hsu_quantitative_1996}. Since abortive RNA
release is thought to follow from backtracking, the APs provide a
position-specific indication of the likelihood to backtrack during initial
transcription. In the present study, we have used a kinetic model of
initial transcription to find the rate constants of initial transcription
using a novel method that combines the APs obtained from steady state bulk
experiments with the time-distribution of abortive cycling from
single-molecule experiments. With this method, we inferred the speed of
initial transcription (i.e.\ the rate constant of the nucleotide addition
cycle (NAC)) and the rate constant for unscrunching and abortive RNA release.
Our results indicate two important notions: 1) the speed transcription is the
same for promoter-bound RNAP as for promoter-free RNAP following promoter
escape, and 2) the rate of the NAC is 10 times faster than unscrunching and
abortive RNA release, in agreement the results from previous studies. We
additionally predict a non-exponential shape of the distribution of abortive
cycles lasting shorter than 3.5 seconds which contains a 1 second lag to
account for the time needed to synthesize an RNA of sufficient length to reach
promoter escape.
