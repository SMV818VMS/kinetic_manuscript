%\addbibresource{/home/jorgsk/Dropbox/phdproject/bibtex/jorgsk.bib}
Transcription of DNA is at the core of cellular metabolism and a highly
regulated process at multiple levels. In bacteria, transcription is performed
by a single RNAP, which when associated with a $\sigma$ factor can identify
and bind to promoter regions of DNA, forming a DNA bubble and the
RNAP-DNA open complex. RNAP is still
bound to the promoter during the first steps of RNA synthesis and its translocation
occurs by the mechanism of scrunching \cite{revyakin_abortive_2006,
kapanidis_initial_2006}, where DNA is pulled into RNAP to expose the enzyme's
active site for NTP binding. This causes the DNA bubble to grow by one base
pair for each NTP added to the RNA's growing 3\ppp end, until it synthesizes an RNA of 8-15 nt in
length,
\cite{carpousis_cycling_1980,hsu_vitro_2003,tang_real-time_2009,hsu_initial_2006},
where it may break promoter bonds,  entering the phase of productive transcription
elongation. However, not all initial transcription attempts result in promoter
escape. At any position before escape is reached, RNAP may instead release the
nascent RNA and fall back to the open complex formation, from where RNAP may
initiate transcription once more \cite{carpousis_cycling_1980}. The process of
repeated RNA synthesis and abortive release is known as abortive cycling. It
has been shown that the extent of abortive cycling depends on the sequence
context of both the core promoter region and the initial transcribed
sequence (ITS) \cite{hsu_initial_2006, hsu_promoter_2002, vo_vitro_2003}.

The complete picture of the mechanisms that are involved in these first phases of the transcription initiation is not fully understood. More specifically,   
the reaction steps that precede abortive RNA release, and their kinetics have not been 
determined yet. For instance, we do not know whether the speed of transcription differs between the promotor bound phase (the scrunching phase) and the elongating RNAP (following the promoter escape). To fully understand transcription regulation, we need to further explore the dynamics of this initial and crucial transcription phase. 

 It is known so far that some or all abortive RNA releases
are associated with the step of backtracking. This step constitutes a reverse motion
where RNAP releases scrunched DNA, forcing the nascent RNA 3\ppp end to
extrude into the NTP entry channel. Backtracking is strongly associated with abortive RNA release, evindently since the amount of abortive product is
reduced in the presence of GreB, a protein that rescues backtracked complexes
by stimulating RNAP's intrinsic RNA cleavage activity
\cite{hsu_initial_2006,hsu_escherichia_1995,feng_grea-induced_1994}.

The current conceptual model of abortive cycling therefore consists of the
progressive steps of scrunching and nucleotide addition (together referred to
as the nucleotide addition cycle, NAC), and the regressive steps of
backtracking, unscrunching, and abortive RNA release. In this model, little is
known about the kinetics of the individual steps. Early work with bulk
transcription experiments on the \textit{lac} UV5 promoter showed a half-life
of full length RNA synthesis on short DNA templates of 1 minute, and that
promoter-binding was not the rate-limiting step \cite{stefano_lac_1979}. Given
the short length of the templates, this implied a rate-limiting step between
promoter binding and promoter escape. It was soon after shown that the
time-delay for full length RNA synthesis coincided with the synthesis of large
quantities of short abortive transcripts, which implied abortive cycling as
the rate limiting step \cite{munson_abortive_1981}. Recent knowledge about the
kinetics of initial transcription has come largely from single-molecule
experiments \cite{revyakin_abortive_2006, kapanidis_initial_2006,
tang_real-time_2009, kapanidis_retention_2005, margeat_direct_2006}. For
bacterial RNAP, it has been found that the backtracking and abortive RNA
release steps are slow relative to forward translocation and RNA synthesis
\cite{revyakin_abortive_2006, margeat_direct_2006}, but so far the exact
kinetic difference of these steps has not been quantified. The most detailed
description of the kinetics of initial transcription in bacteria is the
characterization of the time RNAP spends in abortive cycling on the N25
promoter in the presence of GreB \cite{revyakin_abortive_2006}. This work
showed that RNAP spends between 3 and 20 seconds in abortive cycling over a
stretch of 11 bp of DNA before achieving promoter escape
\cite{revyakin_abortive_2006}.

By singling out (YOU MEAN ISOLATING?) the rate limiting steps and finding the distribution of time
spent in abortive cycling, single-molecule experiments have revealed a general
kinetic picture of initial transcription. However, their time-resolution has
not yet been sufficient for distinguishing the rate constants for the
individual steps of backtracking, unscrunching, and abortive RNA release. Moreover,
while these methods have been used to quantify the speed of transcription for
transcription elongation \cite{wang_force_1998,
tolic-norrelykke_diversity_2004}, their relatively low temporal (ARE YOU SURE TEMPORAL? TIME DYNAMIC RESOLUTION?) resolution have yet not
allowed the identification of the speed of the NAC during initial
transcription \cite{revyakin_abortive_2006, margeat_direct_2006}. As a
consequence, the speed of the promoter-bound stages (abortive cycling) cannot be compared to the freely elongating RNAPs. (MAYBE THE LAST SENTENCE REDUNDANT, ALREADY MENTIONED ABOVE).  

altough single-molecule experiments can reveal highly detailed information about
the movement and position of single RNAPs during initial transcription,
traditional bulk experiments must still be used to quantify the relative molar
abundance of abortive transcripts of different lengths. By knowing the
relative abundance of each abortive RNA species, one may calculate the
probability of producing an abortive transcript of any given length, known as
the abortive probability (AP) \cite{hsu_promoter_2002, hsu_quantitative_1996}.
Since backtracking is a starting point for the abortive RNA releases,
bulk transcription experiments therefore may provide a position-specific indication
of the likelihood of backtracking during initial transcription.

In the present study, we have created a kinetic model of initial transcription
that combines the time-distribution of abortive cycling from single-molecule
experiments and abortive probabilities obtained from steady state bulk
experiments. We inferred from this model the rate constants for the NAC
and the combined step of unscrunching and abortive RNA release. Our results
indicate two important notions: 1) the rate of the NAC is the same for RNAP promoter bound stage as
for the elongation following promoter escape, and 2) the rate of NAC is 10 times faster than the unscrunching and abortive RNA release, suggesting that these latter steps are rate limiting in the process of transcription initiation, and may involve a layer of regulation of the process.  
