%\addbibresource{/home/jorgsk/Dropbox/phdproject/bibtex/jorgsk.bib}
Gene expression is at the core of cellular metabolism and is a highly regulated
process at multiple levels. The first steps in gene expression in bacteria are
marked by the binding of RNAP to a promoter, formation of the open complex,
and the synthesis of nascent RNA. During initial RNA transcription, RNAP
remains bound to the promoter and can therefore not move downstream the template as
for transcription elongation. Instead, translocation during
initial transcription occurs by the mechanism of scrunching
\cite{revyakin_abortive_2006, kapanidis_initial_2006}, where the DNA bubble is
enlarged as downstream DNA is pulled into RNAP to expose the active site for
NTP binding. After having gone through several cycles of scrunching to
synthesize an RNA of 8-15 nt in length,
\cite{carpousis_cycling_1980,hsu_vitro_2003,tang_real-time_2009,hsu_initial_2006},
RNAP may proceed to escape promoter bounds and enter the phase of productive
transcription elongation. However, promoter escape is not the only pathway
available. Before reaching escape, RNAP may instead release the
nascent RNA and fall back to the open complex formation, from where initial
transcription may be initiated once more \cite{carpousis_cycling_1980}. The
process of repeated RNA synthesis and abortive release is known as abortive
cycling. It has been shown that the extent of abortive cycling varies
considerably between different promoters, and that abortive transcript
synthesis is sensitive to mutations both in the core promoter region and in
the initial transcribed sequence (ITS) \cite{hsu_initial_2006,
hsu_promoter_2002, vo_vitro_2003}.

The reaction steps that preceed abortive RNA release have not been determined.
What is known is that some or all abortive RNA releases involve the step of
backtracking. Backtracking constitutes a reverse motion where RNAP releases
scrunched DNA, forcing the nascent RNA 3\ppp end to extrude into the NTP entry
channel. The evidence for backtracking's involvement in abortive RNA release
is that the amount of abortive product is reduced in the presence of GreB, a
protein that rescues backtracked complexes by stimulating RNAP's intrinsic RNA
cleavage activity
\cite{hsu_initial_2006,hsu_escherichia_1995,feng_grea-induced_1994}.
Therefore, although uncertainty remains about the details, the conceptual
model of abortive cycling consists of the progressive steps of scrunching and
nucleotide addition (together referred to as the nucleotide addition cycle,
NAC), and the regressive steps of backtracking, unscrunching, and abortive RNA
release. In this model, little is known about the kinetics of the individual
steps.

Early work with bulk transcription experiments on the \textit{lac} UV5
promoter showed a half-life of full length RNA synthesis on short DNA
templates of 1 minute, and that promoter-binding was not the rate-limiting
step \cite{stefano_lac_1979}. Given the short length of the templates, this
implied a rate-limiting step between promoter binding and promoter escape. It
was soon after shown that the time-delay for full length RNA synthesis
coincided with a high rate of synthesis of short abortive transcripts, which
implied abortive cycling as the rate limiting step
\cite{munson_abortive_1981}. Recent kinetic information abount initial
transcription has come largely from single-molecule experiments
\cite{revyakin_abortive_2006, kapanidis_initial_2006, tang_real-time_2009,
kapanidis_retention_2005, margeat_direct_2006}. For bacterial RNAP, it has
been found that the backtracking and abortive RNA release steps are slow
relative to forward translocation and RNA synthesis
\cite{revyakin_abortive_2006, margeat_direct_2006}. The most detailed view of
the kinetics of initial transcription in bacteria so far is given by
characterization of the distributions of time RNAP spends in abortive cycling
on the N25 promoter in the presence of GreB \cite{revyakin_abortive_2006}.
These distributions show RNAP spending between 3 and 20 seconds in abortive
cycling before achieving promoter escape \cite{revyakin_abortive_2006}.

By identifying the rate limiting steps and the distribution of time spent in
abortive cycling, single-molecule experiments have revealed the general
kinetic picture of initial transcription. However, their time-resolution has
not been sufficient for separating the individual steps of backtracking,
unscrunching, and abortive RNA release; neither have they been able to
characterize the NAC for promoter bound RNAP \cite{revyakin_abortive_2006,
margeat_direct_2006}. This means that the question of how translocation and
nucleotide addition are affected by promoter-bound RNAP remains unanswered.
It is not known if these reactions have different rates during initiation than
for elongation, caused by the different conformation of the enzyme and the
strain of an enlarged DNA bubble.

For investigating initial transcription, traditional bulk experiments retain
an advantage over their single-molecule counterparts in that they provide
information the relative molar abundance of abortive transcripts of different
lengths \cite{hsu_monitoring_2009}. This allows the determination of at which
positions RNAP is more likely to disengage to produce an abortive RNA. By
quantifying the relative abundance of each abortive RNA species, one may
calculate the probability of producing an abortive transcript of a given
length, known as the abortive probability (AP) \cite{hsu_promoter_2002,
hsu_quantitative_1996}. Since backtracking is a starting point for some or
all abortive RNA releases, bulk transcription experiments therefore give a
posision-specific indication of the likelihood of backtracking during initial
transcription.

In this work, we have created a kinetic model of initial transcription that
combines the time-distribution of abortive cycling from single-molecule
experiments and abortive probabilities obtained from steady state bulk
experiments. Applying this model, we obtained the rate constants for the NAC
and the combined step of unscrunching and abortive RNA release. Our results
show show that the rate of the NAC is the same for initial transription as for
elongation elongation, and nearly 10 times faster than the rate-limiting step
of unscrunching and abortive RNA release.

% Now you want to say something about the abortive probability. What is the
% link to kinetics? The link is the frequency of backtracking events.

%For the N25 promoter in the presence of GreB, Revaykin et.\ al determined the
%speed of transcription after promoter escape to be approximately 10 nt
%s$^{-1}$, but could not resolve this number wile RNAP was engaged in abortive
%cycling \cite{revyakin_abortive_2006}. Thus, the rate constants for these
%processes before promoter escape is achieved are at present unknown.

%GreB does not however decrease abortive product synthesis for transcripts
%shorter than 4-5 nt \cite{hsu_initial_2006}, implying that complexes with
%such short RNA-DNA hybrids may dissociate from DNA without backtracking, or
%that GreB cannot rescue short nascent RNA efficiently.  Backtracking during
%initial transcription would cause an unscrunching of the DNA bubble and a
%displacement of the 3\ppp end of the nascent RNA into the NTP entry channel,
%shortening the RNA-DNA hybrid for nascent RNAs shorter than 10 nts. Since
%elongation complexes with a shortened hybrid have been shown to be unstable
%\cite{nudler_rnadna_1997}, the shortened hybrid remains a candidate
%explanation for abortive RNA release resulting from backtracking.

%At the same time, the initially short RNA-DNA hybrid is likely not what
%causes backtracking in the first place. This follows as the amount of
%abortive transcript is not strongly associated with the length of the nascent
%RNA-DNA hybrid \cite{hsu_initial_2006}, and neither is there any association
%between free energy of the RNA-DNA hybrid and promoter productive yield
%\cite{skancke_sequence-dependent_2015}.

%Experiments of initial transcription have shown that the molar amount of
%abortive transcripts often largely exceeds that of productive full length
%transcripts \cite{carpousis_cycling_1980}. For the N25 promoter, a 20-fold
%exess of abortive to full length transcript was found
%\cite{hsu_initial_2006}.  If backtracking were implied in all aborted RNAs,
%and if backtracking were equally likely at each template position, this would
%correspond to a frequency of 1 backtracking event for every 4 nt of
%synthesized RNA for initial transcription on N25, which achieves promoter
%escape after synthesizing a 11 nt RNA. In comparison, the frequency during
%elongation is in the order of 10$^{-3}$ bp$^{-1}$
%\cite{shaevitz_backtracking_2003}. The probability to backtrack from a given
%template position during initial transcription would be proportional to the
%probability of producing an abortive RNA of the corresponding length.  The
%probability of producing an abortive RNA can be calculated from bulk
%transcription studies where the amount of abortive species of each length is
%quantified \cite{hsu_quantitative_1996}. This shows a complex
%promoter-specific pattern of abortive probabilities (APs)
%\cite{hsu_initial_2006,hsu_vitro_2003}. While the AP pattern is not fully
%understood, it is known that it is highly sequence specific
%\cite{hsu_vitro_2006}, and that some of the sequence-specificity can be
%explained by the propensity of RNAP to reside in the pre-translocated state
%at different template positions~\cite{skancke_sequence-dependent_2015}. APs
%are insensitive to changes in NTP concentration \cite{hsu_vitro_2003}, but
%are reduced in the presence of GreB \cite{hsu_initial_2006}. The reduction of
%AP in the presence of GreB likely reflects the rescuing of backtracked
%complexes by stimulation of RNA cleavage, and not GreB binding in the
%secondary channel to reduce backtracking by steric occlusion
%\cite{opalka_structure_2003, hsu_initial_2006}.
