%\addbibresource{/home/jorgsk/Dropbox/phdproject/bibtex/jorgsk.bib}
Transcription of DNA is at the core of cellular metabolism and is highly
regulated at the level of initiation. In bacteria, transcription initiation
begins when RNAP, associated with a $\sigma$ factor, identifies and binds to a
promoter region of DNA, where it forms a DNA bubble and the RNAP-DNA open
complex \cite{saecker_mechanism_2011}. During the first steps of
transcription, RNAP is still bound to the promoter and therefore translocates
by the mechanism of scrunching, where DNA is pulled into RNAP to expose the
enzyme's active site for NTP binding \cite{revyakin_abortive_2006,
kapanidis_initial_2006, winkelman_crosslink_2015}. This causes the DNA bubble
to grow by one base pair for each NTP added to the RNA's growing 3\ppp end,
until the nascent RNA has grown to 8-15 nt in length,
\cite{carpousis_cycling_1980, hsu_vitro_2003,tang_real-time_2009,
hsu_initial_2006}, at which point RNAP may escape promoter bonds,
entering the phase of processive transcription elongation. However, not all
initial transcription attempts result in promoter escape. At any position
before escape is reached, RNAP may release the nascent RNA and fall back to
the open complex formation, from where transcription may be re-initiated
\cite{carpousis_cycling_1980}. The process of repeated RNA synthesis and
abortive release, known as abortive cycling, is known to vary depending on the
sequence context of both the core promoter region and the initial transcribed
sequence (ITS) \cite{hsu_initial_2006, hsu_promoter_2002, vo_vitro_2003}. The
abortive release of RNA has been associated with the step of backtracking
\cite{hsu_initial_2006}. This step constitutes a reverse motion where RNAP
releases scrunched DNA, forcing the nascent RNA 3\ppp end to extrude into the
NTP entry channel. Backtracking is strongly associated with abortive RNA
release, since the amount of abortive product is reduced in the presence of
GreB, a protein that rescues backtracked complexes by stimulating RNAP's
intrinsic RNA cleavage activity \cite{hsu_initial_2006, hsu_escherichia_1995,
feng_grea-induced_1994}.

The complete picture of the mechanisms involved in initial transcription is
not fully understood. In particular, the reaction steps that precede abortive
RNA release, and their kinetics, have not been determined. One question that
therefore remains unanswered is if the speed of transcription differs between
the promoter bound stage (translocation by scrunching) and the elongating
stage (translocation by base-stepping). It conceivable that structural
constraints while accomodating DNA bulges in the scrunched complex cause
transcription to proceed more slowly compared to the elongating complex.

The current conceptual model of abortive cycling can be divided into the
progressive steps of scrunching and nucleotide addition (together referred to
as the nucleotide addition cycle, NAC), and the regressive steps of
backtracking, unscrunching, and abortive RNA release. The kinetics of these
steps have been investigated since the late 70s. Early kinetic work with
bulk transcription experiments on the \textit{lac} UV5 promoter showed a
half-life of full length RNA synthesis on short DNA templates of 1 minute, and
that promoter-binding was not the rate-limiting step \cite{stefano_lac_1979}.
Given the short length of the templates, this implied a rate-limiting step
between promoter binding and promoter escape. It was soon after shown that the
time-delay for full length RNA synthesis coincided with the synthesis of large
quantities of short abortive transcripts, which implied abortive cycling as
the rate limiting step \cite{munson_abortive_1981}. Recent knowledge about the
kinetics of initial transcription has come largely from single-molecule
experiments \cite{revyakin_abortive_2006, kapanidis_initial_2006,
tang_real-time_2009, kapanidis_retention_2005, margeat_direct_2006}. For
bacterial RNAP, it has been found that the backtracking and abortive RNA
release steps are slow relative to forward translocation and RNA synthesis
\cite{revyakin_abortive_2006, margeat_direct_2006}, but no quantitative
description of this difference has been found. For abortive cycling, it has
been found that RNAP on the N25 promoter spends between 3.5 and 20 seconds in
this step over a stretch of 11 bp of DNA before achieving promoter escape
\cite{revyakin_abortive_2006}. Abortive cycling lasting shorter than 3.5 seconds
have not been measured, but Revyakin et al. proposed by extrapolation of their
data that the frequency of abortive cycles lasting between 3.5 and 0 seconds
follows an exponential decline \cite{revyakin_abortive_2006}.  

By proposing the rate limiting steps and finding the distribution of time
spent in abortive cycling, single-molecule experiments have revealed a general
kinetic picture of initial transcription. However, their time-resolution has
not yet been sufficient for distinguishing the rate constants for the
individual steps of backtracking, unscrunching, and abortive RNA release.
Specifically, although these methods have been used to quantify the speed of
transcription for transcription elongation \cite{wang_force_1998,
tolic-norrelykke_diversity_2004}, their relatively low temporal resolution
have yet not allowed the identification of the speed of the NAC during initial
transcription \cite{revyakin_abortive_2006, margeat_direct_2006}.

While single-molecule experiments can reveal highly detailed information
about the movement and position of single RNAPs during initial transcription,
traditional bulk experiments must still be used to quantify the relative molar
abundance of abortive transcripts. By knowing the relative abundance of each
abortive RNA species, one may calculate the probability of producing an
abortive transcript of any given length, known as the abortive probability
(AP) \cite{hsu_promoter_2002, hsu_quantitative_1996}. Since backtracking is a
starting point for abortive RNA releases, bulk transcription experiments
therefore provide a position-specific indication of the likelihood of
backtracking during initial transcription. In the present study, we have
created a kinetic model of initial transcription to find the rate constants of
initial transcription using a novel method that combines the APs obtained from
steady state bulk experiments with time-distribution of abortive cycling from
single-molecule experiments. With this method, we inferred the rate constants
for the NAC and the combined step of unscrunching and abortive RNA release.
Our results indicate two important notions: 1) the speed of the NAC is the
same for promoter-bound RNAP as for promoter-free RNAP following promoter
escape, and 2) the rate of NAC is 10 times faster than the unscrunching and
abortive RNA release, confirming the results from other studies that have
indicated that these steps are rate limiting for promoter escape. We
additinally predict a non-exponential shape of the distribution of abortive
cycles lasting shorter than 3.5 seconds that includes a ~1 second lag that
accounts for the average time needed to synthesize an RNA sufficiently long to
reach promoter escape.
