%\addbibresource{/home/jorgsk/Dropbox/phdproject/bibtex/jorgsk.bib}
\SUBSECTION{Model of initial transcription}
Our model describes the process of initial transcription by considering the
following reactions: i) the NAC ii) backtracking iii) unscrunching and abortive
RNA release (UAR), and iv) promoter escape (\FIG~\ref{fig:model_and_rates}).
We consider backtracking as the first unscrunching step, also known as
backstepping. To find the rate constant of
backtracking we emply a novel method which makes use of abortive probabilities
(APs) calculated from experimental data \cite{hsu_quantitative_1996}. The
method relies on two key points. The first is the model constraint that the
reactions of the NAC and backtracking are in kinetic competition. The second
is the assumption that the probability to backtrack at any given template
position is equal to the AP at that position. Thus we get
\begin{equation*}
    \frac{b_i}{b_i + \text{NAC}_i} = \text{AP}_i,
\end{equation*}
where subscript $i$ indicates position, and NAC and $b$ are the rate constants
the nucleotide addition cycle and backtracking, respectively. From this
expression, we obtain the rate constant for backtracking given a rate constant
of the NAC and an AP:
\begin{equation}
  b_i = \frac{\text{NAC}_i\cdot\text{AP}_i}{1-\text{AP}_i}.
  \label{eq:backtrackingcalc}
\end{equation}
\FIG~\ref{fig:extrap_and_GreB_minus_fit} shows an example calculation of rate
constants of backtracking using NAC = 10 $s^{-1}$.

In this work we run model simulations alternately using APs obtained both for
the absence of GreB (-GreB) and in the presence of GreB (+GreB). When GreB is
present, we note that backtracked complexes may be rescued by GreB stimulated
cleavage of the unaligned 3\ppp end of the transcript. Therefore, APs obtained
in the presence of GreB represent the probability to both backtrack and to
avoid rescue by GreB until abortive RNA release. We therefore assume that
GreB-mediated cleavage and subsequent NACs are rapid steps compared to
unscrunching. This permits using the APs obtained in the presence of GreB as
effective backtracking probabilities. All AP values used in this work are
obtained from Hsu et al. \cite{hsu_initial_2006}.

\SUBSECTION{Implementation and rate constant estimation}
A central result in this work is determining the model rate constants. We do
this running multiple simulations with different combinations of rate
constants and fitting to exeprimental data. The data used for rate constant
estimation is the distribution of time spent in abortive cycling on the N25
promoter in the presence of $100\ \mu M$ NTP determined by Revyakin et al.\
\cite{revyakin_abortive_2006}. This data was obtained from 100 individual
initial transcription events. For experiments with single molecules, there is
an inherent stochastic component in the experimental outcome that derives from
the randomness of molecular motion. That this is evident for transcription can
be inferred from the large variation observed in single-molecule experiments
of the speed of transcription elongation \cite{adelman_single_2002,
tolic-norrelykke_diversity_2004}. To account for this randomness, we perform
the kinetic simulations using the Gillespie algorithm for stochastic
simulations of chemical reactions \cite{gillespie_exact_1977}. Specifically,
we make use the version implemented by Maarleveld et al.\ in the StocPY
software \cite{maarleveld_stochpy:_2013}.

The proceedure for rate constant estimation (illustrated in
\FIG~\ref{fig:extrap_and_GreB_minus_fit}) is as follows (rate constant names indicated
in \FIG~\ref{fig:model_and_rates}): First, we assign random values to the
three rate constants NAC ($k_n$), promoter escape ($k_e$) and unscrunching and
abortive RNA release ($k_u$) within certain fixed boundaries (see below).
These values are chosen independently from a uniform distribution. Secondly,
we use the rate constant of the NAC and the APs to calculate the rate constant
of backtracking ($k_b$) using Eq.~(\ref{eq:backtrackingcalc}). We then
simulate 100 initial transcription events and calculate the distribution of
time spent in abortive cycling as a result of these specific rate constants.
The result is measured against the empirical distribution
\cite{revyakin_abortive_2006} using the root mean square error. By repeating
this procedure several times, we obtain statistics of which values of rate
constants that are associated with the best match with the experimental data.
By virtue of using stochstic simulations, one does not arrive at a unique set
of rate constants that have optimal fit with data; if a simulation is repeated
without changing the rate constants, slightly different results will be
obtained due to the inherent stochastisity of the reaction process. Therefore,
we here find the optimal rate constants by running multiple simulations and
identifying the rate constants that are most often associated with best fit to
data from the distribution of fitness.

The boundaries of the rate constants for parameter estimation were chosen from
extrema estimated from experimental data. The speed of the NAC during initial
transcription cannot be much less than 3 nt/s, since Revyakin et al.\ measured
3.5 seconds as the shortest duration of abortive cycling
\cite{revyakin_abortive_2006} to cover the ~ 11 nt/s required for promoter
escape on N25. At the same time, it should not be larger than at most 25 nt/s,
which is at the upper limit of measured values from single-molecule
experiments \cite{bai_mechanochemical_2007}. While it is not clear how
transcription would proceed faster for promoter-bound RNAP, we use 25 nt/s as
a maximum value also in order to see if the model is able to discard these
values during rate constant estimation. For backtracking and abortive RNA
release, it is known that the rate constant must be faster than 1 s$^{-1}$,
this was the time-resolution of the experimental equipment and this reaction
step could not be resolved \cite{revyakin_abortive_2006}. We therefore set the
minimum and maximum values of all rate constants to be 1 s$^{-1}$ and 25
s$^{-1}$, respectively. 

\begin{figure}[h]
  \caption{{\bf Kinetic scheme of initial transcription on the N25 promoter.}
    From the open complex (OC) transcription proceeds by NACs from one initial
    transcribing complex (ITC) to the next, where each ITC is identified in
    subscript by the length of its nascent RNA. Initial transcription proceeds
    until the nascent RNA has reached the experimentally obtained maximum size
    of abortive transcript; here 11 nt \cite{hsu_initial_2006}. For ITCs with
    an RNA of 2 nt in length or longer, there is a competition between the the
    NAC and backtracking. Backtracking causes termination of transcription,
    and only further backtracking (unscrunching) and abortive RNA release may
    follow, returning RNAP to the open complex. From the open complex forward
    transcription may resume once more. The names of the rate constants are as
    follows: $k_n$ (NAC), $k_e$ (promoter escape), $k_u$ (unscrunching and
    abortive RNA release) and $k_{b,i}$ (backtracking).}
    \label{fig:model_and_rates}
\end{figure}

\begin{figure}[h]
  \caption{ {\bf Scheme of rate constant estimation protocol.} \textbf{1:}
    Rate constants for the NAC, UAR and promoter escape are randomly sampled
    from a uniform distribution (example values are shown). Of these values,
    the NAC is used further to obtain backtracking rates at each template
    position (\textbf{3}). This is done by solving
    Eq.~(\ref{eq:backtrackingcalc}), where the APs are obtained from Hsu et
    al.\ \cite{hsu_initial_2006} (\textbf{2}). The backtracking rate
    constants, together with the NAC, UAR, and promoter escape rate constants
    sampled in step \textbf{1}, encompass the complete kinetic scheme of
    initial transcription (\textbf{4}). This scheme is then used to simulate
    the kinetics of 100 initial transcription events; from these 100 events
    the distribution of time spent in abortive cycling is calculated and
    compared to measured data from Revyakin et al.\
    \cite{revyakin_abortive_2006} (\textbf{5}). From the distance between the
    measured distribution and the simulated distribution, a fitness score is
    produced. This score is associated to the three randomly selected rate
    constants in step \textbf{1}, and is a measure for how well the kinetic
    scheme (the three random rate constants and the AP values) agree with the
    experimental data. By repeating steps \textbf{1}-\textbf{5} multiple
    times, distributions are obtained from where we find which values of the
    rate constants provide the best fit with experimental data
    (\FIGS~\ref{fig:parameter_estimation_proper} and
    \ref{fig:extrap_and_GreB_minus_fit}). }
\end{figure}
