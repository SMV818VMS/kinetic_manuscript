%\addbibresource{/home/jorgsk/Dropbox/phdproject/bibtex/jorgsk.bib}
\SUBSECTION{Kinetic model of initial transcription}
Our model describes the process of initial transcription by considering the
following reactions: i) the NAC ii) backtracking iii) unscrunching and abortive
RNA release (UAR), and iv) promoter escape (\FIG~\ref{fig:model_and_rates}).
We consider backtracking as the first unscrunching step, also known as
backstepping. We calculate the rate constant of backtracking using a novel
method which makes use of APs calculated from experimental data
\cite{hsu_quantitative_1996}. The method relies on two key points. The first
is the feature of the model that the reactions of the NAC and backtracking are in
kinetic competition (\FIG~\ref{fig:model_and_rates}). The second is the
assumption that the probability to backtrack at any given template position is
equal to the AP at that position. In combination, we get
\begin{equation*}
    \frac{b_i}{b_i + \text{NAC}_i} = \text{AP}_i,
\end{equation*}
where subscript $i$ indicates position, and NAC and $b$ are the rate constants
of the nucleotide addition cycle and backtracking, respectively. From this
expression, we obtain the rate constant for backtracking given a rate constant
of the NAC and an AP:
\begin{equation}
  b_i = \frac{\text{NAC}_i\cdot\text{AP}_i}{1-\text{AP}_i}.
  \label{eq:backtrackingcalc}
\end{equation}
\FIG~\ref{fig:param_estimation_scheme} shows an example calculation of rate
constants of backtracking using NAC = 10 $s^{-1}$.

We present a rationale for why APs calculated from abundance of aborted RNA
give a true indication of the probability of the initial backtracking step.
The alternative would be that RNAP could backtrack without producing an
abortive RNA, i.e./ it could enter and remain in long backtracked pauses, as
has been observed for elongating complexes \cite{shaevitz_backtracking_2003}.
In the data we use for rate constant estimation, GreB was specifically
included in the experiments to avoid long backtracked pauses, and no such
pauses were reported \cite{revyakin_abortive_2006}. Additionally, it is likely
that the shortened RNA-DNA hybrid and strain from scrunching make backtracked
initial transcribing complexes far less stable than their elongating
counterparts. We therefore find it reasonable to assume that once backtracking
has commenced in initial transcription, an abortive RNA is eventually
released, which informs the calculation of the AP. The other possibility is
that abortive RNAs can be released instantly without backtracking via
hyper forward translocation. However, this has only been described for long
abortive transcripts (> 16 nt) in N25 promoter mutation variants
\cite{chander_alternate_2007, chander_mechanisms_2015}. Since native N25's
abortive RNAs are shorter than 12 nt \cite{chander_alternate_2007}, we assume
that hyper forward translocation does not take place and that backtracking is
the only mechanism leading to the release of an abortive RNA.

In this work we run simulations alternately using APs obtained for the
absence of GreB (-GreB) and in the presence of GreB (+GreB). When GreB is
present, we note that backtracked complexes may be rescued by GreB stimulated
cleavage of the unaligned 3\ppp end of the transcript. Therefore, APs obtained
in the presence of GreB represent the probability to both backtrack and to
avoid rescue by GreB until abortive RNA release. For comparison with -GreB
data, when RNA cleavage is unstimulated, we assume that GreB-mediated cleavage
and subsequent NACs are rapid steps compared to unscrunching. This permits
using the APs obtained in the presence of GreB as effective backtracking
probabilities. +GreB AP values used in this work are obtained from Figure 4B
in Hsu et al. \cite{hsu_initial_2006}, and -GreB AP values have been
calculated directly from raw data for Table 1 in Hsu et al.\
\cite{hsu_initial_2006} (data provided by Lilian M. Hsu). AP values in Hsu et
al.\ have been obtained in the presence of $100\ \mu M$ NTP.

\SUBSECTION{Implementation and rate constant estimation}
A central result in this work is determining the model rate constants. We
perform this estimation by running multiple simulations with different
combinations of rate constants and fitting to experimental data. The data used
for rate constant estimation is the distribution of time spent in abortive
cycling on the N25 promoter in the presence of $100\ \mu M$ NTP determined by
Revyakin et al.\ \cite{revyakin_abortive_2006}.

The experimental data was obtained from 100 individual initial transcription
events \cite{revyakin_abortive_2006}. For experiments with single molecules,
there is an inherent stochastic component in the experimental outcome that
derives from the randomness of molecular motion. That this is evident for
transcription can be inferred from the large variation in the speed of
transcription elongation observed in single-molecule experiments
\cite{adelman_single_2002, tolic-norrelykke_diversity_2004}. To account for
this randomness, we perform the kinetic simulations using the Gillespie
algorithm for stochastic simulations of chemical reactions
\cite{gillespie_exact_1977}. Specifically, we make use the version implemented
by Maarleveld et al.\ in the StocPY software \cite{maarleveld_stochpy:_2013}.

The procedure for rate constant estimation (illustrated in
\FIG~\ref{fig:param_estimation_scheme}) is as follows (rate constant names
indicated in \FIG~\ref{fig:model_and_rates}): First, we assign random values
to the three rate constants NAC ($k_n$), promoter escape ($k_e$) and
unscrunching and abortive RNA release ($k_u$) within certain fixed boundaries
(see below). These values are chosen independently from a uniform
distribution. Second, we use the rate constant of the NAC and APs to
calculate the rate constant of backtracking ($k_b$) using
Eq.~(\ref{eq:backtrackingcalc}). We then simulate 100 initial transcription
events and calculate the distribution of time spent in abortive cycling as a
result of these specific rate constants. The result is measured against the
empirical distribution \cite{revyakin_abortive_2006} using the root mean
square error. By repeating this procedure several times, we obtain statistics
of which values of rate constants that are associated with the best match with
the experimental data. By virtue of using stochastic simulations, one does not
arrive at a unique set of rate constants that have optimal fit with data; if a
simulation is repeated with identical rate constants, slightly different
results will be obtained due to the inherent stochasticity of the reaction
process. Therefore, we identify the best-fitting rate constants from the
fitness distribution, i.e./ we identify those rate constants that stand out
by most often giving the best fit to data.

The boundaries of the rate constants for parameter estimation were chosen from
extrema estimated from experimental findings. The speed of the NAC during initial
transcription cannot be much less than 3 nt/s, since Revyakin et al.\ measured
3.5 seconds as the shortest duration of abortive cycling
\cite{revyakin_abortive_2006} to cover the ~ 11 nts required for promoter
escape on N25. At the same time, it should not be larger than 25 nt/s,
which is above the upper limit of what has been measured for transcription
elongation \cite{bai_mechanochemical_2007}. While it is not clear how
transcription would proceed faster for promoter-bound RNAP, we use 25 nt/s as
a maximum value also in order to see if the model is able to discard these
extreme values during rate constant estimation. For backtracking and abortive
RNA release, it is known that the rate constant must be faster than 1
s$^{-1}$, since this was the time-resolution of experimental equipment which
could not resolve this value \cite{revyakin_abortive_2006}. To explore
possible values for all rate constants equally, we set the minimum and maximum
values of all rate constants to be 1 s$^{-1}$ and 25 s$^{-1}$, respectively. 

\begin{figure}[h]
  \caption{{\bf Kinetic scheme of initial transcription on the N25 promoter.}
    From the open complex (OC) transcription proceeds by NACs from one initial
    transcribing complex (ITC) to the next, where each ITC is identified in
    subscript by the length of its nascent RNA. Initial transcription proceeds
    until the nascent RNA has reached the experimentally obtained maximum size
    of abortive transcript; here 11 nt \cite{hsu_initial_2006}. For ITCs with
    an RNA of 2 nt in length or longer, there is a competition between the
    NAC and backtracking. Backtracking causes termination of transcription,
    and only further backtracking (unscrunching) and abortive RNA release may
    follow, returning RNAP to the open complex. From the open complex forward
    transcription may resume once more. The names of the rate constants are as
    follows: $k_n$ (NAC), $k_e$ (promoter escape), $k_u$ (unscrunching and
    abortive RNA release) and $k_{b,i}$ (backtracking).}
    \label{fig:model_and_rates}
\end{figure}

\begin{figure}[h]
  \caption{ {\bf Scheme of rate constant estimation protocol.} \textbf{1:}
    Rate constants for the NAC, UAR and promoter escape are randomly sampled
    from a uniform distribution (example values are shown). Of these values,
    the NAC is used further to obtain backtracking rates at each template
    position (\textbf{3}). This is done by solving
    Eq.~(\ref{eq:backtrackingcalc}), where the APs are obtained from Hsu et
    al.\ \cite{hsu_initial_2006} (\textbf{2}). The backtracking rate
    constants, together with the NAC, UAR, and promoter escape rate constants
    sampled in step \textbf{1}, encompass the complete kinetic scheme of
    initial transcription (\textbf{4}). This scheme is then used to simulate
    the kinetics of 100 initial transcription events; from these 100 events
    the distribution of time spent in abortive cycling is calculated and
    compared to measured data from Revyakin et al.\
    \cite{revyakin_abortive_2006} (\textbf{5}). From the distance between the
    measured distribution and the simulated distribution, a fitness score is
    produced. This score is associated to the three randomly selected rate
    constants in step \textbf{1}, and is a measure for how well the kinetic
    scheme (the three random rate constants and the AP values) agree with the
    experimental data. By repeating steps \textbf{1}-\textbf{5} multiple
    times, distributions are obtained from where we find which values of the
    rate constants provide the best fit with experimental data.}
    \label{fig:param_estimation_scheme}
\end{figure}
